\documentclass[spanish,a4paper]{book}
\usepackage{lmodern}
\usepackage{amssymb,amsmath}
\usepackage{ifxetex,ifluatex}
\usepackage{fixltx2e} % provides \textsubscript
\ifnum 0\ifxetex 1\fi\ifluatex 1\fi=0 % if pdftex
  \usepackage[T1]{fontenc}
  \usepackage[utf8]{inputenc}
\else % if luatex or xelatex
  \ifxetex
    \usepackage{mathspec}
  \else
    \usepackage{fontspec}
  \fi
  \defaultfontfeatures{Ligatures=TeX,Scale=MatchLowercase}
\fi
% use upquote if available, for straight quotes in verbatim environments
\IfFileExists{upquote.sty}{\usepackage{upquote}}{}
% use microtype if available
\IfFileExists{microtype.sty}{%
\usepackage{microtype}
\UseMicrotypeSet[protrusion]{basicmath} % disable protrusion for tt fonts
}{}
\usepackage[inner = 5cm, outer = 5cm, top = 2.5cm, bottom = 2.5cm]{geometry}
\usepackage{hyperref}
\hypersetup{unicode=true,
            pdftitle={Tesis},
            pdfauthor={Elio Campitelli},
            pdfborder={0 0 0},
            breaklinks=true}
\urlstyle{same}  % don't use monospace font for urls
\ifnum 0\ifxetex 1\fi\ifluatex 1\fi=0 % if pdftex
  \usepackage[shorthands=off,main=spanish]{babel}
\else
  \usepackage{polyglossia}
  \setmainlanguage[]{spanish}
\fi
\usepackage{longtable,booktabs}
\usepackage{graphicx,grffile}
\makeatletter
\def\maxwidth{\ifdim\Gin@nat@width>\linewidth\linewidth\else\Gin@nat@width\fi}
\def\maxheight{\ifdim\Gin@nat@height>\textheight\textheight\else\Gin@nat@height\fi}
\makeatother
% Scale images if necessary, so that they will not overflow the page
% margins by default, and it is still possible to overwrite the defaults
% using explicit options in \includegraphics[width, height, ...]{}
\setkeys{Gin}{width=\maxwidth,height=\maxheight,keepaspectratio}
\IfFileExists{parskip.sty}{%
\usepackage{parskip}
}{% else
\setlength{\parindent}{0pt}
\setlength{\parskip}{6pt plus 2pt minus 1pt}
}
\setlength{\emergencystretch}{3em}  % prevent overfull lines
\providecommand{\tightlist}{%
  \setlength{\itemsep}{0pt}\setlength{\parskip}{0pt}}
\setcounter{secnumdepth}{5}
% Redefines (sub)paragraphs to behave more like sections
\ifx\paragraph\undefined\else
\let\oldparagraph\paragraph
\renewcommand{\paragraph}[1]{\oldparagraph{#1}\mbox{}}
\fi
\ifx\subparagraph\undefined\else
\let\oldsubparagraph\subparagraph
\renewcommand{\subparagraph}[1]{\oldsubparagraph{#1}\mbox{}}
\fi

%%% Use protect on footnotes to avoid problems with footnotes in titles
\let\rmarkdownfootnote\footnote%
\def\footnote{\protect\rmarkdownfootnote}

%%% Change title format to be more compact
\usepackage{titling}

% Create subtitle command for use in maketitle
\newcommand{\subtitle}[1]{
  \posttitle{
    \begin{center}\large#1\end{center}
    }
}

\setlength{\droptitle}{-2em}
  \title{Tesis}
  \pretitle{\vspace{\droptitle}\centering\huge}
  \posttitle{\par}
\subtitle{Una tesis}
  \author{Elio Campitelli}
  \preauthor{\centering\large\emph}
  \postauthor{\par}
  \date{}
  \predate{}\postdate{}

\linespread{1.25}
\usepackage{subfig}
\usepackage{hyperref}
\usepackage{marginnote}
\usepackage[nomarkers,figuresonly]{endfloat}
\usepackage[spanish]{todonotes}
\usepackage{wrapfig}

\begin{document}
\maketitle

{
\setcounter{tocdepth}{3}
\tableofcontents
}
Resumen.

\todo[inline]{Listado de abreviaturas}
\todo[inline]{Revisar TODOS los epígrafes}

\chapter{Introducción}\label{introduccion}

\begin{itemize}
\tightlist
\item
  Antecedentes\\
  Además de lo que hay en lo de las becas + lo que fui encontrando,
  agregar sobre las climatologías disponibles y sus limitaciones.
\item
  Objetivo General
\item
  Objetivo particular
\end{itemize}

Esto es para probar una referencia bibliográfica: Vera et~al.
(\protect\hyperlink{ref-Vera2004}{2004}) y (Vera et~al.
\protect\hyperlink{ref-Vera2004}{2004})

\chapter{Métodos y Materiales}\label{metodos-y-materiales}

\todo[inline]{Agregar en algún lugar algo sobre las estadísticas circulares}

\section{Conceptos básicos}\label{conceptos-basicos}

\begin{itemize}
\tightlist
\item
  Ondas cuasiestacionarias
\item
  Fourier
\item
  wavelets
\item
  Flujo de actividad de onda.
\end{itemize}

\todo[inline]{chequear este paper: https://link.springer.com/article/10.1007/s00024-012-0635-9}

Ejemplo:

\begin{figure*}
\includegraphics[width=150mm, ]{fig/tesis/fourier-ejemplo-1} \caption{Ejemplo fourier - fig:fourier-ejemplo}\label{fig:fourier-ejemplo}
\end{figure*}

Cosas para ver de \autoref{fig:fourier-ejemplo}:\\
Descripción del ``rol'' de cada número de onda en generar el campo
final. La QS1 es la principal, marcando altas presiones al sur del
pacífico y bajas al sur de África. La onda 3 modifica ese patrón simple
haciendo que los máximos y mínimos no sean continuos.

\begin{itemize}
\tightlist
\item
  Wavelets
\end{itemize}

\begin{figure*}
\subfloat[Campo\label{fig:wavelet-ejemplo1}]{\includegraphics[width=75mm, ]{fig/tesis/wavelet-ejemplo-1} }\subfloat[Análisis de Wavelets en círculos de latitud señalados.\label{fig:wavelet-ejemplo2}]{\includegraphics[width=75mm, ]{fig/tesis/wavelet-ejemplo-2} }\caption{Wavelets - fig:wavelet-ejemplo}\label{fig:wavelet-ejemplo}
\end{figure*}

Cosas para ver:\\
Cambio en el máximo. Localización en vez de un número para cada latitud.

\section{Fuentes de datos}\label{fuentes-de-datos}

\section{Descripción de SPEEDY}\label{descripcion-de-speedy}

\chapter{Climatología observada}\label{climatologia-observada}

En esta sección se presentan campos medios y anomalías zonales de altura
geopotencial, temperatura, viento zonal, viento meridinal, gradiente
meridional de vorticidad absoluta y el número de onda estacionario, y
función corriente como introducción general al estado medio de la
atmósfera sobre el cual se desarrollan las ondas estacionarias. Luego se
analizan los campos de amplitud y varianza explicada por las ondas
cuasiestacionarias (QS) en sí mismas.

\section{Altura geopotencial}\label{altura-geopotencial}

Campo medio:

\begin{figure}

{\centering \includegraphics[width=227mm,angle=90]{fig/tesis/gh-ncep-1} 

}

\caption{Campo de Z (NCEP) - fig:gh-ncep}\label{fig:gh-ncep}
\end{figure}

El campo de altura geopotencial media (Z, \autoref{fig:gh-ncep}) muestra
una estructura más zonal en el HS que en el HN en todos los niveles y
estaciones. En verano el gradiente meridional de Z es máximo en 200hPa,
reduciéndose en 500hPa y por encima de 100hPa. En 50hPa el gradiente es
prácticamente nulo y en niveles superiores, éste se invierte en
comparación a los inferiores (no se muestra). En otoño el máximo de
gradiente todavía se da en 200hPa, pero continua siendo intenso en
niveles superiores. En invierno y primavera, el mayor gradiente se da en
50hPa y es mucho más intenso que los observados en los demás niveles o
estaciones. En contraste con el resto de los niveles, 50hPa y 100hPa
tienen mucha más variabilidad estacional.

El aumento del gradiente meridional de geopotencial en invierno y
primavera en niveles altos está relacionado con la generación del
vórtice polar que aisla las latitudes polares de las latitudes medias.
En 200hPa, en cambio, es evidencia del jet subtropical, que es más
estable a lo largo de todo el año.

\begin{figure}
\includegraphics[ ]{fig/tesis/ghdy-ncep-corte-1} \caption{Gradiente meridional de Z - fig:ghdy-ncep-corte - SÓLO BORRADOR}\label{fig:ghdy-ncep-corte}
\end{figure}

\begin{figure}

{\centering \includegraphics[width=227mm,angle=90]{fig/tesis/ghz-ncep-1} 

}

\caption{Anomalía zonal de altura geopotencial. - fig:ghz-ncep}\label{fig:ghz-ncep}
\end{figure}

Las anomalías zonales de geopotencial (Z*, \autoref{fig:ghz-ncep})
muestran una preponderancia de una onda 1 (QS1) con una amplitud máxima
en la estratósfera de primavera. Pueden diferenciarse dos QS1 distintas;
una centrada en \textasciitilde{}60°S y con el centro anticiclónico al
rededor de la línea de fecha, y la otra sobre la costa del continente y
el centro anticiclónico entre 120 y 60°O. Quintanar y Mechoso
(\protect\hyperlink{ref-Quintanar1995}{1995}) concluyó que el primero
está asociado principalmente con forzantes de latitudes bajas mientras
que el segundo responde a la orografía del continente antártico.

En latitudes tropicales, en verano hay una anomalía negativa sobre el
Pacífico este con máxima amplitud en 200hPa que está presente en las
otras estaciones con menor intensidad. Sobre Sudamérica, en verano y
primavera en ese mismo nivel aparece un centro anticiclónico con un
centro ciclónico al noroeste. Estas anomalías (la alta de Bolivia y la
baja del noroeste) son caracetrísticas del Sistema Monzónico
Sudamericano\todo{buscar cita}.

En la \autoref{fig:ghz-ncep-corte69} se muestra un corte zonal en 60°S
de Z*. Se aprecia la coherencia vertical de la QS1 y es evidente la
inclinación hacia el oeste con la altura en todas las estaciones salvo
en verano.

La inclinación hacia el oeste con la latitud (\autoref{fig:ghz-ncep}) y
con la altura (\autoref{fig:ghz-ncep-corte60}) indican que las
perturbaciones estacionarias están asociadas con transporte hacia el
polo tanto de cantidad de movimiento zonal como de temperatura. Como en
verano las anomalías zonales tienen una estructura barotrópica
equivalente y carecen de inclinación en la horizontal, los transportes
meridionales asociados se reducen considerablemente. \todo{¿Por qué?}

\begin{figure}

{\centering \includegraphics[width=130mm, ]{fig/tesis/ghz-ncep-corte60-1} 

}

\caption{Corte zonal de anomalía de geopotencial en -60°. - fig:ghz-ncep-corte60}\label{fig:ghz-ncep-corte60}
\end{figure}

\begin{figure}
\subfloat[u*v*\label{fig:uzvz-ncep-corte1}]{\includegraphics[ ]{fig/tesis/uzvz-ncep-corte-1} }\newline\subfloat[T*v*\label{fig:uzvz-ncep-corte2}]{\includegraphics[ ]{fig/tesis/uzvz-ncep-corte-2} }\caption{Transportes - fig:uzvz-ncep-corte - SÓLO BORRADOR}\label{fig:uzvz-ncep-corte}
\end{figure}

El ciclo anual de la amplitud de las ondas estacionarias se observa en
\autoref{fig:sd-gh-ncep}, donde se muestra el desvío estándar de Z por
círculo de latitud para cada mes
\todo{¿Explicar la relación de esto con la amplitud de la onda?}. La
mayor amplitud se encuentra centrado al rededor de los 60°S, como ya se
vio en la \autoref{fig:ghz-ncep} y alcanza su máximo entre agosto y
octubre, y el mínimo entre febrero y abril, según el nivel.

\begin{figure}

{\centering \includegraphics[width=130mm, ]{fig/tesis/sd-gh-ncep-1} 

}

\caption{Desvío estándar por círculo de latitud. - fig:sd-gh-ncep}\label{fig:sd-gh-ncep}
\end{figure}

\section{Temperatura}\label{temperatura}

\begin{figure}

{\centering \includegraphics[width=227mm,angle=90]{fig/tesis/t-ncep-1} 

}

\caption{Temperatura media. - fig:t-ncep}\label{fig:t-ncep}
\end{figure}

Pasando a la temperatura, en la \autoref{fig:t-ncep} puede verse que, al
igual que el campo de geopotencial, tiene una estructura principalmente
zonal en todos los niveles y estaciones. Por debajo de los 200hPa, donde
el gradiente meridional de temperatura es mínimo, la temperatura
disminuye con la latitud en todas las estaciones. Por encima de este
nivel, en cambio, en verano la temperatura crece con la latitud, y en el
resto de las estaciones muestra un máximo centrado en 60°S en otoño y en
45°S en invierno y primavera. El nivel de 850hPa se ve particularmente
deformado por los contrastes de temperatura zonales como el máximo sobre
Australia en verano.

\begin{figure}

{\centering \includegraphics[width=130mm, ]{fig/tesis/t-ncep-corte-1} 

}

\caption{Corte meridional de temperatura media. - fig:t-ncep-corte}\label{fig:t-ncep-corte}
\end{figure}

Haciendo un promedio zonal como se muestra en la
\autoref{fig:t-ncep-corte}, se observa un mínimo de temperatura se da en
la estatósfera tropical durante todo el año y uno en la estatósfera
polar con amplitud, extensión y altura máxima en invierno y mínima en
verano. El máximo relativo de tempertura observado por encima de 100hPa
se da como consecuencia de estos dos mínimos.

\begin{figure}

{\centering \includegraphics[width=227mm,angle=90]{fig/tesis/tz-ncep-1} 

}

\caption{Anomalía zonal de temperatura. - fig:tz-ncep}\label{fig:tz-ncep}
\end{figure}

En la \autoref{fig:tz-ncep} se muestran las anomalías zonales de
temperatura. En 850hPa se aprecia mejor el efecto del contraste de
temperatura entre el suelo y el mar que en la \autoref{fig:t-ncep}. Se
observan anomalías positivas sobre los continentes y negativas sobre los
océanos en todas las estaciones, aunque más intensas en verano y
primavera. En niveles más altos éstas pierden su intensidad pero
reaparecen en 100hPa con signo invertido. Estas características tienen
su correlato en la altura geopotencial y corresponden a circulaciones
tipo Walker \todo{¿Es así?}.

En invierno y primavera, los niveles altos están dominados por una QS1
con máximo en el sur de Australia y mínimo en el Atlántico sur. En
niveles más bajos, la onda disminuye su amplitid y se defasa hacia el
este y queda casi en cuadratura (\autoref{fig:t-ncep-corte60}) con el
máximo en 850hPa en Antártida occidental. En otoño esta onda está
presente pero con amplitud muy reducida y máxima en niveles medios.
Finalmente, en verano ésta desaparece por encima de 100hPa.

\begin{figure}

{\centering \includegraphics[width=130mm, ]{fig/tesis/t-ncep-corte60-1} 

}

\caption{Corte zonal de anomalía de temperatura en -60°. - fig:t-ncep-corte60}\label{fig:t-ncep-corte60}
\end{figure}

\section{Viento zonal}\label{viento-zonal}

\begin{figure}

{\centering \includegraphics[width=130mm, ]{fig/tesis/u-ncep-corte-1} 

}

\caption{Viento zonal medio. - fig:u-ncep-corte}\label{fig:u-ncep-corte}
\end{figure}

La media zonal del viento zonal (\autoref{fig:u-ncep-corte}) muestra
sendos máximos en latitudes medias en 200hPa y plares en la estatósfera,
correspondientes al jet subtropical y subpolar respectivamene. El
primero está presente durante todo el año aunque con mayor intensidad y
corrido hacia latitudes más ecuatoriales en invierno y primavera. El
segundo está presente principalmente en invierno y primavera, e
incipiente en otoño. En la estratósfera se observan vientos del este en
latitudes bajas que son más intensos en verano y otoño.

\begin{figure}

{\centering \includegraphics[width=227mm,angle=90]{fig/tesis/u-ncep-1} 

}

\caption{Viento zonal. - fig:u-ncep}\label{fig:u-ncep}
\end{figure}

En la \autoref{fig:u-ncep} se observa que el jet subpolar es más intenso
al sur de África, donde además se encuentra en una latitud más
ecuatorial que en la región del Pacífico. El jet subtropical también
tiene un máximo al sur de África y otro al norte de Nueva Zelanda
--especialmente en invierno--, donde además se produce una bifurcación
del jet. Se trata de una región de persistentes y frecuentes bloqueos
(Trenberth y Mo \protect\hyperlink{ref-Trenberth1985}{1985}).

Esta bifurcación del jet sobre Nueva Zelanda se evidencia en el campo de
anomalías zonales de viento zonal (\autoref{fig:uz-ncep}) como anomalías
negativas durante todo el año. Este campo también presenta varios pares
de QS1 antisimétricos respecto a 60°S. Estas anomalías se corresponden
con la variación meridional del jet observado en la \autoref{fig:u-ncep}
y son consistentes con la QS1 de geopotencial observada en la
\autoref{fig:ghz-ncep}, las cuales generan viento zonal anómalo del este
y del oeste al norte y sur del anticiclón respectivamente.

En verano, entre 300hPa y 100hPa sobre el Pacífico ecuatorial existe una
zona de anomalías del viento zonal positivas al este y negativas al
oeste. Consistente con los campos de temperatura, esto implica
divergencias en niveles altos y convergencias en niveles bajos (no se
muestra). Evidencia de la circulación tropical forzada por la
temperatura superficial del pacífico.\todo{buscar cita.}

\begin{figure}

{\centering \includegraphics[width=227mm,angle=90]{fig/tesis/uz-ncep-1} 

}

\caption{Anomalía zonal de viento zonal. - fig:uz-ncep}\label{fig:uz-ncep}
\end{figure}

Los campos de viento zonal, temperatura y altura geopotencial están
ligados por el balance de viento geostrófico y de viento térmico. El
máximo del jet se encuentra en regiones de máxima baroclinicidad y
máximo gradiente meridional de Z y donde el gradiente meridional de
temperatura se anula.

\section{Viento meridional}\label{viento-meridional}

\begin{figure}

{\centering \includegraphics[width=130mm, ]{fig/tesis/v-ncep-corte-1} 

}

\caption{Media zonal del viento meridional. - fig:v-ncep-corte}\label{fig:v-ncep-corte}
\end{figure}

En los campos medios zonales de viento meridional
(\autoref{fig:v-ncep-corte}) la principal característica descable son
los máximos tropicales presentes en superficie y altura en todas las
estaciones. Se trata de la circulación de Hadley. En verano, la rama
ascendente de encuentra en el hemisferio sur y se tiene convergencias en
niveles bajos y divergencias en niveles altos. En invierno, en cambio,
sólo se ve la rama descendente, mucho más intensa que en verano, que
genera convergencias en niveles altos y divergencias en niveles bajos al
rededor de los 30°S.

Presente durante todo el año, también se observa un maximo de vientos
del sur en la costa antártica. Esta es la señal de los vientos
catabáticos antárticos producidos por una capa muy estable cerca de
superficie y la consistente inclinación de la topografía del
continente.\todo{cita para esto}.

En el resto del dominio, la media zonal es prácticamente nula, lo cual
implica que la anomalía zonal de V es prácticamente idéntica al campo
total; por lo tanto, sólo se muestra éste último.

\begin{figure}

{\centering \includegraphics[width=227mm,angle=90]{fig/tesis/v-ncep-1} 

}

\caption{Viento meridional medio. - fig:v-ncep}\label{fig:v-ncep}
\end{figure}

Los campos horizontales de V se muestran en la \autoref{fig:v-ncep}.
Consistente con los campos de Z* (\autoref{fig:ghz-ncep}), en niveles
altos se observa una QS1 que alcanza su máximo en la estratósfera de
primavera.

En invierno entre 500hPa y 100hPa, existe evidencia de un tren de ondas
de Rossby que se propaga desde el Índico occidental sudeste llegando a
su máxima latitud en 150°O donde comienza a propagarse hacia el norte
hasta llegar al sur de Sudamérica. Este tren de ondas puede
identificarse en el campo de Z* (\autoref{fig:ghz-ncep}), pero con mayor
dificultad debido a la interferencia de la QS1.

También en invierno (verano del HN), en los trópicos se puede observar
el viento meridional hacia el norte en 850hPa en la costa oeste de
África asociado al monzón de la India. El monzón sudamericano, por su
parte, se evidencia en verano por un aumento de la intensidad de los
vientos del norte sobre el continente americano. En altura, el monzón de
la India se muestra como viento hacia el sur producto de la divergencia
de niveles altos generada por la convección anómala. El Sudamérica, la
alta de Bolivia genera viento hacia el norte y hacia el sur al este y al
oeste de Bolivia respectivamente.

\begin{figure}

{\centering \includegraphics[width=227mm,angle=90]{fig/tesis/ghminus1-ncep-1} 

}

\caption{Z* menos QS1. - fig:ghminus1-ncep - SÓLO BORRADOR}\label{fig:ghminus1-ncep}
\end{figure}

\section{Gradiente meridional de vorticidad
absoluta}\label{gradiente-meridional-de-vorticidad-absoluta}

En la teoría de propagación meridional de ondas de Rossby, el gradiente
meridional de vorticidad absoluta (\(\psi_y\))
\todo[inline]{es importante porque blavalbla ¿Resumen de la deducción en James?}

\begin{figure}

{\centering \includegraphics[width=227mm,angle=90]{fig/tesis/etady-ncep-1} 

}

\caption{Gradiente meridional de vorticidad absoluta * 1e11 - fig:etady-ncep}\label{fig:etady-ncep}
\end{figure}

La \autoref{fig:etady-ncep} muestra que el campo de \(\psi_y\) tiene dos
regiones con valores negativos. La primera, centrada sobre Nueva
Zelanda, se da en invierno en 300hPa y 200hPa coincidiendo con la región
de bloqueos. En invierno está franqueada por el jet intenso, dando lugar
a gradientes meridionales de U negativos más intensos que \(\beta\). La
segunda se da en algunas regiones de la costa Antártica.

Estas zonas de \(\psi_y\) negativo forman una \emph{región prohibida}
que impide la propagación meridional de ondas de Rossby. Esta figura
reproduce y extiende el resultado de Berbery, Nogués-Paegle, y Horel
(\protect\hyperlink{ref-Berbery1992}{1992}) (su Figura~3) utilizando 5
años de análisis objetivo del Centro Europeo de Predicción a Plazo Medio
(ECMWF). Aún con \(\psi_y\) positivo, las ondas de Rossby sólo se pueden
propagar si su número de onda zonal es menor que el número de onda
estacionario (James \protect\hyperlink{ref-James}{1994}). En la
\autoref{fig:ks-ncep} se muestra el número de onda estacionario para el
nivel de 200hPa y en la \autoref{fig:ks-ncep-corte} se muestra un corte
en 180°. Además de la \emph{región prohibida}, las ondas cortas no
pueden propagarse meridionalmente en latitudes altas. Las ondas largas
con k \textless{} 3 pueden propagarse meridionalmente libremente en
verano al norte de los 60°S aproximadamente, pero quedan atrapadas al
sur de 45°S en toda una franja de longitudes entre 60°E y 120°O durante
el resto del año.

El corte de \(\psi_y\) en 180° evidencia que la QS3 sólo puede
propagarse meridionalmente dentro de una franja angosta de latitudes,
entre 60°S y 50°S. Una región donde este modo muestra un máximo de
amplitud (\autoref{fig:ampl-ncep}).

\begin{figure}
\includegraphics[ ]{fig/tesis/ks-ncep-1} \caption{Número de onda estacionario en 200hPa. - fig:ks-ncep}\label{fig:ks-ncep}
\end{figure}

\begin{figure}
\includegraphics[ ]{fig/tesis/ks-ncep-cortelev-1} \caption{Número de onda estacionario en 180° - fig:ks-ncep-cortelev - SÓLO BORRADOR}\label{fig:ks-ncep-cortelev}
\end{figure}

\section{Función corriente}\label{funcion-corriente}

\begin{figure}

{\centering \includegraphics[width=227mm,angle=90]{fig/tesis/psi-ncep-1} 

}

\caption{Función corriente x 1099 - fig:psi-ncep}\label{fig:psi-ncep}
\end{figure}

Para analizar la influencia tropical en la circulación no es posible
usar la altura geopotencial, ya que el balance geostrófico pierde
validez cerca del ecuador. Por lo tanto, es útil analizar el campo de
función corriente (\(\psi\)). El reanálisis de NCEP provee esta variable
en niveles sigma en vez de presión. En la \autoref{fig:psi-ncep} se
muestra \(\psi\) en el nivel 0.2101 sigma, que equivale aproximadamente
a 250hPa. Además del campo medio, se muestran las anomalías zonales en
sombreado y los flujos de actividad de onda en flechas.

Las características del campo de \(\psi\), tanto el total como las
anomalías zonales, es similar al de Z (\autoref{fig:gh-ncep}, 200hPa)
con una estructura eminentemente zonal y un aumento del gradiente
meridional en invierno y primavera y los mismos centros de anomalías. La
principal diferencia es, además del cambio de signo dada por la
dependencia de \(\psi\) con f, es que los patrones presentes en las
latitudes tropicales se ven con mayor magnitud que los de latitudes
medias y altas.

Los flujos de actividad de onda en verano muestran transporte de energía
desde el Pacífico este hacia el sur de África pasando por el Atlántico
que se sostiene durante todo el año con menor intensidad. Desde ese
lugar también se observa transferencia de energía hacia el hemisferio
norte, que se junta con otra región de flujos intensos que viene desde
el Pacífico oeste. Sobre el Índico, los flujos son de mayor magnitud en
invierno transportando energía hacia el sur.

\section{Ondas Quasiestacionarias}\label{ondas-quasiestacionarias}

Como se describió en \autoref{métodos-y-materiales}, el análisis
estacional de la amplitud y \(r^2\) de Fourier puede hacerse a partir de
la media delos campos mensuales o aplicando Fourier a los campos
estacionales. En el caso de los datos de NCEP existe poca diferencia,
por lo que sólo se muestran los resultados conseguidos mediante esta
última metodología.

El campo de Z* está caracterizado principalmente por un patrón de QS1 en
altas latitudes latitudes (\autoref{fig:ghz-ncep}). No es sorprendente,
entonces, que la QS1 explica la mayor parte de la variabilidad del
geopotencial en virtualmente todo el dominio al sur de los
45°S(\autoref{fig:r2-ncep}). La QS2 es preponderante en la estratósfera
ecuatorial, en la costa antártica y alrededor de 35°S, donde es el modo
dominante en toda la columna de aire en verano. La QS3, a diferencia de
las ondas anteriores es importante en una región reducida. Explica una
parte substancial de la varianza en niveles bajos al rededor de los 45°S
y mayormente en invierno. La QS4 explica muy poca varianza a excepción
de cerca de superficie entre 15°S y 30°S. Ondas más cortas son aún menos
importantes (no se muestra).

\begin{figure}

{\centering \includegraphics[width=227mm,angle=90]{fig/tesis/r2-ncep-1} 

}

\caption{$R^2$ de Fourier. - fig:r2-ncep}\label{fig:r2-ncep}
\end{figure}

El \(r^2\) permite analizar la importancia relativa de cada modo con
respecto a la variabilidad total, pero lo que importa desde el punto de
vista físico es la amplitud de la onda
\todo{Esto está muy mal escrito, hay que mejorar}. Las diferencias entre
los campos de \(r^2\) y los de amplitud son evidentes comparando las
figuras \ref{fig:r2-ncep} y \ref{fig:ampl-ncep} (notar la escala
logarítmica en los colores). La amplitud de la QS1 muestra un máximo
bien definido centrado en 60°S que en verano se encuentra en niveles más
bajos que en las otras estaciones. También existe un máximo relativo
entre 15°S y 30°S en verano que migra a latitudes más altas en invierno
y primavera. El mismo está presente también en las otras ondas
estacionarias.

En el caso de la QS2, se evidencia que a pesar de tener máximos de
\(r^2\) en la estratósfera al norte de 45°S, alcanza su máxima amplitud
al sur de esa latitud y en 200hPa en verano y en 30hPa en invierno. Su
actividad en la costa antártica se extiende en toda la tropósfera en
invierno (a pesar de que en \(r^2\) pierde importancia por encima de los
200hPa)

La región de amplitud máxima de la QS3, coincide aproximadamente con la
región de máximo \(r^2\) entre y otoño y primavera, aunque con menos
actividad en superficie y extensión en toda la columna. En verano, en
cambio aparece un máximo de amplitud importante que no se observa en el
campo de \(r^2\).

Finalmente, fuera de la superficie, la QS4 presenta un máximo de
amplitud bien definido sólo en verano. El máximo entre 15°S y 30°S sigue
presente.

\begin{figure}

{\centering \includegraphics[width=227mm,angle=90]{fig/tesis/ampl-ncep-1} 

}

\caption{Ampllitud de Fourier. - fig:ampl-ncep}\label{fig:ampl-ncep}
\end{figure}

\todo[inline]{Acá podría poner alguna reflexión general de lo que se ve. Por ejemplo, cómo en verano la variabilidad está más acotada a la tropóstera mientras que en invierno y primavera hay más contacto con la estratósfera.}

\chapter{Onda 3}\label{onda-3}

\todo[inline]{introducción a QS3; decir que esto es una onda reconstruida a partir de Fourier}

\section{Características típicas}\label{caracteristicas-tipicas}

\begin{figure*}
\includegraphics[width=150mm, ]{fig/tesis/qs3-ncep-1} \caption{Media de reconstrucción de onda 3. - fig:qs3-ncep}\label{fig:qs3-ncep}
\end{figure*}

En la \autoref{fig:qs3-ncep} se muestra el campo de geopotencial
reconstruido sólo a partir de la QS3 en 300 hPa, que ilustra lo que
sucede en todos los niveles dado que su estructura es barotrópica
equivalente (\autoref{fig:qs3-ncep-corte}. Se reproducen las
características de la amplitud ya vistas en la \autoref{fig:ampl-ncep}
(máximo entre 60°S y 45°S con menor intensidad en primavera, mayor
amplitud en 300hPa con mayor desarrollo vertical en otoño y verano) pero
además permite observar el efecto de la variación de la fase. Se observa
que existe un corrimiento de la fase entre verano e invierno de poco más
de 15° (algo ya observado por
??\todo{buscar referencia. Raphael referencia este movimiento, pero no lo encuentro en los papers que cita.})
anticipando que el efecto de la QS3 sobre cada lugar \todo{mal escrito}
pueda tener una componente estacional.

La inclinación meridional de las perturbaciones de geopotencial
asociadas a la QS3 sugieren que ésta está asociada a transportes de
cantidad de movimiento hacia el polo en las estaciones de transición,
pero en menor medida en verano.

\begin{longtable}[]{@{}lr@{}}
\caption{Transporte meridinoal de cantidad de viento zonal NO SE
PUBLICA}\tabularnewline
\toprule
Estación & {[}u*v*{]}\tabularnewline
\midrule
\endfirsthead
\toprule
Estación & {[}u*v*{]}\tabularnewline
\midrule
\endhead
Verano & -0.100\tabularnewline
Otoño & -0.431\tabularnewline
Invierno & -0.016\tabularnewline
Primavera & -0.163\tabularnewline
\bottomrule
\end{longtable}

\begin{figure*}
\includegraphics[width=150mm, ]{fig/tesis/qs3-ncep-corte-1} \caption{Corte - fig:qs3-ncep-corte}\label{fig:qs3-ncep-corte}
\end{figure*}

\begin{figure*}
\includegraphics[width=150mm, ]{fig/tesis/qs3sd-ncep-1} \caption{Desvío estándar de la reconstrucción de QS3. - fig:qs3sd-ncep}\label{fig:qs3sd-ncep}
\end{figure*}

En la \autoref{fig:qs3sd-ncep} se muestra el desvío estándar de la
reconstrucción de la QS3. La variabilidad máxima se da entre los centros
ciclónicos y anticiclónicos (marcados en la \autoref{fig:qs3sd-ncep} con
contornos negros), indicando que la variabilidad del geopotencial
asociada a la QS3 se debe al corrimiento de la fase.

\todo[inline]{hasta acá revision}

\subsection{Wavelets}\label{wavelets}

\begin{figure}

{\centering \includegraphics[width=130mm, ]{fig/tesis/wavelet-fourier-ncep-1} 

}

\caption{Amplitud de wavelets (sombreados) y de fourier (contornos) - fig:wavelet-fourier-ncep}\label{fig:wavelet-fourier-ncep}
\end{figure}

Para la QS3, la amplitud media obtenida mediante Wavelets
\todo{agregar referencia a sección} es virtualmente idéntica a la
amplitud obtenida con Fourier (\autoref{fig:wavelet-fourier-ncep}). Sin
embargo, también permite obtener información de la variación meridional
de la amplitud de la QS3. Al igual que con Fourier, esto puede hacerse
promediando estacionalmente la amplitud de los campos mensuales o
calculando la amplitud de los campos estacionales. Pero a diferencia de
Fourier, los resultados de cada metodología son opuestos. Esto es
evidente en la \autoref{fig:waveletz-ncep} donde se muestra el campo de
amplitud de la QS3 según wavelets en líneas y su anomalía zonal en
sombreado. Los valores positivos representan regiones donde la amplitud
de la QS3 es mayor que la media zonal y viceversa.

No existe mucha diferencia en la amplitud media zonal entre ambas
metodologías a excepción de que la primera siempre es mayor que la
segunda
\todo{Esta es una propiedad general que SIEMPRE se cumple. En realidad podría ir en metodología}.
Las principales diferencias se dan en las anomalías zonales.

\begin{figure}

{\centering \includegraphics[width=227mm,angle=90]{fig/tesis/waveletz-ncep-1} 

}

\caption{Campo medio de la amplitud de la onda 3 según wavelets (contornos) y su anomalía zonal (sombreado) en 300hPa. - fig:waveletz-ncep}\label{fig:waveletz-ncep}
\end{figure}

Las anomalías zonales presentan, en todas las estaciones, un máximo al
sur del Índico centrado en 45°S que se desplaza hacia el este en
latitudes más altas. Existe un segundo máximo en altas latitudes que en
otoño y primavera se encuentran en 120°E y en invierno se encuentra en
180°. En verano éste no aparece, pero sí existe un máximo centrado en
15°S y 120°O.

La principal diferencia con la media de la amplitud es que las mayores
anomalías zonales se dan en latitudes altas (al rededor de 60°S) con un
mínimo al sur del Índico en vez de un máximo. Al norte de 30°S, las
diferencias son menores.

\todo{Estoy pensando en no poner la \autoref{fig:wavelet-ncep-corte} porque no encuentro mucho para decir y no veo que se gane demasiado. Queda para el doctorado? :P}

\begin{figure}

{\centering \includegraphics[width=227mm,angle=90]{fig/tesis/wavelet-ncep-corte-1} 

}

\caption{Corte zonal en -60° de la amplitud media de la onda 3 según wavelets (contornos) y su anomalía zonal (sombreado). - fig:wavelet-ncep-corte}\label{fig:wavelet-ncep-corte}
\end{figure}

Cosas para ver:\\
* Si bien el máximo medio de la amplitud se da en
\textasciitilde{}300hPa (\autoref{fig:wavelet-fourier-ncep}) igual que
en fourier, el análisis por longitud muestra algo un poco más complejo.
En verano y otoño, la máxima amplitud sigue en 300hPa, pero ésta
asciende a entre 100 y 50hPa en invierno y primavera alrededor de 120°O.
Además, durante estas estaciones hay indicación de que el máximo cambia
de latitud con la altura.

\begin{figure}
\subfloat[campo\label{fig:estacionaridad1}]{\includegraphics[ ]{fig/tesis/estacionaridad-1} }\newline\subfloat[media zonal\label{fig:estacionaridad2}]{\includegraphics[ ]{fig/tesis/estacionaridad-2} }\caption{Ratio de amplitud de la media y media de la amplitud (¿medida de estacionaridad?) - fig:estacionaridad - SÓLO BORRADOR}\label{fig:estacionaridad}
\end{figure}

Las diferencias vistas en la \autoref{fig:waveletz-ncep} están
relacionadas con la estacionariedad\todo{¿Es una palabra?} de las ondas
\todo{de nuevo, esto es general para la metodología, la explicación va en métodos},
se puede utilizar la relación entre ambas para generar un \emph{índice
de estacionariedad} \todo{¿Estoy seguro de que esto es así?}. Cuanto más
similar sea el resultado, más estacionarias son las ondas. Así, la
estacionariedad sería máxima en latitudes bajas y al rederor de 50°S
--principalmente en verano-- y mínima en dos franjas angostas cerca de
30°S y 75°S. \todo{¿Mostrar el gŕafico \ref{fig:estacionaridad}?}
\todo[inline]{Esta discusión también aplica a fourier (\autoref{fig:estacionaridad-fourier}), en realidad, pero ahí no estoy mostrando ambos resultados porque da parecido. Quizás este párrafo se puede mover.. o borrar todo si resulta puro sinsentido.}

\begin{figure}
\includegraphics[ ]{fig/tesis/estacionaridad-fourier-1} \caption{Estacionaridad según fourier - fig:estacionaridad-fourier - SÓLO BORRADOR}\label{fig:estacionaridad-fourier}
\end{figure}

Estas observaciones destacan la utilidad de wavelets en el análisis de
ondas cuasiestacionarias. Mientras que el tratamiento con fourier asume
a las ondas como una propiedad media de cada círculo de latitud,
wavelets permite reconocer su heterogeneidad meridional. Evaluando esta
heterogeneidad, sería posible distinguir entre campos donde una onda con
un determinado número de onda está presente en todo un círculo de
latitud de campos donde ésta está localizada en una región acotada.

Por otro lado, la no ortogonalidad de los wavelets complejizan la
interpretación de los resultados ya que no es posible la separación de
un campo en modos oscilatorios con distinto número de onda. El análisis
de una QS específica, por lo tanto está contaminado por la actividad de
otras QS con longitud de onda cercana.

\begin{figure}
\includegraphics[ ]{fig/tesis/wavelet-reconstr-1} \caption{Reconstrucción de QS3 usando wavelets - fig:wavelet-reconstr - SÓLO BORRADOR}\label{fig:wavelet-reconstr}
\end{figure}

Wavelets, en resumen, puede entenderse como una \emph{corrección} a
fourier que agrega información de asimetrías zonales. Dado que la
variabilidad zonal es del orden de un 10\% de la amplitud media, en lo
que sigue de la tesis se utilizará sólo fourier, dejando el análisis e
interpretación de wavelets para futuros trabajos.

\section{Antecedentes}\label{antecedentes}

Breve comentario sobre los índices usados en otros lados. Discutir
ventajas y debilidades.

\begin{itemize}
\tightlist
\item
  Amplitud
\item
  Fase (impacto en SA)
\end{itemize}

De todo eso, motiva decisión del índice.

\begin{itemize}
\tightlist
\item
  Niveles elegidos
\item
  Promedio vs.~máximo
\item
  Composiciones de campos y flujos.
\item
  Decisión del índice.
\end{itemize}

Quiero hacer el índice a partir de la actividad de la onda 3 tomando la
región del máximo (latitud entre -65 y -40, y entre 700 y 100 hPa).
Variables posibles: amplitud media, amplitud máxima, r2, correlación
entre campo teórico y observado.

\section{Amplitud}\label{amplitud}

\subsection{Máximo o media.}\label{maximo-o-media.}

Existen varios estadísticos que podrían utilizarse para representar la
amplitud de la QS3 en una región extendida. La media, la máxima, la
moda, la mediana, etc\ldots{} En este caso, se estudió la posibilidad de
representarla con la media o la
máxima.\todo{esto fue escrito a las 23:30 luego de un largo día de mirar campos y números... es un delirio.}

\begin{figure*}
\includegraphics[width=150mm, ]{fig/tesis/ampl-max-mean-1} \caption{Distribució de amplitud para 12 fechas. En rojo la amplitud máxima, en azul la amplitud media. - fig:ampl-max-mean}\label{fig:ampl-max-mean}
\end{figure*}

\begin{figure*}
\includegraphics[width=150mm, ]{fig/tesis/ampl-max-mean-corte-1} \caption{Corte vertical de amplitud - fig:ampl-max-mean-corte}\label{fig:ampl-max-mean-corte}
\end{figure*}

\begin{figure*}
\includegraphics[width=150mm, ]{fig/tesis/ghz-ncep-select-1} \caption{Anomalía zonal geopotencial en 300hPa para fechas seleccionadas. - fig:ghz-ncep-select}\label{fig:ghz-ncep-select}
\end{figure*}

Se seleccionaron manualmente 9 casos que representan distintas
características de la amplitud media la máxima. Sus valores se muestran
en la \autoref{fig:ampl-max-mean}, los cortes meridionales de amplitud,
en la \autoref{fig:ampl-mac-mean-corte} y los campos de Z* (con las QS1
y QS2 restadas) en la \autoref{fig:ghz-ncep-select}.

Comparando mayo de 1997 con abril de 2012, ambos tienen una media muy
similar, pero la máxima del primero es menor que la del segundo.
Observando el campo de geopotencial, la QS3 se aprecia mucho más
claramente en 2012 que en 1997.

Enero de 1985 y julio de 1988 son un caso similar en cuanto a la
relación de las métricas de amplitud, pero en este
caso\todo{repite "caso"} ambos campos de Z* son muy similares en cuanto
a intensidad y claridad de la QS3. Los dos meses presentan un tren de
ondas que ocupan aproximadamente 1/3 de círculo de latitud. A pesar de
que la amplitud máxima de 1985 es menor que la de 1988, el tren de ondas
de 1988 se ve algo más claro que el de 1985.

El par septiembre de 2000 y diciembre de 1990 es más claro. Ambas
medidas de amplitud son mayores en 1990 y, en efecto, el campo de Z*
tiene una estructura de QS3 zonal más clara que el de 2000. Sin embargo,
las anomalías sí están presentes en 1990 --un tren de ondas similar al
de enero de 1985, aunque con distinta fase-- son más intensas, por lo
que su efecto local puede ser mayor que las de 2000.

Más extrema aún es la diferencia entre septiembre de 2000 y octubre de
2003. Ambos meses tienen una métrica de amplitud similares, pero la QS3
es apenas distinguible en el segundo mes.

Estos casos ilustran algunas limitaciones del análisis que continua.
Algunos problemas son inherentes al intentar representar una estructura
con variabilidad espacial a partir de un sólo número y otros están
atados a la limitación de la descomposición de Fourier que trata toda
onda como una onda
planetaria.\todo{Aclarar qué comparación ilustra qué problema}

\begin{figure}
\includegraphics[ ]{fig/tesis/cor-mean-max-1} \caption{Correlación entre amplitud máxima y media.{fig:cor-mean-max}}\label{fig:cor-mean-max}
\end{figure}

Es importante tener en cuenta que estos 9 casos fueron seleccionados
específicamente para ilustrar estas limitaciones y que no son
necesariamente representativos de la totalidad de casos posibles. Como
se muestra en la \autoref{fig:cor-mean-max}, la amplitud media máxima
tienen una excelente correlación (\(r^2>0.9\)) y una relación lineal.
Debido a esto, a fines estadísticos la elección de una métrica o la otra
no tiene una influencia importante. Desde este momento se usará la
amplitud media\todo{expresar mejor esto}.

El ciclo anual de la amplitud media se muestra en la
Figura~\ref{fig:ampl-ts1} donde los puntos son datos puntuales. La
amplitud máxima en invierno es evidente, así como la mayor variabilidad.
Esto es de esperarse en una variable definida positiva que toma valores
cercanos a cero. Notar que la amplitud media no es mínima en primavera,
en contraste con lo observado en el análisis climatológico
(\autoref{fig:ampl-ncep}) y los campos reconstruidos
(\autoref{fig:qs3-ncep}). La resolución a este problema radica en el
análisis de la fase (\autoref{fase}).

\begin{figure*}
\subfloat[Ciclo anual\label{fig:ampl-ts1}]{\includegraphics[width=150mm, ]{fig/tesis/ampl-ts-1} }\newline\subfloat[Serie temporal\label{fig:ampl-ts2}]{\includegraphics[width=150mm, ]{fig/tesis/ampl-ts-2} }\caption{Amplitud media - fig:ampl-ts}\label{fig:ampl-ts}
\end{figure*}

La serie temporal de la amplitud media se muestra en la
Figura~\ref{fig:ampl-ts2} donde en líneas horizontales se marca la
amplitud media anual para cada año indicando en color rojo o azul si
dicho valor está por encima o debajo de la media de la serie
respectivamente. Se observan series de años con anomalías positivas
consistentemente seguidos por anoamlías negativas (1985-1990, 1992-1996
y 1999-2005) y otras con persistencia de anomalías positivas o negativas
(2005-2009 y 2012-2015). No hay evidencia visual de periodicidades ni de
una tendencia secular.

\begin{figure}
\includegraphics[ ]{fig/tesis/wavelet-period-1} \caption{Análisis de wavelet para la amplitud de la onda 3. - fig:wavelet-period - SÓLO BORRADOR}\label{fig:wavelet-period}
\end{figure}

\begin{figure}
\includegraphics[ ]{fig/tesis/acf-ampl-1} \caption{Función de autocorrelación para la amplitud media anual de la onda 3 - fig:acf-ampl - SÓLO BORRADOR}\label{fig:acf-ampl}
\end{figure}

\section{Fase}\label{fase}

Además de la amplitud, las sondas planetarias se caracterizan por su
fase. \todo[inline]{intro sobre la fase}

\begin{figure}

{\centering \includegraphics[width=227mm,angle=90]{fig/tesis/fase-boxplot-1} 

}

\caption{Ciclo anual de la fase (20 mayores amplitudes para cada mes) - fig:fase-boxplot}\label{fig:fase-boxplot}
\end{figure}

La \autoref{fig:fase-boxplot} muestra la fase media (localización media
del máximo de geopotencial) y el rango delimitado por \(\pm\) 1 desvío
estándar para cada mes para los 20 años con más amplitud de cada mes. En
rojo y azul se indican la localización de los centros de máxima y mínima
anomalía de geopotencial respectivamente de los 10 casos más extremos.
El mapa se muestra para referencia, pero notar que la posición de los
puntos en el eje vertical no tiene significado geográfico.

Se observa el ciclo anual en la fase ya se podía apreciar en la
\autoref{fig:qs3-ncep}. La fase media se centra en 55°O en enero y se
desplaza a 90°O en junio con valores intermedios en los meses de
transición. Es esperable que esta variación anual implica que
circulación causada por la QS3 tenga impactos contrarios en verano
comparado con invierno; alternando entre advecciones de aire del norte y
del sur sobre el continente.

Superpuesto a el ciclo anual, en la \autoref{fig:fase-boxplot} también
se puede apreciar la variabilidad interanual para cada mes. Ésta es
considerable y de una magnitud comparable a la del ciclo anual. En
particular, es notorio el aumento en la variabilidad en los meses de
primavera, al punto de que en noviembre la fase prácticamente no tiene
una posición predilecta.

La gran variabilidad presente durante los meses de primavera, en
comparación con el resto del año, explica por qué en los campos medios
la QS3 aparece débil a pesar de que su amplitud mensual no es menor. Al
hacer el promedio, los campos que están defasados en entre 1/4 y 3/4 de
longitud de onda (entre 30° y 90° en el caso de la QS3) interfieren
destructivamente entre ellos, eliminando la señal en los campos medios.
En primavera, más del 30\% de los meses tienen algún nivel de
interferencia destructiva con el campo medio, comparado con el 13\% en
verano.

\begin{longtable}[]{@{}rr@{}}
\caption{Desvío estándar de la fase para cada estación - NO SE
PUBLICA}\tabularnewline
\toprule
Estación & SD\tabularnewline
\midrule
\endfirsthead
\toprule
Estación & SD\tabularnewline
\midrule
\endhead
1 & 17.33446\tabularnewline
2 & 14.76324\tabularnewline
3 & 20.77567\tabularnewline
4 & 18.05311\tabularnewline
5 & 24.53463\tabularnewline
6 & 22.75816\tabularnewline
7 & 19.78503\tabularnewline
8 & 22.25149\tabularnewline
9 & 22.27158\tabularnewline
10 & 27.65658\tabularnewline
11 & 41.90526\tabularnewline
12 & 33.14720\tabularnewline
\bottomrule
\end{longtable}

\begin{longtable}[]{@{}lr@{}}
\caption{Porcentaje de meses con interferencia destructiva - NO SE
PUBLICA}\tabularnewline
\toprule
season & V1\tabularnewline
\midrule
\endfirsthead
\toprule
season & V1\tabularnewline
\midrule
\endhead
Verano & 12.903\tabularnewline
Otoño & 18.280\tabularnewline
Invierno & 26.882\tabularnewline
Primavera & 31.183\tabularnewline
\bottomrule
\end{longtable}

Observando ahora la distribución de los centros ciclónicos y
anticiclónicos (puntos rojos y azules,
respectivamente\todo{esto ya está dicho antes, pero lo aclaro de nuevo?})
se nota que, a pesar de que climatolǵoicamente la Sudamérica está
afectada por un máximo de Z* de la QS3, la variabilidad interanual
implica que hay un número no despreciable de años donde el continente
tiene un mínimo de Z* asociado a esa onda. Aún sólo mirando a los 20
años más intensos de cada mes, en noviembre hay 5 casos y en diciembre,
4.

\section{Estaciones}\label{estaciones}

En las secciones anteriores se mostraron campos medios estacionales
utilizando la definición tradicional de las estaciones climatológicas
(verano = DEF, otoño = MAM, invierno = JJA, primavera = SON). Sin
embargo, como éstas son definidas a partir del ciclo anual de
temperatura en latitudes medias no constituyen necesariamente el mejor
agrupamiento de los datos para otras variables u otras latitudes (por
ejemplo, la Antártida \todo{cita de King}).

Una metodología muy utilizada para la clasificación de campos es el
análisis de componentes principales
(PCA)\todo{citas de PCA}\todo{asumo que no hace falta explicar PCA}.

La tabla xx \ref{eoftabla}\todo{cómo referenciar tablas?} muestra la
varianza explicada de cada componente principal obtenida a partir de los
campos reconstruidos de QS3. Las primeras dos componentes explican más
del 80\% de la varianza y cada una explica una parte similar de la
varianza, indicando que se trata de autovalores
degenerados\todo{buscar cita de esto}. Sabiendo, además, que los campos
de QS3 prácticamente sólo tienen dos grados de libertad (amplitud y
fase), la elección de seleccionar las dos primera componentes es natural
además de justificada por la tabla \ref{eoftabla}.

\begin{longtable}[]{@{}rr@{}}
\caption{Varianza explicada por las 5 primeras componentes principales
de los campos de QS3 reconstruidos.}\tabularnewline
\toprule
PC & \(R^2\)\tabularnewline
\midrule
\endfirsthead
\toprule
PC & \(R^2\)\tabularnewline
\midrule
\endhead
1 & 0.436\tabularnewline
2 & 0.397\tabularnewline
3 & 0.054\tabularnewline
4 & 0.038\tabularnewline
5 & 0.026\tabularnewline
\bottomrule
\end{longtable}

\begin{figure}
\includegraphics[ ]{fig/tesis/eof-1} \caption{Primeras dos componentes principales del campo de QS3 - fig:eof}\label{fig:eof}
\end{figure}

La forma de las dos primeras componentes principales del campo de QS3
(\autoref{fig:eof-field}) son dos QS3 en cuadratura cuya combinación
lineal resulta en otra QS3 cuya fase depende de la amplitud relativa de
cada componente. En verano predomina la PC1, mientras que en invierno
predomina la PC2 como se muestra en la \autoref{fig:pc1-pc2}. En este
diagrama se puede mapear aproximadamente la amplitud y la fase media de
cada mes a la distancia y el ángulo con respecto al origen.

En la \autoref{fig:pc1-pc2} también se hace una posible modificación de
las estaciones clásicas adaptada para el análisis de la QS3 basada en la
posición media de cada mes en el espacio de componentes principales.
Enero, febrero y marzo tienen preponderancia del PC1 y casi nulo PC2,
abril, mayo, agosto, septiembre y octubre tienen una mezcla similar de
componentes, pero es conveniente separar los dos primeros para respetar
la progresión temporal. Junio y julio no están tan juntos como los demás
meses, pero se los puede agrupar por tener gran magnitud de PC2.
Finalmente, noviembre y diciembre aparecen como \emph{outliers} en este
diagrama debido a que su mayor variabilidad (como se notó en la
\autoref{fase}) hace que no predomine ninguna componente principal. Es
posible clasificarlos juntos como meses de ``no estacionaridad''
indicando que se trata de una época del año donde la QS3 no está
presente.

\begin{figure}
\includegraphics[ ]{fig/tesis/pc1-pc2-1} \caption{Valor medio de las dos primeras componentes principales del campo de QS3 - fig:pc1-pc2}\label{fig:pc1-pc2}
\end{figure}

El efecto de esta nueva clasificación en los campos medios se presenta
en las figuras \ref{fig:qs3-qsseason-ncep} y \ref{qs3-qsseason-ncep}.
Comparando con las figuras \ref{fig:qs3-season-ncep} y
\ref{qs3-season-ncep} de la sección \ref{onda-3} se ve que los campos de
las estaciones de transición (AM y ASO) son más similares entre ellas
tanto en el campo horizontal como en el corte meridional. Los meses no
estacionarios (ND), por su parte, prácticamente carecen de QS3.

\begin{figure*}
\includegraphics[width=150mm, ]{fig/tesis/qs3-qsseason-ncep-1} \caption{Media de reconstrucción de onda 3 en 300hPa - fig:qs3-qsseason-ncep}\label{fig:qs3-qsseason-ncep}
\end{figure*}

\begin{figure*}
\includegraphics[width=150mm, ]{fig/tesis/qs3-qsseason-ncep-corte-1} \caption{Corte - fig:qs3-qsseason-ncep-corte}\label{fig:qs3-qsseason-ncep-corte}
\end{figure*}

El uso de componentes principales para el análisis de una onda que
cambia de fase es similar a la metodología utilizada para el monitoreo
de la MJO\todo{cita de MJO} por lo que sería posible su utilización como
indicador de la actividad de la QS3 distinto de la amplitud media de
Fourier. La exploración de dicho indicador está por fuera del objetivo
de este trabajo.

Una desventaja de esta clasificación es que no todas las estaciones
tienen la misma cantidad de meses, lo cual dificulta la comparación
estadística entre distintas estaciones. \#\# Persistencia (ver dónde va)

\begin{figure}
\includegraphics[ ]{fig/tesis/lag-cor-1} \caption{Correlación lageada para cada mes con los 12 sigientes. - fig:lag-cor}\label{fig:lag-cor}
\end{figure}

En la \autoref{fig:lag-cor} se muestra la correlación del campo de QS3
de cada mes con los demás. La línea escalonada marca la separación del
año de manera que un número a la izquierda de la misma implica
correlación de ese mes con meses del año siguiente. Las correlaciones
justo a la izquierda de la línea escalonada son positivas y
relativamente altas para casi todos los meses salvo noviembre, diciembre
y agosto. Esto implica que estos meses tienen poca similaridad de un año
a otro. Noviembre y diciembre también presentan bajas correlaciones en
general con los demás meses y ambas observaciones se puede comprender a
la luz de los resultados de las secciones anteriores: al ser meses con
actividad de la onda 3 poco estacionaria, sus campos de QS3 no son
consistentemente similares con ningún otro mes. Esta interpretació no
parece posible para agosto, ya que su variabilidad no es paricularmente
alta (\autoref{fig:fase-boxplot})\todo{¿qué pasa con agosto?}.

Los valores un mes a la derecha de la linea escalonada son también
generalmente altos indicando buena concordancia entre los campos de un
mes y el siguiente. Para esto nuevamente las excepciónes son noviembre,
diciembre y julio. La explicación para estas bajas correlaciones
posiblemente sea la misma que para las anteriores.

Las correlaciones entre meses corridos 6 meses son bajas para los meses
de verano e invierno y medias para los meses de transición. Es decir,
los meses de verano son muy distintos de los de invierno, mientras que
los de transición son medianamente parecidos a todos. Esta es una
consecuencia del ciclo anual de la fase (\autoref{fig:fase-boxplot}).

\section{R2}\label{r2}

En la \autoref{fig:ondas-quasiestacionarias}, se mostró la estructura
vertical de la varianza explicada por la QS3 para cada estación
(\autoref{fig:r2-ncep}). En esta sección se explora la estructura
horizontal de dicha propiedad. Para esto, se toma como \(r^2\) la
correlación cuadrada entre el valor de Z* y el de QS3 para cada punto de
grilla y cada mes.

\begin{figure}
\includegraphics[ ]{fig/tesis/cor-campo-1} \caption{Correlación cuadrada media para estaciones según onda3. - fig:cor-campo}\label{fig:cor-campo}
\end{figure}

Los campos horizontales de \(r^2\) para 300hPa se muestran en la
\autoref{fig:cor-camp}, con los centros de máximas anomalías marcados
con líneas. En las cuatro estaciones la QS3 explica la mayor parte de la
varianza en el hemiferio oeste entre 60°S y 45°S. Lejos de ser
homogéneos, los campos muestran tres máximos localizados con cierta
persistencia durante el año. El primero, al sur del Índico, está
presente en verano y otoño en 60°E que se mueve hacia el este en
invierno y primavera. Algo similar sucede on el segundo máximo al sur
del Pacífico, que pasa de 180° a 120° entre otoño y primavera. Hay un
tercer máximo en el Atlántico sur cuyo desplazamiento hacia el este es
bastante menor. Finalmente, en verano y otoño hay un máximo en latitudes
bajas en el pacífico central.

Si se compara la posición de los máximos de \(r^2\) con los centros de
QS3, el máximo del Índico, por ejemplo, coincide con un anticiclón en
verano pero se encuentra entre dos centros en otoño y lo mismo pasa con
el máximo del Atlántico. Éste último coincide con un centro
anticiclónico en verano pero está más cerca de uno ciclónico en
invierno. No parece haber una asociación entre los máximos de \(r^2\) y
los centros de QS3.

Comparando con la \autoref{fig:waveletz-ncep} de campos de amplitud de
wavelets, se puede observar cierta correspondencia entre las anomalías
zonales de wavelets y los campos de \(r^2\). Ambos son más intensos en
el hemisferio occidental, muestran una translación general hacia el este
entre verano e invierno e incluso ambos presentan máximos en el pacífico
central. Que estas características se recuperen utilizando dos
metodologías independientes brindan robustez al resultado.

Si se realiza un análisis análogo pero en base a las estaciones
definidas en la \autoref{estaciones} las principales diferencias son un
debilitamiento del los máximos durante EFM y una fuerte intensificación
del máximo del Atlántico durante DN (no se muestra). Las características
generales no cambian.

\begin{figure}
\includegraphics[ ]{fig/tesis/cor-campo2-1} \caption{Correlación cuadrada media para estaciones según onda3. - fig:cor-campo2 - SÓLO BORRADOR}\label{fig:cor-campo2}
\end{figure}

\todo[inline]{hasta acá}

\begin{figure}
\includegraphics[ ]{fig/tesis/r2-cor2-1} \caption{Relación entre R2 medio y R2 reconstruido. - fig:r2-cor2 - SÓLO BORRADOR}\label{fig:r2-cor2}
\end{figure}

Conclusión: no voy a usar el r2 a partir del campo reconstruido.

\section{Regresiones}\label{regresiones}

\subsection{Geopotencial}\label{geopotencial}

\begin{figure}

{\centering \includegraphics[width=227mm,angle=90]{fig/tesis/regr-gh-ncep-1} 

}

\caption{Regresión sobre amplitud. - fig:regr-gh-ncep}\label{fig:regr-gh-ncep}
\end{figure}

Cosas para ver:\\
* Además los patrones de onda 3, en julio y diciembre aparece un patrón
de SAM positivo y negativo respectivamente.

\begin{figure}
\includegraphics[ ]{fig/tesis/regr-gh-polar-1} \caption{Igual que figura  XX, pero en proyección polar para julio y septiembre.{fig:regr-gh-polar}}\label{fig:regr-gh-polar}
\end{figure}

\begin{figure}
\includegraphics[ ]{fig/tesis/sam-ampl-1} \caption{Relación entre amplitud media de la onda 3 y el SAM{fig:sam-ampl}}\label{fig:sam-ampl}
\end{figure}

\subsection{Función Corriente}\label{funcion-corriente-1}

\begin{figure}

{\centering \includegraphics[width=227mm,angle=90]{fig/tesis/regr-psi-ncep-1} 

}

\caption{Regresión de Psi con la amplitud. - fig:regr-psi-ncep}\label{fig:regr-psi-ncep}
\end{figure}

\section{Composición de campos.}\label{composicion-de-campos.}

Descripción de la selección.

\begin{figure}

{\centering \includegraphics[width=227mm,angle=90]{fig/tesis/seleccion-tabla-1} 

}

\caption{Tabla de selección{fig:seleccion-tabla}}\label{fig:seleccion-tabla}
\end{figure}

Cosas para ver:\\
Años con coincidencia, años sin coincidencia. Meses donde la fase
coincide (julio) vs meses donde no coincide (septiembre). También, años
donde hay seguidilla de meses seleccionados (1999). Aunque posiblemente
sea casualidad (no hay mucha persistencia mes a mes.)

Pequeña digresión: Efecto de la fase.\\
La climatología de la fase se va a discutir más adelante, pero\ldots{}
discutir el efecto de promediar campos con similar amplitud pero fase
distinta. Del gráfico, septiembre tiene 1997 y 2003 con fase a 180°, lo
que significa que va a a haber cancelación parcial. Enero, por el
contrario, no tiene ningún año en contrafase, aunque sí algunos a 90°,
que desdibujan el patrón.

\begin{figure}

{\centering \includegraphics[width=130mm, ]{fig/tesis/interaccion-tabla-1} 

}

\caption{Tabla de interacción{fig:interaccion-tabla}}\label{fig:interaccion-tabla}
\end{figure}

Cosas para ver:\\
Ambos criterios coinciden en casi todos los años seleccionados, así que
no hay mucha diferencia. En efecto, las composiciones son casi iguales
(no se muestra). Voy a usar la amplitud.

\begin{figure}

{\centering \includegraphics[width=227mm,angle=90]{fig/tesis/gh-comp-1} 

}

\caption{Composición de campos{fig:gh-comp}}\label{fig:gh-comp}
\end{figure}

\begin{figure}

{\centering \includegraphics[width=150mm, ]{fig/tesis/gh-qs3-select-ene-1} 

}

\caption{Campos para los 10 eneros seleccionados. - fig:gh-qs3-select-ene}\label{fig:gh-qs3-select-ene}
\end{figure}

\begin{figure}

{\centering \includegraphics[width=150mm, ]{fig/tesis/gh-qs3-select-sep-1} 

}

\caption{Campos para los 10 septiembres seleccionados. - fig:gh-qs3-select-sep}\label{fig:gh-qs3-select-sep}
\end{figure}

Estos gráficos me parecen importantes para ver lo que hay ``adentro'' de
la composición, pero no sé bien qué decir sobre ellos. Supongo que lo
principal es que hay años donde la onda

\section{Fuentes de variabilidad
interna}\label{fuentes-de-variabilidad-interna}

(Discusión escrita más de papers), Pero nos concentramos en la fuente
externa.

\section{Fuentes externas}\label{fuentes-externas}

\subsection{SST}\label{sst}

\begin{figure}

{\centering \includegraphics[width=227mm,angle=90]{fig/tesis/regr-sst-ncep-1} 

}

\caption{Regresión de SST con la amplitud de la onda 3{fig:regr-sst-ncep}}\label{fig:regr-sst-ncep}
\end{figure}

Campos de correlación con SST y OLR, principalmente ¿Discusión de otros
forzantes?

\chapter{Experimentos}\label{experimentos}

\section{Validación SPEEDY}\label{validacion-speedy}

Validación de la corrida control

\begin{itemize}
\tightlist
\item
  Comparación campos medios.
\end{itemize}

Acá un problemita es que en speedy no tengo el nivel de 50hPa, sólo
tengo 925, 850, 700, 500, 300, 200, 100, 30. Podría usar 30, pero eso es
la tapa del modelo\ldots{}

\begin{figure}
\includegraphics[ ]{fig/tesis/cor-sp-nc-1} \caption{Correlación lineal entre campos de SPEEDY y NCEP.{fig:cor-sp-nc}}\label{fig:cor-sp-nc}
\end{figure}

\begin{itemize}
\item
  Altura geopotencial (gh) Para el caso del campo total, la correlación
  del campo es buena (\textgreater{}0.8)en casi todos los niveles y
  meses, excepto en 30hPa durante verano donde los campos ¡está
  anticorrelacionados! La parte asimétrica zonal muestra valores
  menores, indicando que gran parte de la correlación del campo total se
  debe a la capacidad del modelo de reproducir el gradiente meridional.
  Sin embargo, se siguen obteniendo correlaciones \textgreater{}0.6 en
  casi todos los niveles y estaciones. Se observa un mínimo relativo en
  500hPa donde en se tienen correlaciones menores durante casi todo el
  año y uno en niveles altos centrado en invierno y primavera.
\item
  Viento zonal (U) Las correlaciones con el campo total son
  \textgreater{}=0.8 en todo el año y todos los niveles, sin embargo, la
  parte asimétrica muestra correlaciones mucho más baja con un máximo de
  \textasciitilde{}0.6 en 925hPa. Esto indica que el modelo resuelve
  correctamente la estructura media del Jet, pero no sus variaciones
  zonales.
\item
  Viento meridional (V) Los campos de correlación son prácticamente
  idénticos entre parte total y parte asimétrica. Ésta muestra un patrón
  de bajas correlaciones en general.
\item
  Temperatura (T) La correlación con el campo total muestra una
  estructura similar que la altura geopotencial, con una excelente
  correlación en todos los meses para niveles mayores a 200hPa, pero
  anticorrelacionado en niveles altos en todos los meses salvo en
  invierno. La parte asimétrica muestra correlaciones bajas en todos los
  niveles salvo en 925hPa.
\item
  Gradiente meridional de vorticidad absoluta. Tiene correlación
  moderada con un mínimo relativo en 200hPa en verano que en las otras
  estacione se convierte en un máximo.
\end{itemize}

\subsection{Altura Geopotencial}\label{altura-geopotencial-1}

Anomalía

\begin{figure}

{\centering \includegraphics[width=227mm,angle=90]{fig/tesis/ghz-sp-nc-1} 

}

\caption{Anomalía zonal de altura geopotencial (speedy sombreado, ncep contornos){fig:ghz-sp-nc}}\label{fig:ghz-sp-nc}
\end{figure}

\begin{figure}

{\centering \includegraphics[width=227mm,angle=90]{fig/tesis/ghz-dif-sp-nc-1} 

}

\caption{Diferencia entre speedy y ncep{fig:ghz-dif-sp-nc}}\label{fig:ghz-dif-sp-nc}
\end{figure}

Veredicto:\\
* Agarra bien la anomalía zonal aunque con magnitud menor.

\begin{figure}

{\centering \includegraphics[width=130mm, ]{fig/tesis/ghz-sp-nc-corte60-1} 

}

\caption{Corte zonal de anomalía de geopotencial en -60° (speedy sombreado, ncep contornos).{fig:ghz-sp-nc-corte60}}\label{fig:ghz-sp-nc-corte60}
\end{figure}

El corte zonal evidencia que además de tener menor amplitud, la
estructura vertical de las anomalías es barotrópica equivalente en
Speedy, a diferencia de la estructura baroclínica de NCEP.

\subsection{Temperatura}\label{temperatura-1}

\begin{figure}
\includegraphics[ ]{fig/tesis/t-nc-sp-1} \caption{Temperatura{fig:t-nc-sp}}\label{fig:t-nc-sp}
\end{figure}

\begin{figure}
\includegraphics[ ]{fig/tesis/tz-sp-nc-1} \caption{T*{fig:tz-sp-nc}}\label{fig:tz-sp-nc}
\end{figure}

En 850 tiene una fuerte onda 1 en latitudes polares producida por la
topografía de la Antártida y representa bien las anomalías causadas por
los continentes, aunque en menor amplitud.\\
En 200, falla miserablemente. La onda 1 polar que se ve bien claro en
NCEP ni aparece en SPEEDY, mientras que en latitudes medias aparecen
ligeras anomalías que no se observan en las observaciones.

\subsection{Viento zonal}\label{viento-zonal-1}

\begin{figure}

{\centering \includegraphics[width=130mm, ]{fig/tesis/u-sp-nc-corte-1} 

}

\caption{Viento zonal medio (speedy contornos, ncep sombreado). - fig:u-sp-nc-corte}\label{fig:u-sp-nc-corte}
\end{figure}

Cosas para ver:\\
* Speedy no logra desarrollar un jet polar por la falta de niveles
verticales en la estratósfera. Tampoco reproduce los estes
estratosféricos en latitudes bajas. Su jet subtropical es más intenso y
su máximo se da ligéramente en niveles más altos en NCEP.

Campo medio (me parece que no lo voy a poner, no agrega información que
no esté en el anterior y en el de geopotencial.)

\begin{figure}

{\centering \includegraphics[width=227mm,angle=90]{fig/tesis/u-sp-nc-1} 

}

\caption{Viento zonal (contornos ncep, sombreado speedy).{fig:u-sp-nc}}\label{fig:u-sp-nc}
\end{figure}

\begin{figure}
\includegraphics[ ]{fig/tesis/u-dif-sp-nc-1} \caption{Diferencia entre ncep y speedy en viento zonal{fig:u-dif-sp-nc}}\label{fig:u-dif-sp-nc}
\end{figure}

Cosas para ver:\\
Jet polar en invierno y primavera en niveles altos (\textless{} 100
hPa). Jest subtropical en niveles ``medios''.

\subsection{Gradiente meridional de vorticidad
absoluta}\label{gradiente-meridional-de-vorticidad-absoluta-1}

\begin{figure}

{\centering \includegraphics[width=227mm,angle=90]{fig/tesis/etady-sp-nc-1} 

}

\caption{Gradiente meridional de vorticidad absoluta (speedy).{fig:etady-sp-nc}}\label{fig:etady-sp-nc}
\end{figure}

Comparando con la figura \autoref{fig:eta-dy-ncep}, los gradientes son
menores y más zonales. Es significativo que la región ``prohibida'' en
invierno de niveles altos es menor en 200hPa y casi desaparece en
300hPa.

\subsection{Número de onda
estacionaria}\label{numero-de-onda-estacionaria}

\begin{figure}
\includegraphics[ ]{fig/tesis/ks-sp-1} \caption{Número de onda estacionario en 300hPa (speedy).{fig:ks-sp}}\label{fig:ks-sp}
\end{figure}

Comparando con \autoref{fig:k-steady-ncep} la mayor diferencia es la
desaparición de una región de propagación impedida en
\textasciitilde{}-40° en el Índico y el Pacífico en Otoño.

\begin{figure}

{\centering \includegraphics[width=130mm, ]{fig/tesis/ks-sp-nc-corte-1} 

}

\caption{Número de onda estacionario medio por círculo de latitud.{fig:ks-sp-nc-corte}}\label{fig:ks-sp-nc-corte}
\end{figure}

En promedio zonal, sin embargo, SPEEDY funciona bien.

\subsection{Función corriente}\label{funcion-corriente-2}

\begin{figure}

{\centering \includegraphics[width=227mm,angle=90]{fig/tesis/psi-sp-1} 

}

\caption{Función corriente x 1099{fig:psi-sp}}\label{fig:psi-sp}
\end{figure}

\subsection{Onda 3}\label{onda-3-1}

\begin{figure*}
\includegraphics[width=150mm, ]{fig/tesis/ampl-sp-nc-1} \caption{Amplitud de Fourier (speedy en sombreado, ncep en contornos). - fig:ampl-sp-nc}\label{fig:ampl-sp-nc}
\end{figure*}

\begin{figure*}
\includegraphics[width=150mm, ]{fig/tesis/qs2-sp-nc-1} \caption{Media de reconstrucción de onda 3 (sombreado speedy, contornos ncep){fig:qs2-sp-nc}}\label{fig:qs2-sp-nc}
\end{figure*}

La onda 3 no está muy bien representada en el modelo. Aunque la
estructura vertical y la posición meridional está bien (salvo en otoño),
la amplitud es mucho menor. La fase, además, está corrida ligeramente en
verano, pero quedando en cuadratura en invierno y defasado 180 en
primavera.

\section{Comparación}\label{comparacion}

Comparación entre corridas \#\#\# Altura geopotencial

\begin{figure}

{\centering \includegraphics[width=227mm,angle=90]{fig/tesis/ghz-sp-runs-1} 

}

\caption{Anomalía zonal de altura geopotencial.{fig:ghz-sp-runs}}\label{fig:ghz-sp-runs}
\end{figure}

\begin{figure}
\includegraphics[ ]{fig/tesis/ghz-dif-sp-runs-1} \caption{Diferencia Control - corrida para Z* - fig:ghz-dif-sp-runs}\label{fig:ghz-dif-sp-runs}
\end{figure}

\subsection{Temperatura}\label{temperatura-2}

ONo hay casi diferencia entre las corridas.

\begin{figure}
\includegraphics[ ]{fig/tesis/t-sp-runs-1} \caption{Temperatura media en 850hPa.{fig:t-sp-runs}}\label{fig:t-sp-runs}
\end{figure}

\begin{figure}
\includegraphics[ ]{fig/tesis/tz-sp-runs-1} \caption{Temperatura media en 850hPa.{fig:tz-sp-runs}}\label{fig:tz-sp-runs}
\end{figure}

\begin{figure}

{\centering \includegraphics[width=227mm,angle=90]{fig/tesis/tz-dif-sp-runs-1} 

}

\caption{Diferencia Control - corrida para T* - fig:tz-dif-sp-runs}\label{fig:tz-dif-sp-runs}
\end{figure}

\subsection{Viento zonal}\label{viento-zonal-2}

\begin{figure}

{\centering \includegraphics[width=227mm,angle=90]{fig/tesis/uz-sp-runs-1} 

}

\caption{Viento zonal{fig:uz-sp-runs}}\label{fig:uz-sp-runs}
\end{figure}

\begin{figure}
\includegraphics[ ]{fig/tesis/u-dif-sp-runs-1} \caption{Diferencia control - corrida para U. - fig:u-dif-sp-runs}\label{fig:u-dif-sp-runs}
\end{figure}

\subsection{Función corriente}\label{funcion-corriente-3}

\begin{figure}

{\centering \includegraphics[width=227mm,angle=90]{fig/tesis/psi-sp-runs-1} 

}

\caption{Anomalía zonal de función corriente y flujos de acción de onda.{fig:psi-sp-runs}}\label{fig:psi-sp-runs}
\end{figure}

\begin{figure}
\includegraphics[ ]{fig/tesis/psiz-dif-sp-runs-1} \caption{Diferencia en psi.z y flujos de acción de onda.{fig:psiz-dif-sp-runs}}\label{fig:psiz-dif-sp-runs}
\end{figure}

\subsection{Onda 3}\label{onda-3-2}

\begin{figure}
\includegraphics[ ]{fig/tesis/ampl-sp-runs-1} \caption{Amplitud media de la onda 3 para cada corrida. - fig:ampl-sp-runs}\label{fig:ampl-sp-runs}
\end{figure}

\begin{figure}
\includegraphics[ ]{fig/tesis/ampl-dif-sp-runs-1} \caption{Diferencia de amplitud entre la corrida control y cada corrida. - fig:ampl-dif-sp-runs}\label{fig:ampl-dif-sp-runs}
\end{figure}

\begin{figure}
\includegraphics[ ]{fig/tesis/index-sp-boxplot-1} \caption{Ciclo anual de amplitud de onda 3.{fig:index-sp-boxplot}}\label{fig:index-sp-boxplot}
\end{figure}

\section{Regresión}\label{regresion}

\begin{figure}
\includegraphics[ ]{fig/tesis/regr-psi-sp-runs-1} \caption{Regresión en función corriente. - fig:regr-psi-sp-runs}\label{fig:regr-psi-sp-runs}
\end{figure}

\section{Cosas inesperadas\ldots{}}\label{cosas-inesperadas}

\begin{itemize}
\tightlist
\item
  ??
\item
  protif!
\end{itemize}

\chapter{Conclusiones}\label{conclusiones}

\chapter{Agradecimientos}\label{agradecimientos}

\chapter*{Referencias}\label{referencias}
\addcontentsline{toc}{chapter}{Referencias}

\hypertarget{refs}{}
\hypertarget{ref-Berbery1992}{}
Berbery, E H, J Nogués-Paegle, y J D Horel. 1992. «Wavelike southern
hemisphere extratropical teleconnections».
doi:\href{https://doi.org/DOI:\%2010.1175/1520-0469(1992)049\%3C0155:WSHET\%3E2.0.CO;2}{DOI: 10.1175/1520-0469(1992)049\textless{}0155:WSHET\textgreater{}2.0.CO;2}.

\hypertarget{ref-James}{}
James, I. N. 1994. \emph{Introduction to circulating atmospheres}.
Cambridge: Cambridge University Press.
doi:\href{https://doi.org/10.1017/CBO9780511622977}{10.1017/CBO9780511622977}.

\hypertarget{ref-Quintanar1995}{}
Quintanar, Arturo I., y Carlos R. Mechoso. 1995. «Quasi-Stationary Waves
in the Southern Hemisphere. Part II: Generation Mechanisms».
\emph{Journal of Climate} 8 (11): 2673-90.
doi:\href{https://doi.org/10.1175/1520-0442(1995)008\%3C2673:QSWITS\%3E2.0.CO;2}{10.1175/1520-0442(1995)008\textless{}2673:QSWITS\textgreater{}2.0.CO;2}.

\hypertarget{ref-Trenberth1985}{}
Trenberth, Kevin E., y K. C. Mo. 1985. «Blocking in the Southern
Hemisphere».
doi:\href{https://doi.org/10.1175/1520-0493(1985)113\%3C0003:BITSH\%3E2.0.CO;2}{10.1175/1520-0493(1985)113\textless{}0003:BITSH\textgreater{}2.0.CO;2}.

\hypertarget{ref-Vera2004}{}
Vera, Carolina, Gabriel Silvestri, Vicente Barros, y Andrea Carril.
2004. «Differences in El Niño response over the Southern Hemisphere».
\emph{Journal of Climate} 17 (9): 1741-53.
doi:\href{https://doi.org/10.1175/1520-0442(2004)017\%3C1741:DIENRO\%3E2.0.CO;2}{10.1175/1520-0442(2004)017\textless{}1741:DIENRO\textgreater{}2.0.CO;2}.


\end{document}
