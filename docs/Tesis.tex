\documentclass[spanish,a4paper]{book}
\usepackage{lmodern}
\usepackage{amssymb,amsmath}
\usepackage{ifxetex,ifluatex}
\usepackage{fixltx2e} % provides \textsubscript
\ifnum 0\ifxetex 1\fi\ifluatex 1\fi=0 % if pdftex
  \usepackage[T1]{fontenc}
  \usepackage[utf8]{inputenc}
\else % if luatex or xelatex
  \ifxetex
    \usepackage{mathspec}
  \else
    \usepackage{fontspec}
  \fi
  \defaultfontfeatures{Ligatures=TeX,Scale=MatchLowercase}
\fi
% use upquote if available, for straight quotes in verbatim environments
\IfFileExists{upquote.sty}{\usepackage{upquote}}{}
% use microtype if available
\IfFileExists{microtype.sty}{%
\usepackage{microtype}
\UseMicrotypeSet[protrusion]{basicmath} % disable protrusion for tt fonts
}{}
\usepackage[inner = 5cm, outer = 5cm, top = 2.5cm, bottom = 2.5cm]{geometry}
\usepackage{hyperref}
\hypersetup{unicode=true,
            pdftitle={Tesis},
            pdfauthor={Elio Campitelli},
            pdfborder={0 0 0},
            breaklinks=true}
\urlstyle{same}  % don't use monospace font for urls
\ifnum 0\ifxetex 1\fi\ifluatex 1\fi=0 % if pdftex
  \usepackage[shorthands=off,main=spanish]{babel}
\else
  \usepackage{polyglossia}
  \setmainlanguage[]{spanish}
\fi
\usepackage{longtable,booktabs}
\usepackage{graphicx,grffile}
\makeatletter
\def\maxwidth{\ifdim\Gin@nat@width>\linewidth\linewidth\else\Gin@nat@width\fi}
\def\maxheight{\ifdim\Gin@nat@height>\textheight\textheight\else\Gin@nat@height\fi}
\makeatother
% Scale images if necessary, so that they will not overflow the page
% margins by default, and it is still possible to overwrite the defaults
% using explicit options in \includegraphics[width, height, ...]{}
\setkeys{Gin}{width=\maxwidth,height=\maxheight,keepaspectratio}
\IfFileExists{parskip.sty}{%
\usepackage{parskip}
}{% else
\setlength{\parindent}{0pt}
\setlength{\parskip}{6pt plus 2pt minus 1pt}
}
\setlength{\emergencystretch}{3em}  % prevent overfull lines
\providecommand{\tightlist}{%
  \setlength{\itemsep}{0pt}\setlength{\parskip}{0pt}}
\setcounter{secnumdepth}{5}
% Redefines (sub)paragraphs to behave more like sections
\ifx\paragraph\undefined\else
\let\oldparagraph\paragraph
\renewcommand{\paragraph}[1]{\oldparagraph{#1}\mbox{}}
\fi
\ifx\subparagraph\undefined\else
\let\oldsubparagraph\subparagraph
\renewcommand{\subparagraph}[1]{\oldsubparagraph{#1}\mbox{}}
\fi

%%% Use protect on footnotes to avoid problems with footnotes in titles
\let\rmarkdownfootnote\footnote%
\def\footnote{\protect\rmarkdownfootnote}

%%% Change title format to be more compact
\usepackage{titling}

% Create subtitle command for use in maketitle
\newcommand{\subtitle}[1]{
  \posttitle{
    \begin{center}\large#1\end{center}
    }
}

\setlength{\droptitle}{-2em}
  \title{Tesis}
  \pretitle{\vspace{\droptitle}\centering\huge}
  \posttitle{\par}
\subtitle{Una tesis}
  \author{Elio Campitelli}
  \preauthor{\centering\large\emph}
  \postauthor{\par}
  \date{}
  \predate{}\postdate{}

\linespread{1.25}
\usepackage{subfig}
\usepackage{hyperref}
\usepackage{marginnote}
\usepackage[nomarkers,figuresonly]{endfloat}
\usepackage[spanish]{todonotes}
\usepackage{wrapfig}

\begin{document}
\maketitle

{
\setcounter{tocdepth}{3}
\tableofcontents
}
Resumen.

\todo[inline]{Listado de abreviaturas}
\todo[inline]{Revisar TODOS los epígrafes}

\chapter{Introducción}\label{introduccion}

\begin{itemize}
\tightlist
\item
  Antecedentes\\
  Además de lo que hay en lo de las becas + lo que fui encontrando,
  agregar sobre las climatologías disponibles y sus limitaciones.
\item
  Objetivo General
\item
  Objetivo particular
\end{itemize}

Esto es para probar una referencia bibliográfica: Vera et~al.
(\protect\hyperlink{ref-Vera2004}{2004}) y (Vera et~al.
\protect\hyperlink{ref-Vera2004}{2004})

\chapter{Métodos y Materiales}\label{metodos-y-materiales}

\todo[inline]{Agregar en algún lugar algo sobre las estadísticas circulares}

\section{Conceptos básicos}\label{conceptos-basicos}

\begin{itemize}
\tightlist
\item
  Ondas cuasiestacionarias
\item
  Fourier
\item
  wavelets
\end{itemize}

\todo[inline]{chequear este paper: https://link.springer.com/article/10.1007/s00024-012-0635-9}

Ejemplo:

\begin{figure*}
\includegraphics[width=150mm, ]{fig/tesis/fourier-ejemplo-1} \caption{Ejemplo fourier - fig:fourier-ejemplo}\label{fig:fourier-ejemplo}
\end{figure*}

Cosas para ver de \autoref{fig:fourier-ejemplo}:\\
Descripción del ``rol'' de cada número de onda en generar el campo
final. La QS1 es la principal, marcando altas presiones al sur del
pacífico y bajas al sur de África. La onda 3 modifica ese patrón simple
haciendo que los máximos y mínimos no sean contínuos.

\begin{itemize}
\tightlist
\item
  Wavelets
\end{itemize}

\begin{figure*}
\subfloat[Campo\label{fig:wavelet-ejemplo1}]{\includegraphics[width=75mm, ]{fig/tesis/wavelet-ejemplo-1} }\subfloat[Análisis de Wavelets en círculos de latitud señalados.\label{fig:wavelet-ejemplo2}]{\includegraphics[width=75mm, ]{fig/tesis/wavelet-ejemplo-2} }\caption{Wavelets - fig:wavelet-ejemplo}\label{fig:wavelet-ejemplo}
\end{figure*}

Cosas para ver:\\
Cambio en el máximo. Localización en vez de un número para cada latitud.

\section{Fuentes de datos}\label{fuentes-de-datos}

\section{Descripción de SPEEDY}\label{descripcion-de-speedy}

\chapter{Climatología observada}\label{climatologia-observada}

\section{Altura geopotencial}\label{altura-geopotencial}

Campo medio:

\begin{figure}

{\centering \includegraphics[width=227mm,angle=90]{fig/tesis/gh-ncep-1} 

}

\caption{Campo de Z (NCEP) - fig:gh-ncep}\label{fig:gh-ncep}
\end{figure}

El campo de altura geopotencial media (Z, \autoref{fig:gh-ncep}) muestra
una estructura más zonal en el HS que en el HN. Es evidente la
generación del vórtice polar en la estratósfera de invierno y primavera
y el aumento del gradiente meridional de Z con la altura.

\begin{figure}
\includegraphics[ ]{fig/tesis/ghdy-ncep-corte-1} \caption{Gradiente meridional de Z - fig:ghdy-ncep-corte}\label{fig:ghdy-ncep-corte}
\end{figure}

\begin{figure}

{\centering \includegraphics[width=227mm,angle=90]{fig/tesis/ghz-ncep-1} 

}

\caption{Anomalía zonal de altura geopotencial. - fig:ghz-ncep}\label{fig:ghz-ncep}
\end{figure}

Las anomalías zonales (\autoref{fig:ghz-ncep}) muestran una
preponderancia de una onda 1 (QS1) con una amplitud máxima en la
estratósfera de primavera y una estructura coherente en la vertical en
60°S (\autoref{fig:ghz-ncep-corte60}). Pueden diferenciarse dos QS1
distintas; una centrada en \textasciitilde{}60°S y con el centro
anticiclónico al rededor de la línea de fecha, y la otra sobre la costa
del continente y el centro anticilónico entre 120 y 60°O. Quintanar y
Mechoso (\protect\hyperlink{ref-Quintanar1995}{1995}) concluyó que el
primero está asociado principalmente con forzantes de latitudes bajas
mientras que el segundo responde a la orografía del continente
antártico.

Salvo en verano, la inclinación hacia el oeste en la horizontal y en la
vertical indica que las perturbaciones estacionarias están asociadas con
transporte hacia el polo tanto de cantidad de movimiento zonal como de
temperatura.
\todo[inline]{¿Debería agregar una justificaicón de esto? ¿Un resumen del desarrollo matemático que está en el James?}

En verano, las anomalías zonales tienen una estructura barotrópica
equivalente y carecen de inclinación en la horizontal. Consecuentemente,
los transportes meridionales asociados se reducen considerablemente.
\todo{¿Por qué?}

\begin{figure}

{\centering \includegraphics[width=130mm, ]{fig/tesis/ghz-ncep-corte60-1} 

}

\caption{Corte zonal de anomalía de geopotencial en -60°.{fig:ghz-ncep-corte60}}\label{fig:ghz-ncep-corte60}
\end{figure}

\begin{figure}
\subfloat[uz*vz*\label{fig:uzvz-ncep-corte1}]{\includegraphics[ ]{fig/tesis/uzvz-ncep-corte-1} }\newline\subfloat[T*v*\label{fig:uzvz-ncep-corte2}]{\includegraphics[ ]{fig/tesis/uzvz-ncep-corte-2} }\caption{Transportes - fig:uzvz-ncep-corte - SÓLO BORRADOR}\label{fig:uzvz-ncep-corte}
\end{figure}

El ciclo anual de la amplitud de las ondas estacionarias se observa en
\autoref{fig:sd-gh-ncep}, donde se muestra el desvío estándar de Z por
círculo de latitud para cada mes
\todo{¿Explicar la relación de esto con la amplitud de la onda?}. La
mayor amplitud se enuentra centrado al rededor de los 60°S, como ya se
vio en la \autoref{fig:ghz-ncep} y alcanza su máximo entre agosto y
octubre, y el mínimo entre febrero y abril, según el nivel.

\begin{figure}

{\centering \includegraphics[width=130mm, ]{fig/tesis/sd-gh-ncep-1} 

}

\caption{Desvío estándar por círculo de latitud. - fig:sd-gh-ncep}\label{fig:sd-gh-ncep}
\end{figure}

\section{Temperatura}\label{temperatura}

\begin{figure}

{\centering \includegraphics[width=227mm,angle=90]{fig/tesis/t-ncep-1} 

}

\caption{Temperatura media. - fig:t-ncep}\label{fig:t-ncep}
\end{figure}

En la tropósfera, la temperatura (\autoref{fig:t-ncep}) presenta una
estructura zonal con mayor baroclinicidad en invierno. En 200 hPa los
gradientes meridionales son mínimos y por encima de ese nivel éstos se
revierten, dando lugar a tempearturas ecuatoriales menores que las
polares (\autoref{fig:tz-ncep-corte}). Por encima de 100 hPa de invierno
y primavera, hay un máximo relativo en \textasciitilde{}45°
\todo{¿Por qué?} y una región de anomalías zonales positivas que se ve
claramente incluso en el campo total, particularmente en primavera.

\begin{figure}

{\centering \includegraphics[width=130mm, ]{fig/tesis/tz-ncep-corte-1} 

}

\caption{Corte meridional de temperatura media.{fig:tz-ncep-corte}}\label{fig:tz-ncep-corte}
\end{figure}

\begin{figure}

{\centering \includegraphics[width=227mm,angle=90]{fig/tesis/tz-ncep-1} 

}

\caption{Anomalía zonal de temperatura. - fig:tz-ncep}\label{fig:tz-ncep}
\end{figure}

Como se puede apreciar en el campo de T* (\autoref{fig:tz-ncep}), en
invierno y primavera, los niveles altos están dominados por una QS1 con
máximo en el sur de Autralia y mínimo en el Atlántico sur. En niveles
más bajos, la onda se defasa hacia el este y queda casi en cuadratura
(\autoref{fig:tz-ncep-corte60}) con el máximo en 850hPa en Antártida
occidental.

\todo[inline]{¿Más detalle?}

En los trópicos, las anomlaías de temperatura presentan un patrón de
onda 3 con máximos en los continentes y mínimos en los océanos que se
revierten de signo en 100hPa. Estas características tienen su correlato
en la altura geopotencial y corresponden a circulaciones tipo Walker
\todo{¿Es así?}.

\begin{figure}

{\centering \includegraphics[width=130mm, ]{fig/tesis/tz-ncep-corte60-1} 

}

\caption{Corte zonal de anomalía de temperatura en -60°. - fig:tz-ncep-corte60}\label{fig:tz-ncep-corte60}
\end{figure}

\section{Viento zonal}\label{viento-zonal}

\begin{figure}

{\centering \includegraphics[width=130mm, ]{fig/tesis/u-ncep-corte-1} 

}

\caption{Viento zonal medio. - fig:u-ncep-corte}\label{fig:u-ncep-corte}
\end{figure}

La estructura meridional del viento zonal (\autoref{fig:u-ncep-corte})
muestra la localización e intensidad de los jets subtropical y subpolar.
El primero está presente durante todo el año aunque con mayor intensidad
y corrido hacia latitudes más ecuatoriales en invierno y primavera. El
segundo está presente principalmente en invierno y primavera, e
incipiente en otoño. En la estratósfera se observan vientos del este en
latitudes bajas que son más intensos en verano y otoño.

\begin{figure}

{\centering \includegraphics[width=227mm,angle=90]{fig/tesis/u-ncep-1} 

}

\caption{Viento zonal. - fig:u-ncep}\label{fig:u-ncep}
\end{figure}

En la \autoref{fig:u-ncep} se observa que el jet subpolar es más intenso
al sur de África, donde además se encuentra en una latitud más
ecuatorial que en la región del Pacífico. El jet subtropical también
tiene un máximo al sur de África y otro al norte de Nueva Zelanda
--especialmente en invierno--, donde además se produce una bifurcación
del jet. Se trata de una región de persistentes y frecuentes bloqueos
(Trenberth y Mo \protect\hyperlink{ref-Trenberth1985}{1985}) que se
evidencia como anomalías zonales negativas durante todo el año
(\autoref{fig:uz-ncep}).

Las anomalías zonales en la \autoref{fig:uz-ncep} también evidencian la
circulación tropical forzada por la temperatura superficial del
pacífico. Consistene con los campos de temperatura, en verano del
pacífico central existen divergencias en niveles altos y convergencias
en niveles bajos (no se muestra) que se traducen anomalías zonales
positivas y negativas al este y al oeste de 180° respectivamente.

\begin{figure}

{\centering \includegraphics[width=227mm,angle=90]{fig/tesis/uz-ncep-1} 

}

\caption{Anomalía zonal de viento zonal. - fig:uz-ncep}\label{fig:uz-ncep}
\end{figure}

\section{Viento meridional}\label{viento-meridional}

\begin{figure}

{\centering \includegraphics[width=130mm, ]{fig/tesis/v-ncep-corte-1} 

}

\caption{Media zonal del viento meridional.{fig:v-ncep-corte}}\label{fig:v-ncep-corte}
\end{figure}

En los campos medios zonales de viento meridional
(\autoref{fig:v-ncep-corte}) se muestra claramente la circulación de
Hadley. En verano, la rama ascendente de encuentra en el hemisferio sur
y se tiene convergencias en niveles bajos y divergencias en niveles
altos. En invierno, en cambio, sólo se ve la rama descendente, mucho más
intensa que en verano, que genera convergencias en niveles altos y
divergencias en niveles bajos al rededor de los 30°S.

Presentes durante todo el año, los vientos de drenaje antárticos también
son evidentes en la \autoref{fig:v-ncep-corte}. Estos vientos
catabáticos son máximos en la costa del continente y alcanzan su máxima
intensidad en invierno.

\begin{figure}

{\centering \includegraphics[width=227mm,angle=90]{fig/tesis/v-ncep-1} 

}

\caption{Viento meridional medio. - fig:v-ncep}\label{fig:v-ncep}
\end{figure}

Los campos horizontales de V se muestran en la \autoref{fig:v-ncep}. En
invierno, existe evidencia de un tren de ondas de Rossby que se propaga
desde el Índico occidental sudeste llegando a su máxima latutud en 150°O
donde comienza a propagarse hacia el norte hasta llegar al sur de
Sudamérica. Este tren de ondas puede identificarse con mayor dificultad
en el campo de geopotencial (\autoref{fig:ghz-ncep}) debido a la
interferencia de la QS1.

También en invierno, en los trópicos se puede observar el viento
meridional hacia el norte en 850hPa en la costa oeste de África asociado
al monzón de la India. El monzón sudamericano, por su parte, se evidenia
por un aumento de la intensidad de los vientos del norte en verano sobre
el continente americano. En altura, el monzón de la India se muestra
como viento hacia el sur producto de la divergencia de niveles altos
generada por la convección anómala. El sudamérica, la alta de Bolivia
genera viento hacia el nort y hacia el sur al este y al oeste de Bolivia
respectivamente.

La anomalía zonal de V es prácticamente idéntica al campo total ya que
la media zonal es casi nula en gran parte del dominio
(\autoref{fig:v-ncep-corte}); por lo tanto, no se muestra.

\begin{figure}

{\centering \includegraphics[width=227mm,angle=90]{fig/tesis/ghminus1-ncep-1} 

}

\caption{Z* menos QS1. - fig:ghminus1-ncep - SÓLO BORRADOR}\label{fig:ghminus1-ncep}
\end{figure}

\section{Gradiente meridional de vorticidad
absoluta}\label{gradiente-meridional-de-vorticidad-absoluta}

En la teoría de propagación meridional de ondas de Rossby, el gradiente
meridional de vorticidad absoluta (\(\psi_y\))
\todo[inline]{es importante porque blavalbla ¿Resumen de la deducción en James?}

\begin{figure}

{\centering \includegraphics[width=227mm,angle=90]{fig/tesis/etady-ncep-1} 

}

\caption{Gradiente meridional de vorticidad absoluta * 1e11 - fig:etady-ncep}\label{fig:etady-ncep}
\end{figure}

La \autoref{fig:etady-ncep} muestra que el campo de \(\psi_y\) impide la
propagación meridional de ondas de Rossby en una \emph{región prohibida}
acotada centrada sobre Nueva Zelanda principalmente en invierno entre
300 y 200 hPa. Ésta coincide con la región de bloqueos que, en invierno
está franqueada por el jet intenso, dando lugar a gradientes
meridionales de U negativos más intensos que \(\beta\). Estas figuras
extienden el restulado de Berbery, Nogués-Paegle, y Horel
(\protect\hyperlink{ref-Berbery1992}{1992}) (su Figura 3) utilizando 5
años de análisis objetivo del Centro Europeo de Predicción a Plazo Medio
(ECMWF).

\section{Número de onda estacionaria}\label{numero-de-onda-estacionaria}

Aún con \(\psi_y\) positivo, las ondas de Rossby sólo se pueden propagar
si su número de onda zonal es menor que el número de onda estacionario
(James \protect\hyperlink{ref-James}{1994}). En la \autoref{fig:ks-ncep}
se muestra el número de onda estacionario para el nivel de 200hPa y en
la \autoref{fig:ks-ncep-corte} se muestra un corte en 180°. Además de la
\emph{región prohibida}, las ondas cortas no pueden propagarse
meridionalmente en latitudes altas. Las ondas largas con k \textless{} 3
pueden propagarse meridionalmente libremente en verano al norte de los
60°S aproximadamente, pero quedan atrapadas al sur de 45°S en toda una
franja de longitudes entre 60°E y 120°O durante el resto del año.

El corte de \(\psi_y\) en 180° evidencia que la onda 3 sólo puede
propagarse meridionalmente dentro de una franja angosta de latitudes,
entre 60°S y 50°S. Una región donde este modo muestra un máximo de
amplitud (\autoref{fig:ampl-ncep}).

\begin{figure}
\includegraphics[ ]{fig/tesis/ks-ncep-1} \caption{Número de onda estacionario en 200hPa. - fig:ks-ncep}\label{fig:ks-ncep}
\end{figure}

\begin{figure}

{\centering \includegraphics[width=130mm, ]{fig/tesis/ks-ncep-corte-1} 

}

\caption{Número de onda estacionario medio por círculo de latitud. - fig:ks-ncep-corte}\label{fig:ks-ncep-corte}
\end{figure}

\begin{figure}
\includegraphics[ ]{fig/tesis/ks-ncep-cortelev-1} \caption{Número de onda estacionario en 180° - fig:ks-ncep-cortelev - SÓLO BORRADOR}\label{fig:ks-ncep-cortelev}
\end{figure}

\section{Función corriente}\label{funcion-corriente}

\begin{figure}

{\centering \includegraphics[width=227mm,angle=90]{fig/tesis/psi-ncep-1} 

}

\caption{Función corriente x 1099 - fig:psi-ncep}\label{fig:psi-ncep}
\end{figure}

No sé bien qué decir de la \autoref{fig:psi-ncep} :(.

\section{Ondas Quasiestacionarias}\label{ondas-quasiestacionarias}

Como se describió en \autoref{metodos-y-materiales}, el análisis
estacional de la amplitud y \(r^2\) de Fourier puede hacerse a partir de
la media delos campos mensuales o aplicando Fourier a los campos
estacionales. En el caso de los datos de NCEP existe poca diferencia,
por lo que sólo se muestran los resultados conseguidos mediante esta
última metodología.

El campo de Z* está caracterizado principalmente por un patrón de QS1 en
altas latidues latitudes (\autoref{fig:ghz-ncep}). No es sorprendente,
entonces, que la QS1 explica la mayor parte de la variabilidad del
geopotencial en virtualmente todo el dominio al sur de los
45°S(\autoref{fig:r2-ncep}). La QS2 es preponderante en la estratósfera
ecuatorial, en la costa antártica y al rederor de 35°S, donde es el modo
dominante en toda la columna de aire en verano. La QS3, a diferencia de
las ondas anteriores es importante en una región reducidda. Explica una
parte substancial de la varianza en niveles bajos al rededor de los 45°S
y mayormente en invierno. La QS4 explica muy poca varianza a excepción
de cerca de superficie entre 15°S y 30°S. Ondas más cortas son aún menos
importantes (no se muestra).

\begin{figure}

{\centering \includegraphics[width=227mm,angle=90]{fig/tesis/r2-ncep-1} 

}

\caption{$R^2$ de Fourier. - fig:r2-ncep}\label{fig:r2-ncep}
\end{figure}

El \(r^2\) permite analizar la importancia relativa de cada modo con
respecto a la variabilidad total, pero lo que importa desde el punto de
vista físico es la amplitud de la onda
\todo{Esto está muy mal escrito, hay que mejorar}. Las diferencias entre
los campos de \(r^2\) y los de amplitud son evidentes comparando las
figuras \ref{fig:r2-ncep} y \ref{fig:ampl-ncep} (notar la escala
logarítima en los colores). La amplitud de la QS1 muestra un máximo bien
definido centrado en 60°S que en verano se encuentra en niveles más
bajos que en las otras estaciones. También existe un máximo relativo
entre 15°S y 30°S en verano que migra a latitudes más altas en invierno
y primavera. El mismo está presente tambień en las otras ondas
estacionarias.

En el caso de la QS2, se evidencia que a pesar de tener máximos de
\(r^2\) en la estratósfera al norte de 45°S, alcanza su máxima amplitud
al sur de esa latitud y en 200hPa en verano y en 30hPa en invierno. Su
actividad en la costa antártica se extiende en toda la tropósfera en
invierno (a pesar de que en \(r^2\) pierde importancia por encima de los
200hPa)

La región de amplitud máxima de la QS3, coincide aproximadamente con la
región de máximo \(r^2\) entre y otoño y primavera, aunque con menos
actividad en superficie y extensión en toda la columna. En verano, en
cambio aparece un máximo de amplitud importante que no se observa en el
campo de \(r^2\).

Finalmente, fuera de la superficie, la QS4 presenta un máxmimo de
amplitud bien definido sólo en verano. El máximo entre 15°S y 30°S sigue
presente.

\begin{figure}

{\centering \includegraphics[width=227mm,angle=90]{fig/tesis/ampl-ncep-1} 

}

\caption{Ampllitud de Fourier. - fig:ampl-ncep}\label{fig:ampl-ncep}
\end{figure}

\todo[inline]{Acá podría poner alguna reflexión general de lo que se ve. Por ejemplo, cómo en verano la variabilidad está más acotada a la tropóstera mientras que en invierno y primavera hay más contacto con la estratósfera.}

\chapter{Onda 3}\label{onda-3}

\todo[inline]{introducción a QS3; decir que esto es una onda reconstruida a partir de Fourier}

\section{Características típicas}\label{caracteristicas-tipicas}

\begin{figure*}
\includegraphics[width=150mm, ]{fig/tesis/qs3-ncep-1} \caption{Media de reconstrucción de onda 3. - fig:qs3-ncep}\label{fig:qs3-ncep}
\end{figure*}

En la \autoref{fig:qs3-ncep} se muestra el campo de geopotencial
reconstruido sólo a partir de la QS3 en 300 hPa, que ilustra lo que
sucede en todos los niveles dado que su estructura es barotrópica
equivalente (\autoref{fig:qs3-ncep-corte}. Se reproduen las
caracterísicas de la amplitud ya vistas en la \autoref{fig:ampl-ncep}
(máximo entre 60°S y 45°S con menor intensidad en primavera, mayor
amplitud en 300hPa con mayor desarrollo vertical en otoño y verano) pero
además permite observar el efecto de la variación de la fase. Se obseva
que existe un corrimiento de la fase entre verano e invierno de poco más
de 15° (algo ya observado por
??\todo{buscar referencia. Raphael referencia este movimiento, pero no lo encuentro en los papers que cita.})
anticipando que el efecto de la QS3 sobre cada lugar \todo{mal escrito}
pueda tener una componente estacional.

La inclinación meridinal de las pertubaciones de geopotencial asociadas
a la QS3 sugieren que ésta está asociada a transportes de cantidad de
movimiento hacia el polo en las estaciones de transición, pero en menor
medida en verano.

\begin{longtable}[]{@{}lr@{}}
\caption{Transporte meridinoal de cantidad de viento zonal NO SE
PUBLICA}\tabularnewline
\toprule
Estación & {[}u\emph{v}{]}\tabularnewline
\midrule
\endfirsthead
\toprule
Estación & {[}u\emph{v}{]}\tabularnewline
\midrule
\endhead
Verano & -0.100\tabularnewline
Otoño & -0.431\tabularnewline
Invierno & -0.016\tabularnewline
Primavera & -0.163\tabularnewline
\bottomrule
\end{longtable}

\begin{figure*}
\includegraphics[width=150mm, ]{fig/tesis/qs3-ncep-corte-1} \caption{Corte - fig:qs3-ncep-corte}\label{fig:qs3-ncep-corte}
\end{figure*}

\begin{figure*}
\includegraphics[width=150mm, ]{fig/tesis/qs3sd-ncep-1} \caption{Desvío estándar de la reconstrucción de QS3. - fig:qs3sd-ncep}\label{fig:qs3sd-ncep}
\end{figure*}

\todo[inline]{No tengo idea de cómo interpretar la \autoref{fig:qs3sd-ncep}. Quizás hablar de que la variabilidad asociada a la onda 3 se da pricipalmente en los nodos y que entonces ahí es donde hay que ver la variación del efecto? no sé... }

\subsection{Wavelets}\label{wavelets}

\begin{figure}

{\centering \includegraphics[width=130mm, ]{fig/tesis/wavelet-fourier-ncep-1} 

}

\caption{Amplitud de wavelets (sombreados) y de fourier (contornos) - fig:wavelet-fourier-ncep}\label{fig:wavelet-fourier-ncep}
\end{figure}

Para la QS3, la amplitud media obtenida mediante Wavelets
\todo{agregar referencia a sección} es virtualmente idéntica a la
amplitud obtenida con Fourier (\autoref{fig:wavelet-fourier-ncep}). Sin
embargo, también permite obtener información de la variación meridional
de la amplitud de la QS3. Al igual que con Fourier, esto puede hacerse
promediando estacionalmente la amplitud de los campos mensuales o
calculando la amplitud de los campos estacionales. Pero a diferencia de
Fourier, los resultados de cada metodología son opuestos. Esto es
evidente en la \autoref{fig:waveletz-ncep} donde se muestra el campo de
amplitud de la QS3 según wavelets en líneas y su anomalía zonal en
sombreado. Los valores positivos representan regiones donde la amplitud
de la QS3 es mayor que la media zonal y viceversa.

No existe mucha diferencia en la amplitud media zonal entre ambas
metodologías a excepción de que la primera siempre es mayor que la
segunda
\todo{Esta es una propiedad general que SIEMPRE se cumple. En realidad podría ir en metodología}.
Las principales diferencias se dan en las anomalías zonales.

\begin{figure}

{\centering \includegraphics[width=150mm, ]{fig/tesis/waveletz-ncep-1} 

}

\caption{Campo medio de la amplitud de la onda 3 según wavelets (contornos) y su anomalía zonal (sombreado) en 300hPa. - fig:waveletz-ncep}\label{fig:waveletz-ncep}
\end{figure}

Las anomalías zonales presentan, en todas las estaciones, un máximo al
sur del Índico centrado en 45°S que se desplaza hacia el este en
latitudes más altas. Existe un segundo máximo en altas latitudes que en
otoño y primavera se encuentran en 120°E y en invierno se encuentra en
180°. En verano éste no aparece, pero sí existe un máximo centrado en
15°S y 120°O.

La principal diferencia con la media de la amplitud es que las mayores
anomalías zonales se dan en latitudes altas (al rededor de 60°S) con un
mínimo al sur del Índico en vez de un máximo. Al norte de 30°S, las
diferencias son menores.

\todo{Estoy pensando en no poner la \autoref{fig:wavelet-ncep-corte} porque no encuentro mucho para decir y no veo que se gane demasiado. Queda para el doctorado? :P}

\begin{figure}

{\centering \includegraphics[width=227mm,angle=90]{fig/tesis/wavelet-ncep-corte-1} 

}

\caption{Corte zonal en -60° de la amplitud media de la onda 3 según wavelets (contornos) y su anomalía zonal (sombreado). - fig:wavelet-ncep-corte}\label{fig:wavelet-ncep-corte}
\end{figure}

Cosas para ver:\\
* Si bien el máximo medio de la amplitud se da en
\textasciitilde{}300hPa (\autoref{fig:wavelet-fourier-ncep}) igual que
en fourier, el análisis por longitud muestra algo un poco más complejo.
En verano y otoño, la máxima amplitud sigue en 300hPa, pero ésta
asciente a entre 100 y 50hPa en invierno y primavera al rederor de
120°O. Además, durante estas estaciones hay indicación de que el máximo
cambia de latitud con la altura.

\begin{figure}
\subfloat[campo\label{fig:estacionaridad1}]{\includegraphics[ ]{fig/tesis/estacionaridad-1} }\newline\subfloat[media zonal\label{fig:estacionaridad2}]{\includegraphics[ ]{fig/tesis/estacionaridad-2} }\caption{Ratio de amplitud de la media y media de la amplitud (¿medida de estacionaridad?) - fig:estacionaridad - SÓLO BORRADOR}\label{fig:estacionaridad}
\end{figure}

Las diferencias vistas en la \autoref{fig:waveletz-ncep} están
relacionadas con la estacionariedad\todo{¿Es una palabra?} de las ondas
\todo{de nuevo, esto es general para la metodología, la explicación va en métodos},
se puede utilizar la relación entre ambas para generar un \emph{índice
de estacionariedad} \todo{¿Estoy seguro de que esto es así?}. Cuanto más
similar sea el resutlado, más estacionarias son las ondas. Así, la
estacionariedad sería máxima en latitudes bajas y al rederor de 50°S
--principalmente en verano-- y mínima en dos franjas angostas cerca de
30°S y 75°S. \todo{¿Mostrar el gŕafico \ref{fig:estacionaridad}?}
\todo[inline]{Esta discusión también aplica a fourier (\autoref{fig:estacionaridad-fourier}), en realidad, pero ahí no estoy mostrando ambos resultados porque da parecido. Quizás este párrafo se puede mover.. o borrar todo si resulta puro sinsentido.}

\begin{figure}
\includegraphics[ ]{fig/tesis/estacionaridad-fourier-1} \caption{Estacionaridad según fourier - fig:estacionaridad-fourier - SÓLO BORRADOR}\label{fig:estacionaridad-fourier}
\end{figure}

Estas observaciones destacan la utilidad de wavelets en el análisis de
ondas cuasiestacionarias. Mientras que el tratamiento con fourier asume
a las ondas como una propiedad media de cada círculo de latitud,
wavelets permite reconocer su heterogeneidad meridional. Evaluando esta
heterogeneidad, sería posible distinguir entre campos donde una onda con
un determinado número de onda está presente en todo un círuclo de
latitud de campos donde ésta está localizada en una región acotada.

Por otro lado, la no ortogonalidad de los wavelets complejizan la
interpretación de los resultados ya que no es posible la separación de
un campo en modos oscilatorios con distinto número de onda. El análisis
de una QS específica, por lo tanto está contaminado por la actividad de
otras QS con longitud de onda cercana.

\begin{figure}
\includegraphics[ ]{fig/tesis/wavelet-reconstr-1} \caption{Reconstrucción de QS3 usando wavelets - fig:wavelet-reconstr - SÓLO BORRADOR}\label{fig:wavelet-reconstr}
\end{figure}

Wavelets, en resumen, puede entenderse como una \emph{corrección} a
fourier que agrega información de asimetrías zonales. Dado que la
variabilidad zonal es del orden de un 10\% de la amplitud media, en lo
que sigue de la tesis se utilizará sólo fourier, dejando el análisis e
interpretación de wavelets para futuros trabajos.

\section{Antecedentes}\label{antecedentes}

Breve comentario sobre los índices usados en otros lados. Discutir
ventajas y debilidades.

\begin{itemize}
\tightlist
\item
  Amplitud
\item
  Fase (impacto en SA)
\end{itemize}

De todo eso, motiva decisión del índice.

\begin{itemize}
\tightlist
\item
  Niveles elegidos
\item
  Promedio vs.~máximo
\item
  Composiciones de campos y flujos.
\item
  Decisión del índice.
\end{itemize}

Quiero hacer el íncide a partir de la actividad de la onda 3 tomando la
región del máximo (latitud entre -65 y -40, y entre 700 y 100 hPa).
Variables posibles: amplitud media, amplitud máxima, r2, correlación
entre campo teórico y observado.

\section{Amplitud}\label{amplitud}

\subsection{Máximo o media.}\label{maximo-o-media.}

Existen varios estadísticos que podrían utilizarse para representar la
amplitud de la QS3 en una región extendida. La media, la máxima, la
moda, la mediana, etc\ldots{} En este caso, se estudió la posibilidad de
representarla con la media o la
máxima.\todo{esto fue escrito a las 23:30 luego de un largo día de mirar campos y números... es un delirio.}

\begin{figure*}
\includegraphics[width=150mm, ]{fig/tesis/ampl-max-mean-1} \caption{Distribució de amplitud para 12 fechas. En rojo la amplitud máxima, en azul la amplitud media. - fig:ampl-max-mean}\label{fig:ampl-max-mean}
\end{figure*}

\begin{figure*}
\includegraphics[width=150mm, ]{fig/tesis/ampl-max-mean-corte-1} \caption{Corte vertical de amplitud - fig:ampl-max-mean-corte}\label{fig:ampl-max-mean-corte}
\end{figure*}

\begin{figure*}
\includegraphics[width=150mm, ]{fig/tesis/ghz-ncep-select-1} \caption{Anomalía zonal geopotencial en 300hPa para fechas seleccionadas. - fig:ghz-ncep-select}\label{fig:ghz-ncep-select}
\end{figure*}

Se seleccionaron manualmente 9 casos que representan distintas
características de la amplitud media la máxima. Sus valores se muestran
en la \autoref{fig:ampl-max-mean}, los cortes meridionales de amplitud,
en la \autoref{fig:ampl-mac-mean-corte} y los camps de Z* (con las QS1 y
QS2 restadas) en la \autoref{fig:ghz-ncep-select}.

Comparando mayo de 1997 con abril de 2012, ambos tienen una media muy
similar, pero la máxima del primero es menor que la del segundo.
Observando el campo de geopotencial, la QS3 se aprecia mucho más
claramente en 2012 que en 1997.

Enero de 1985 y julio de 1988 son un caso similar en cuanto a la
relación de las métricas de amplitud, pero en este
caso\todo{repite "caso"} ambos campos de Z* son muy similares en cuanto
a intensidad y claridad de la QS3. Los dos meses presentan un tren de
ondas que ocupan aproximadamente 1/3 de círculo de latitud. A pesar de
que la amplitud máxima de 1985 es menor que la de 1988, el tren de ondas
de 1988 se ve algo más claro que el de 1985.

El par septiembre de 2000 y diciembre de 1990 es más claro. Ambas
medidas de amplitud son mayores en 1990 y, en efecto, el campo de Z*
tiene una estructura de QS3 zonal más clara que el de 2000. Sin embargo,
las anomalías sí están presentes en 1990 --un tren de ondas similar al
de enero de 1985, aunque con distinta fase-- son más intensas, por lo
que su efecto local puede ser mayor que las de 2000.

Más extrema aún es la diferencia entre septiembre de 2000 y octubre de
2003. Ambos meses tienen una métrica de amplitud similares, pero la QS3
es apenas distinguible en el segundo mes.

Estos casos ilustran algunas limitaciones del análisis que continua.
Algunos problemas son inherentes al intentar representar una estructura
con variabilidad espacial a partir de un sólo número y otros están
atados a la limitación de la descomposición de Fourier que trata toda
onda como una onda
planetaria.\todo{Aclarar qué comparación ilustra qué problema}

\begin{figure}
\includegraphics[ ]{fig/tesis/cor-mean-max-1} \caption{Correlación entre amplitud máxima y media.{fig:cor-mean-max}}\label{fig:cor-mean-max}
\end{figure}

Es importante tener encuenta que estos 9 casos fueron seleccionados
específicamente para ilustrar estas limitaciones y que no son
necesariamente representativos de la totalidad de casos posibles. Como
se muestra en la \autoref{fig:cor-mean-max}, la amplitud media máxima
tienen una excelente correlación (\(r^2>0.9\)) y una relación lineal.
Debido a esto, a fines estadísticos la elección de una métrica o la otra
no tiene una influencia importante. Desde este momento se usará la
amplitud media\todo{expresar mejor esto}.

El ciclo anual de la amplitud media se muestra en la
Figura~\ref{fig:ampl-ts1} donde los puntos son datos puntuales. La
amplitud máxima en invierno es evidente, así como la mayor variabilidad.
Esto es de esperarse en una variable definida positiva que toma valores
cercanos a cero. Notar que la amplitud media no es mínima en primavera,
en contraste con lo observado en el análisis climatológico
(\autoref{fig:ampl-ncep}) y los campos reconstruidos
(\autoref{fig:qs3-ncep}). La resolución a este problema radica en el
análisis de la fase (\autoref{fase}).

La serie temporal de la amplitud media se muestra en la
Figura~\ref{fig:ampl-ts2} donde en líneas horizontales se marca la
amplitud media anual para cada año. No hay evidencia clara de
periodicidades aunque se puede observar indicios de un ciclo de dos años
en la anticorrelación de los promedios anuales. Dicha periodicidad no es
estadísticamente significativa (no se muestra\todo{¿no se muestra?})

\begin{figure*}
\subfloat[Ciclo anual\label{fig:ampl-ts1}]{\includegraphics[width=150mm, ]{fig/tesis/ampl-ts-1} }\newline\subfloat[Serie temporal\label{fig:ampl-ts2}]{\includegraphics[width=150mm, ]{fig/tesis/ampl-ts-2} }\caption{Amplitud media - fig:ampl-ts}\label{fig:ampl-ts}
\end{figure*}

\begin{figure}
\includegraphics[ ]{fig/tesis/acf-ampl-1} \caption{Función de autocorrelación para la amplitud media anual de la onda 3 - fig:acf-ampl - SÓLO BORRADOR}\label{fig:acf-ampl}
\end{figure}

\section{Fase}\label{fase}

\begin{figure}
\includegraphics[ ]{fig/tesis/fase-boxplot-1} \caption{Ciclo anual de la fase (20 mayores amplitudes para cada mes) - fig:fase-boxplot}\label{fig:fase-boxplot}
\end{figure}

\begin{longtable}[]{@{}lr@{}}
\caption{Desvío estándar de la fase para cada estación - NO SE
PUBLICA}\tabularnewline
\toprule
Estación & SD\tabularnewline
\midrule
\endfirsthead
\toprule
Estación & SD\tabularnewline
\midrule
\endhead
Verano & 21.20765\tabularnewline
Otoño & 22.25817\tabularnewline
Invierno & 25.08954\tabularnewline
Primavera & 29.41456\tabularnewline
\bottomrule
\end{longtable}

Además de la amplitud, las sondas planterias se caracterizan por su
fase. La \autoref{fig:fase-boxplot} muestra la fase media y el rango
delimitado por \(\pm\) 1 desvío estándar para cada mes (notar que la
variación en latitud es únicamente a modo de separar los meses). En la
\autoref{fig:qs3-ncep} ya se observada el ciclo anual de la fase, pero
en la \autoref{fig:fase-boxplot} también brinda una idea de la
variabilidad existente año a año.

La gran variabilidad presente durante los meses de primavera, en
comparación con el resto del año, explica pór qué en los campos medios
la QS3 aparece débil a pesar de que su amplitud mensual no es menor. Al
hacer el promedio, los campos que están defasados en entre 1/4 y 3/4 de
longitud de onda (entre 30° y 90° en el caso de la QS3) interfieren
destructivamente entre ellos, eliminando la señal en los campos medios.
En primavera, más del 30\% de los meses tienen generan algún nivel de
interferencia destructiva con el campo medio, comparado con el 13\% en
verano.

\todo[inline]{Hasta acá.}

\begin{longtable}[]{@{}lr@{}}
\caption{Porcentaje de meses con interferencia destructiva - NO SE
PUBLICA}\tabularnewline
\toprule
season & V1\tabularnewline
\midrule
\endfirsthead
\toprule
season & V1\tabularnewline
\midrule
\endhead
Verano & 12.903\tabularnewline
Otoño & 18.280\tabularnewline
Invierno & 26.882\tabularnewline
Primavera & 31.183\tabularnewline
\bottomrule
\end{longtable}

Ciclo anual de la fase. Cosas para ver:\\
Se ve el ciclo anual que ya se veía en una figura anterior. Pero hay
mucha variabilidad.\\
Recordar que este es aproximadamente el centro del máximo de
geopotencial y que, por construcción, hay un centro de mínima a 60°. Eso
implica que en los casos donde hay puntos cerca de -120° o 0°, hay el
centro de mínima está en \textasciitilde{}-60°.\\
También notar que, como se trata en realidad de una variable cíclica,
los puntos extremos representan una situación similar.\\
¿Hay mucha diferencia entre el boxplot con todos los datos y el de sólo
los 20 más intensos? Yo no veo mucha y me parece que excluir a los 10
más débiles invita más preguntas que las que responde.

\begin{figure}

{\centering \includegraphics[width=227mm,angle=90]{fig/tesis/centros-10max-1} 

}

\caption{Centros de máxima (puntos rojos) y mínima (puntos azules) para los 10 años con mayor amplitud de la onda 3, para cada mes.{fig:centros-10max}}\label{fig:centros-10max}
\end{figure}

Cosas para ver:\\
La mayoría del tiempo estamos afectados por centro de máxima, pero hay
casos intensos donde hay mínima. Especialmente en junio y julio.

\section{Estaciones}\label{estaciones}

¿Cómo establecer una ``estacionalidad''? Algún criterio para agrupar
meses. Claramente las estaciones climatológicas de siempre no sirven.

Una forma es, a partir del índice, agrupar los meses según su amplitud y
fase media:

\begin{figure}
\includegraphics[ ]{fig/tesis/ampl-vs-fase-1} \caption{Amplitud y fase media para cada mes.{fig:ampl-vs-fase}}\label{fig:ampl-vs-fase}
\end{figure}

Otro ejemplo podría usar SVD:

\begin{figure}
\subfloat[Varianza explicada de cada componente\label{fig:eof1}]{\includegraphics[ ]{fig/tesis/eof-1} }\newline\subfloat[Campo asociado a las primeras 2 componentes\label{fig:eof2}]{\includegraphics[ ]{fig/tesis/eof-2} }\newline\subfloat[Factor de peso para cada mes\label{fig:eof3}]{\includegraphics[ ]{fig/tesis/eof-3} }\caption{Análisis SVD de QS3.probablemente basura{fig:eof}}\label{fig:eof}
\end{figure}

\begin{figure}
\includegraphics[ ]{fig/tesis/eof1-eof2-1} \caption{'Magnitud' y 'Fase' de cada mes a partir de las componentes principales{fig:eof1-eof2}}\label{fig:eof1-eof2}
\end{figure}

Cosas para ver:\\
Los campos son descriptos bien por sólo 2 componentes principales de
igual magnitud que esencialmente describen el corrimiento de fase de la
onda. La ``magnitud'' del vector de componentes mapea bien a la amplitud
de fourier y la diferencia mapea bien a la fasae. No hay nada nuevo por
acá.

Grupos: * Enero + Febrero + Marzo es un grupo bastante seguro. Se ven
juntos en el diagrama amplitud-fase, tienne alta correlación entre ellos
y en componentes principales tienen PC1 significativamente mayor que
PC2. * Junio + Julio parecen ir por su cuenta. Tienen mucha correlación
entre ellos y ambos tienen PC2 mayor que PC1 aunque su fase difiere
bastante.

Los otros meses quedan medio mezclados, pero pueden agruparse como meses
de ``transición''.

\section{Persistencia (ver dónde va)}\label{persistencia-ver-donde-va}

\begin{figure}
\includegraphics[ ]{fig/tesis/lag-cor-1} \caption{Correlación lageada para cada mes con los 12 sigientes.{fig:lag-cor}}\label{fig:lag-cor}
\end{figure}

\section{R2}\label{r2}

Se puede estimar de dos maneras distintas: a partir del ajuste de
fourier para cada nivel y latitud (figura blabla) y haciendo un
promedio, o reconstruyendo el campo tridimensional de la onda 3 y
haciendo la correlación (global) con el campo tridimensional observado.
Esta segunda forma da casi siempre un valor menor.

\begin{figure}
\includegraphics[ ]{fig/tesis/r2-cor2-1} \caption{Relación entre R2 medio y R2 reconstruido.{fig:r2-cor2}}\label{fig:r2-cor2}
\end{figure}

Para ver:\\
La relación ya no es lineal y hay bastante más scatter. Ergo, hay
diferencia en la información.\\
Dos regímenes: Cuando el R\^{}2 es bajo, la relación es ``menor'' que
1:1 y hay algunos casos donde el R2 reconstruido es mayor que el r2
medio. Para R2 más grandes, la pendiente es 1. Modelar la
relación\ldots{} ¿Cómo se interpreta?

\begin{figure*}
\includegraphics[width=150mm, ]{fig/tesis/r2-corte-1} \caption{R2 medio{fig:r2-corte}}\label{fig:r2-corte}
\end{figure*}

Cosas para ver:\\
Comparando con las figuras anteriores, casos donde la amplitud es grande
pero el R2 no tanto.

\begin{figure*}
\subfloat[Serie temporal\label{fig:r2-ts1}]{\includegraphics[width=150mm, ]{fig/tesis/r2-ts-1} }\newline\subfloat[Ciclo anual\label{fig:r2-ts2}]{\includegraphics[width=150mm, ]{fig/tesis/r2-ts-2} }\caption{R2 medio - fig:r2-ts}\label{fig:r2-ts}
\end{figure*}

Cosas para ver:\\
El ciclo anual no es para nada tan claro. Varios outliers. La
correlación reconstruida (azul) no tiene casi ciclo anual.

Una ventaja de la correlación entre el campo real y el reconstruido es
que puede hacerse para cada punto y analizar la variación espacial de la
misma.

\begin{figure}
\includegraphics[ ]{fig/tesis/cor-campo-1} \caption{Correlación cuadrada media para estaciones según onda3.{fig:cor-campo}}\label{fig:cor-campo}
\end{figure}

Cosas para ver:\\
Esto es, para cada punto de grilla, la correlación entre el campo
observado y el reconstruido en todos los meses y años. Además de la
dependencia latitudinal de la importancia de la onda 3 (que se puede ver
en los cortes anteriores), se ve la dependencia zonal. La onda 3 es más
importante en el Pacífico sur que en el Atlántico o el Índico. Además,
se ve un patrón de altas correlaciones que asemejan a un tren de
ondas.\\
¿Confirma? lo que se ve en en análisis de wavelets.

Conclusión: no voy a usar el r2 a partir del campo reconstruido.

\section{Regresiones}\label{regresiones}

\subsection{Geopotencial}\label{geopotencial}

\begin{figure}

{\centering \includegraphics[width=150mm, ]{fig/tesis/regr-gh-ncep-1} 

}

\caption{Regresión sobre amplitud.{fig:regr-gh-ncep}}\label{fig:regr-gh-ncep}
\end{figure}

Cosas para ver:\\
* Además los patrones de onda 3, en julio y diciembre aparece un patrón
de SAM positivo y negativo respectivamente.

\begin{figure}
\includegraphics[ ]{fig/tesis/regr-gh-polar-1} \caption{Igual que figura  XX, pero en proyección polar para julio y septiembre.{fig:regr-gh-polar}}\label{fig:regr-gh-polar}
\end{figure}

\begin{figure}
\includegraphics[ ]{fig/tesis/sam-ampl-1} \caption{Relación entre amplitud media de la onda 3 y el SAM{fig:sam-ampl}}\label{fig:sam-ampl}
\end{figure}

\subsection{Función Corriente}\label{funcion-corriente-1}

\begin{figure}

{\centering \includegraphics[width=150mm, ]{fig/tesis/regr-psi-ncep-1} 

}

\caption{Regresión de Psi con la amplitud. - fig:regr-psi-ncep}\label{fig:regr-psi-ncep}
\end{figure}

\section{Composición de campos.}\label{composicion-de-campos.}

Descripción de la seleccion.

\begin{figure}

{\centering \includegraphics[width=227mm,angle=90]{fig/tesis/seleccion-tabla-1} 

}

\caption{Tabla de selección{fig:seleccion-tabla}}\label{fig:seleccion-tabla}
\end{figure}

Cosas para ver:\\
Años con coincidencia, años sin coindicentica. Meses donde la fase
coincide (julio) vs meses donde no coindice (septiembre). También, años
donde hay seguidilla de meses seleccionados (1999). Aunque posiblemente
sea casualidad (no hay mucha persistencia mes a mes.)

Pequeña digresión: Efecto de la fase.\\
La climatología de la fase se va a discutir más adelante, pero\ldots{}
discutir el efecto de promediar campos con similar amplitud pero fase
distinta. Del gráfico, septiembre tiene 1997 y 2003 con fase a 180°, lo
que significa que va a a haber cancelación parcial. Enero, por el
contrario, no tiene ningún año en contrafase, aunque sí algunos a 90°,
que desdibujan el patrón.

\begin{figure}

{\centering \includegraphics[width=130mm, ]{fig/tesis/interaccion-tabla-1} 

}

\caption{Tabla de interacción{fig:interaccion-tabla}}\label{fig:interaccion-tabla}
\end{figure}

Cosas para ver:\\
Ambos criterios coinciden en casi todos los años seleccionados, así que
no hay mucha diferencia. En efecto, las composiciones son casi iguales
(no se muestra). Voy a usar la amplitud.

\begin{figure}

{\centering \includegraphics[width=227mm,angle=90]{fig/tesis/gh-comp-1} 

}

\caption{Composición de campos{fig:gh-comp}}\label{fig:gh-comp}
\end{figure}

\begin{figure}

{\centering \includegraphics[width=150mm, ]{fig/tesis/gh-qs3-select-ene-1} 

}

\caption{Campos para los 10 eneros seleccionados.{fig:gh-qs3-select-ene}}\label{fig:gh-qs3-select-ene}
\end{figure}

\begin{figure}

{\centering \includegraphics[width=150mm, ]{fig/tesis/gh-qs3-select-sep-1} 

}

\caption{Campos para los 10 septiembres seleccionados.{fig:gh-qs3-select-sep}}\label{fig:gh-qs3-select-sep}
\end{figure}

Estos gráficos me parecen importantes para ver lo que hay ``adentro'' de
la composición, pero no sé bien qué decir sobre ellos. Supongo que lo
principal es que hay años donde la onda

\section{Fuentes de variabilidad
interna}\label{fuentes-de-variabilidad-interna}

(Discusión escrita más de papers), Pero nos concentramos en la fuente
externa.

\section{Fuentes externas}\label{fuentes-externas}

\subsection{SST}\label{sst}

\begin{figure}

{\centering \includegraphics[width=227mm,angle=90]{fig/tesis/regr-sst-ncep-1} 

}

\caption{Regresión de SST con la amplitud de la onda 3{fig:regr-sst-ncep}}\label{fig:regr-sst-ncep}
\end{figure}

Campos de correlación con SST y OLR, principalmente ¿Discusión de otros
forzantes?

\chapter{Experimentos}\label{experimentos}

\section{Validación SPEEDY}\label{validacion-speedy}

Validación de la corrida control

\begin{itemize}
\tightlist
\item
  Comparación campos medios.
\end{itemize}

Acá un problemita es que en speedy no tengo el nivel de 50hPa, sólo
tengo 925, 850, 700, 500, 300, 200, 100, 30. Podría usar 30, pero eso es
la tapa del modelo\ldots{}

\begin{figure}
\includegraphics[ ]{fig/tesis/cor-sp-nc-1} \caption{Correlación lineal entre campos de SPEEDY y NCEP.{fig:cor-sp-nc}}\label{fig:cor-sp-nc}
\end{figure}

\begin{itemize}
\item
  Altura geopotencial (gh) Para el caso del campo total, la correlación
  del campo es buena (\textgreater{}0.8)en casi todos los niveles y
  meses, excepto en 30hPa durante verano donde los campos ¡está
  anticorrelacionados! La parte asimétrica zonal muestra valores
  menores, indicando que gran parte de la correlación del campo total se
  debe a la capacidad del modelo de reproducir el gradiente meridional.
  Sin embargo, se siguen obteniendo correlaciones \textgreater{}0.6 en
  casi todos los niveles y estaciones. Se observa un mínimo relativo en
  500hPa donde en se tienen correlaciones menores durante casi todo el
  año y uno en niveles altos centrado en invierno y primavera.
\item
  Viento zonal (U) Las correlaciones con el campo total son
  \textgreater{}=0.8 en todo el año y todos los niveles, sin embargo, la
  parte asimétrica muestra correlaciones mucho más baja con un máximo de
  \textasciitilde{}0.6 en 925hPa. Esto indica que el modelo resuelve
  correctamente la estructura media del Jet, pero no sus variaciones
  zonales.
\item
  Viento meridional (V) Los campos de correlación son prácticamente
  idénticos entre parte total y parte asimétrica. Ésta muestra un patrón
  de bajas correlaciones en general.
\item
  Temperatura (T) La correlación con el campo total muestra una
  estructura similar que la altura geopotencial, con una excelente
  correlación en todos los meses para niveles mayores a 200hPa, pero
  anticorrelacionado en niveles altos en todos los meses salvo en
  invierno. La parte asimétrica muestra correlaciones bajas en todos los
  niveles salvo en 925hPa.
\item
  Gradiente meridional de vorticidad absoluta. Tiene correlación
  moderada con un mínimo relativo en 200hPa en verano que en las otras
  estacione se convierte en un máximo.
\end{itemize}

\subsection{Altura Geopotencial}\label{altura-geopotencial-1}

Anomalía

\begin{figure}

{\centering \includegraphics[width=227mm,angle=90]{fig/tesis/ghz-sp-nc-1} 

}

\caption{Anomalía zonal de altura geopotencial (speedy sombreado, ncep contornos){fig:ghz-sp-nc}}\label{fig:ghz-sp-nc}
\end{figure}

\begin{figure}

{\centering \includegraphics[width=227mm,angle=90]{fig/tesis/ghz-dif-sp-nc-1} 

}

\caption{Diferencia entre speedy y ncep{fig:ghz-dif-sp-nc}}\label{fig:ghz-dif-sp-nc}
\end{figure}

Veredicto:\\
* Agarra bien la anomalía zonal aunque con magnitud menor.

\begin{figure}

{\centering \includegraphics[width=130mm, ]{fig/tesis/ghz-sp-nc-corte60-1} 

}

\caption{Corte zonal de anomalía de geopotencial en -60° (speedy sombreado, ncep contornos).{fig:ghz-sp-nc-corte60}}\label{fig:ghz-sp-nc-corte60}
\end{figure}

El corte zonal evidencia que además de tener menor amplitud, la
estructura vertical de las anomalías es barotrópica equivalente en
Speedy, a diferencia de la estructura baroclínica de NCEP.

\subsection{Temperatura}\label{temperatura-1}

\begin{figure}
\includegraphics[ ]{fig/tesis/t-nc-sp-1} \caption{Temperatura{fig:t-nc-sp}}\label{fig:t-nc-sp}
\end{figure}

\begin{figure}
\includegraphics[ ]{fig/tesis/tz-sp-nc-1} \caption{T*{fig:tz-sp-nc}}\label{fig:tz-sp-nc}
\end{figure}

En 850 tiene una fuerte onda 1 en latitudes polares producida por la
topografía de la Antártida y representa bien las anomalías causadas por
los continentes, aunque en menor amplitud.\\
En 200, falla miserablemente. La onda 1 polar que se ve bien claro en
NCEP ni aparece en SPEEDY, mientras que en latitudes medias aparecen
ligeras anomalías que no se observan en las observaciones.

\subsection{Viento zonal}\label{viento-zonal-1}

\begin{figure}

{\centering \includegraphics[width=130mm, ]{fig/tesis/u-sp-nc-corte-1} 

}

\caption{Viento zonal medio (speedy contornos, ncep sombreado). - fig:u-sp-nc-corte}\label{fig:u-sp-nc-corte}
\end{figure}

Cosas para ver:\\
* Speedy no logra desarrollar un jet polar por la falta de niveles
verticales en la estratósfera. Tampoco reproduce los estes
estratosféricos en latitudes bajas. Su jet subtropical es más intenso y
su máximo se da ligéramente en niveles más altos en NCEP.

Campo medio (me parece que no lo voy a poner, no agrega información que
no esté en el anterior y en el de geopotencial.)

\begin{figure}

{\centering \includegraphics[width=227mm,angle=90]{fig/tesis/u-sp-nc-1} 

}

\caption{Viento zonal (contornos ncep, sombreado speedy).{fig:u-sp-nc}}\label{fig:u-sp-nc}
\end{figure}

\begin{figure}
\includegraphics[ ]{fig/tesis/u-dif-sp-nc-1} \caption{Diferencia entre ncep y speedy en viento zonal{fig:u-dif-sp-nc}}\label{fig:u-dif-sp-nc}
\end{figure}

Cosas para ver:\\
Jet polar en invierno y primavera en niveles altos (\textless{} 100
hPa). Jest subtropical en niveles ``medios''.

\subsection{Gradiente meridional de vorticidad
absoluta}\label{gradiente-meridional-de-vorticidad-absoluta-1}

\begin{figure}

{\centering \includegraphics[width=227mm,angle=90]{fig/tesis/etady-sp-nc-1} 

}

\caption{Gradiente meridional de vorticidad absoluta (speedy).{fig:etady-sp-nc}}\label{fig:etady-sp-nc}
\end{figure}

Comparando con la figura \autoref{fig:eta-dy-ncep}, los gradientes son
menores y más zonales. Es significativo que la región ``prohibida'' en
invierno de niveles altos es menor en 200hPa y casi desaparece en
300hPa.

\subsection{Número de onda
estacionaria}\label{numero-de-onda-estacionaria-1}

\begin{figure}
\includegraphics[ ]{fig/tesis/ks-sp-1} \caption{Número de onda estacionario en 300hPa (speedy).{fig:ks-sp}}\label{fig:ks-sp}
\end{figure}

Comparando con \autoref{fig:k-steady-ncep} la mayor diferencia es la
desaparición de una región de propagación impedida en
\textasciitilde{}-40° en el Índico y el Pacífico en Otoño.

\begin{figure}

{\centering \includegraphics[width=130mm, ]{fig/tesis/ks-sp-nc-corte-1} 

}

\caption{Número de onda estacionario medio por círculo de latitud.{fig:ks-sp-nc-corte}}\label{fig:ks-sp-nc-corte}
\end{figure}

En promedio zonal, sin embargo, SPEEDY funciona bien.

\subsection{Función corriente}\label{funcion-corriente-2}

\begin{figure}

{\centering \includegraphics[width=227mm,angle=90]{fig/tesis/psi-sp-1} 

}

\caption{Función corriente x 1099{fig:psi-sp}}\label{fig:psi-sp}
\end{figure}

\subsection{Onda 3}\label{onda-3-1}

\begin{figure*}
\includegraphics[width=150mm, ]{fig/tesis/ampl-sp-nc-1} \caption{Amplitud de Fourier (speedy en sombreado, ncep en contornos).{fig:ampl-sp-nc}}\label{fig:ampl-sp-nc}
\end{figure*}

\begin{figure*}
\includegraphics[width=150mm, ]{fig/tesis/qs2-sp-nc-1} \caption{Media de reconstrucción de onda 3 (sombreado speedy, contornos ncep){fig:qs2-sp-nc}}\label{fig:qs2-sp-nc}
\end{figure*}

La onda 3 no está muy bien representada en el modelo. Aunque la
estructura vertical y la posición meridional está bien (salvo en otoño),
la amplitud es mucho menor. La fase, además, está corrida ligeramente en
verano, pero quedando en cuadratura en invierno y defasado 180 en
primavera.

\section{Comparación}\label{comparacion}

Comparación entre corridas \#\#\# Altura geopotencial

\begin{figure}

{\centering \includegraphics[width=227mm,angle=90]{fig/tesis/ghz-sp-runs-1} 

}

\caption{Anomalía zonal de altura geopotencial.{fig:ghz-sp-runs}}\label{fig:ghz-sp-runs}
\end{figure}

\begin{figure}
\includegraphics[ ]{fig/tesis/ghz-dif-sp-runs-1} \caption{Diferencia Control - corrida para Z* - fig:ghz-dif-sp-runs}\label{fig:ghz-dif-sp-runs}
\end{figure}

\subsection{Temperatura}\label{temperatura-2}

ONo hay casi diferencia entre las corridas.

\begin{figure}
\includegraphics[ ]{fig/tesis/t-sp-runs-1} \caption{Temperatura media en 850hPa.{fig:t-sp-runs}}\label{fig:t-sp-runs}
\end{figure}

\begin{figure}
\includegraphics[ ]{fig/tesis/tz-sp-runs-1} \caption{Temperatura media en 850hPa.{fig:tz-sp-runs}}\label{fig:tz-sp-runs}
\end{figure}

\begin{figure}

{\centering \includegraphics[width=227mm,angle=90]{fig/tesis/tz-dif-sp-runs-1} 

}

\caption{Diferencia Control - corrida para T* - fig:tz-dif-sp-runs}\label{fig:tz-dif-sp-runs}
\end{figure}

\subsection{Viento zonal}\label{viento-zonal-2}

\begin{figure}

{\centering \includegraphics[width=227mm,angle=90]{fig/tesis/uz-sp-runs-1} 

}

\caption{Viento zonal{fig:uz-sp-runs}}\label{fig:uz-sp-runs}
\end{figure}

\begin{figure}
\includegraphics[ ]{fig/tesis/u-dif-sp-runs-1} \caption{Diferencia control - corrida para U. - fig:u-dif-sp-runs}\label{fig:u-dif-sp-runs}
\end{figure}

\subsection{Función corriente}\label{funcion-corriente-3}

\begin{figure}

{\centering \includegraphics[width=227mm,angle=90]{fig/tesis/psi-sp-runs-1} 

}

\caption{Anomalía zonal de función corriente y flujos de acción de onda.{fig:psi-sp-runs}}\label{fig:psi-sp-runs}
\end{figure}

\begin{figure}
\includegraphics[ ]{fig/tesis/psiz-dif-sp-runs-1} \caption{Diferencia en psi.z y flujos de acción de onda.{fig:psiz-dif-sp-runs}}\label{fig:psiz-dif-sp-runs}
\end{figure}

\subsection{Onda 3}\label{onda-3-2}

\begin{figure}
\includegraphics[ ]{fig/tesis/ampl-sp-runs-1} \caption{Amplitud media de la onda 3 para cada corrida.{fig:ampl-sp-runs}}\label{fig:ampl-sp-runs}
\end{figure}

\begin{figure}
\includegraphics[ ]{fig/tesis/ampl-dif-sp-runs-1} \caption{Diferencia de amplitud entre la corrida control y cada corrida.{fig:ampl-dif-sp-runs}}\label{fig:ampl-dif-sp-runs}
\end{figure}

\begin{figure}
\includegraphics[ ]{fig/tesis/index-sp-boxplot-1} \caption{Ciclo anual de amplitud de onda 3.{fig:index-sp-boxplot}}\label{fig:index-sp-boxplot}
\end{figure}

\section{Regresión}\label{regresion}

\begin{figure}
\includegraphics[ ]{fig/tesis/regr-psi-sp-runs-1} \caption{Regresión en función corriente. - fig:regr-psi-sp-runs}\label{fig:regr-psi-sp-runs}
\end{figure}

\section{Cosas inesperadas\ldots{}}\label{cosas-inesperadas}

\begin{itemize}
\tightlist
\item
  ??
\item
  protif!
\end{itemize}

\chapter{Conclusiones}\label{conclusiones}

\chapter{Agradecimientos}\label{agradecimientos}

\chapter*{Referencias}\label{referencias}
\addcontentsline{toc}{chapter}{Referencias}

\hypertarget{refs}{}
\hypertarget{ref-Berbery1992}{}
Berbery, E H, J Nogués-Paegle, y J D Horel. 1992. «Wavelike southern
hemisphere extratropical teleconnections».
doi:\href{https://doi.org/DOI:\%2010.1175/1520-0469(1992)049\%3C0155:WSHET\%3E2.0.CO;2}{DOI: 10.1175/1520-0469(1992)049\textless{}0155:WSHET\textgreater{}2.0.CO;2}.

\hypertarget{ref-James}{}
James, I. N. 1994. \emph{Introduction to circulating atmospheres}.
Cambridge: Cambridge University Press.
doi:\href{https://doi.org/10.1017/CBO9780511622977}{10.1017/CBO9780511622977}.

\hypertarget{ref-Quintanar1995}{}
Quintanar, Arturo I., y Carlos R. Mechoso. 1995. «Quasi-Stationary Waves
in the Southern Hemisphere. Part II: Generation Mechanisms».
\emph{Journal of Climate} 8 (11): 2673-90.
doi:\href{https://doi.org/10.1175/1520-0442(1995)008\%3C2673:QSWITS\%3E2.0.CO;2}{10.1175/1520-0442(1995)008\textless{}2673:QSWITS\textgreater{}2.0.CO;2}.

\hypertarget{ref-Trenberth1985}{}
Trenberth, Kevin E., y K. C. Mo. 1985. «Blocking in the Southern
Hemisphere».
doi:\href{https://doi.org/10.1175/1520-0493(1985)113\%3C0003:BITSH\%3E2.0.CO;2}{10.1175/1520-0493(1985)113\textless{}0003:BITSH\textgreater{}2.0.CO;2}.

\hypertarget{ref-Vera2004}{}
Vera, Carolina, Gabriel Silvestri, Vicente Barros, y Andrea Carril.
2004. «Differences in El Niño response over the Southern Hemisphere».
\emph{Journal of Climate} 17 (9): 1741-53.
doi:\href{https://doi.org/10.1175/1520-0442(2004)017\%3C1741:DIENRO\%3E2.0.CO;2}{10.1175/1520-0442(2004)017\textless{}1741:DIENRO\textgreater{}2.0.CO;2}.


\end{document}
