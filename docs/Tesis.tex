\documentclass[spanish,a4paper]{book}
\usepackage{lmodern}
\usepackage{amssymb,amsmath}
\usepackage{ifxetex,ifluatex}
\usepackage{fixltx2e} % provides \textsubscript
\ifnum 0\ifxetex 1\fi\ifluatex 1\fi=0 % if pdftex
  \usepackage[T1]{fontenc}
  \usepackage[utf8]{inputenc}
\else % if luatex or xelatex
  \ifxetex
    \usepackage{mathspec}
  \else
    \usepackage{fontspec}
  \fi
  \defaultfontfeatures{Ligatures=TeX,Scale=MatchLowercase}
\fi
% use upquote if available, for straight quotes in verbatim environments
\IfFileExists{upquote.sty}{\usepackage{upquote}}{}
% use microtype if available
\IfFileExists{microtype.sty}{%
\usepackage{microtype}
\UseMicrotypeSet[protrusion]{basicmath} % disable protrusion for tt fonts
}{}
\usepackage[inner = 3cm, outer = 2cm, top = 2.5cm, bottom = 2.5cm]{geometry}
\usepackage{hyperref}
\hypersetup{unicode=true,
            pdftitle={Tesis},
            pdfauthor={Elio Campitelli},
            pdfborder={0 0 0},
            breaklinks=true}
\urlstyle{same}  % don't use monospace font for urls
\ifnum 0\ifxetex 1\fi\ifluatex 1\fi=0 % if pdftex
  \usepackage[shorthands=off,main=spanish]{babel}
\else
  \usepackage{polyglossia}
  \setmainlanguage[]{spanish}
\fi
\usepackage{graphicx,grffile}
\makeatletter
\def\maxwidth{\ifdim\Gin@nat@width>\linewidth\linewidth\else\Gin@nat@width\fi}
\def\maxheight{\ifdim\Gin@nat@height>\textheight\textheight\else\Gin@nat@height\fi}
\makeatother
% Scale images if necessary, so that they will not overflow the page
% margins by default, and it is still possible to overwrite the defaults
% using explicit options in \includegraphics[width, height, ...]{}
\setkeys{Gin}{width=\maxwidth,height=\maxheight,keepaspectratio}
\IfFileExists{parskip.sty}{%
\usepackage{parskip}
}{% else
\setlength{\parindent}{0pt}
\setlength{\parskip}{6pt plus 2pt minus 1pt}
}
\setlength{\emergencystretch}{3em}  % prevent overfull lines
\providecommand{\tightlist}{%
  \setlength{\itemsep}{0pt}\setlength{\parskip}{0pt}}
\setcounter{secnumdepth}{5}
% Redefines (sub)paragraphs to behave more like sections
\ifx\paragraph\undefined\else
\let\oldparagraph\paragraph
\renewcommand{\paragraph}[1]{\oldparagraph{#1}\mbox{}}
\fi
\ifx\subparagraph\undefined\else
\let\oldsubparagraph\subparagraph
\renewcommand{\subparagraph}[1]{\oldsubparagraph{#1}\mbox{}}
\fi

%%% Use protect on footnotes to avoid problems with footnotes in titles
\let\rmarkdownfootnote\footnote%
\def\footnote{\protect\rmarkdownfootnote}

%%% Change title format to be more compact
\usepackage{titling}

% Create subtitle command for use in maketitle
\newcommand{\subtitle}[1]{
  \posttitle{
    \begin{center}\large#1\end{center}
    }
}

\setlength{\droptitle}{-2em}
  \title{Tesis}
  \pretitle{\vspace{\droptitle}\centering\huge}
  \posttitle{\par}
\subtitle{Una tesis}
  \author{Elio Campitelli}
  \preauthor{\centering\large\emph}
  \postauthor{\par}
  \date{}
  \predate{}\postdate{}

\linespread{1.25}
\usepackage{subfig}
\usepackage{hyperref}
\usepackage{marginnote}
\usepackage[nomarkers,figuresonly]{endfloat}

\begin{document}
\maketitle

{
\setcounter{tocdepth}{3}
\tableofcontents
}
Resumen.

\chapter{Introducción}\label{introduccion}

\begin{itemize}
\tightlist
\item
  Antecedentes\\
  Además de lo que hay en lo de las becas + lo que fui encontrando,
  agregar sobre las climatologías disponibles y sus limitaciones.
\item
  Objetivo General
\item
  Objetivo particular
\end{itemize}

Esto es para probar una referencia bibliográfica: Vera et~al.
(\protect\hyperlink{ref-Vera2004}{2004}) y (Vera et~al.
\protect\hyperlink{ref-Vera2004}{2004})

\chapter{Métodos y Materiales}\label{metodos-y-materiales}

\section{Conceptos básicos}\label{conceptos-basicos}

\begin{itemize}
\tightlist
\item
  Ondas cuasiestacionarias
\item
  Fourier
\end{itemize}

Ejemplo:

\begin{figure*}
\includegraphics[width=150mm, ]{fig/tesis/fourier-ejemplo-1} \caption{Ejemplo fourier - fig:fourier-ejemplo}\label{fig:fourier-ejemplo}
\end{figure*}

Cosas para ver de \autoref{fig:fourier-ejemplo}:\\
Descripción del ``rol'' de cada número de onda en generar el campo
final. La onda 1 es la principal, marcando altas presiones al sur del
pacífico y bajas al sur de África. La onda 3 modifica ese patrón simple
haciendo que los máximos y mínimos no sean contínuos.

\begin{itemize}
\tightlist
\item
  Wavelets
\end{itemize}

\begin{figure*}
\subfloat[Campo\label{fig:wavelet-ejemplo1}]{\includegraphics[width=75mm, ]{fig/tesis/wavelet-ejemplo-1} }\subfloat[Análisis de Wavelets en círculos de latitud señalados.\label{fig:wavelet-ejemplo2}]{\includegraphics[width=75mm, ]{fig/tesis/wavelet-ejemplo-2} }\caption{Wavelets {fig:wavelet-ejemplo}}\label{fig:wavelet-ejemplo}
\end{figure*}

Cosas para ver:\\
Cambio en el máximo. Localización en vez de un número para cada latitud.

\section{Fuentes de datos}\label{fuentes-de-datos}

\section{Descripción de SPEEDY}\label{descripcion-de-speedy}

\chapter{Climatología observada}\label{climatologia-observada}

\section{Altura geopotencial}\label{altura-geopotencial}

Campo medio:

\begin{figure}

{\centering \includegraphics[width=227mm,angle=90]{fig/tesis/gh-ncep-1} 

}

\caption{Campo de Z (NCEP) - fig:gh-ncep}\label{fig:gh-ncep}
\end{figure}

El campo de altura geopotencial media (Z, \autoref{fig:gh-ncep}) muestra
una estructura más zonal en el HS que en el HN. Es evidente la
generación del vórtice polar en la estratósfera de invierno y primavera
y el aumento del gradiente meridional de Z con la altura.

\begin{figure}
\includegraphics[ ]{fig/tesis/ghdy-ncep-corte-1} \caption{Gradiente meridional de Z - fig:ghdy-ncep-corte}\label{fig:ghdy-ncep-corte}
\end{figure}

\begin{figure}

{\centering \includegraphics[width=227mm,angle=90]{fig/tesis/ghz-ncep-1} 

}

\caption{Anomalía zonal de altura geopotencial.{fig:ghz-ncep}}\label{fig:ghz-ncep}
\end{figure}

Las anomalías zonales (\autoref{fig:ghz-ncep}) muestran una
preponderancia de una onda 1 (QS1) con una amplitud máxima en la
estratósfera de primavera y una estructura coherente en la vertical en
60°S (\autoref{fig:ghz-ncep-corte}). Salvo en verano, la inclinación
hacia el oeste en la horizontal y en la vertical indica que las
perturbaciones estacionarias están asociadas con transporte hacia el
polo tanto de cantidad de movimiento zonal como de temperatura.
\footnote{¿Debería agregar una justificaicón de esto? ¿Un resumen del desarrollo matemático que está en el James?}

En verano, las anomalías zonales tienen una estructura barotrópica
equivalente y carecen de inclinación en la horizontal. Consecuentemente,
los transportes meridionales asociados se reducen considerablemente.
\footnote{¿Por qué?}

\begin{figure}

{\centering \includegraphics[width=130mm, ]{fig/tesis/ghz-ncep-corte60-1} 

}

\caption{Corte zonal de anomalía de geopotencial en -60°.{fig:ghz-ncep-corte60}}\label{fig:ghz-ncep-corte60}
\end{figure}

\begin{figure}
\subfloat[uz*vz*\label{fig:uzvz-ncep-corte1}]{\includegraphics[ ]{fig/tesis/uzvz-ncep-corte-1} }\newline\subfloat[T*v*\label{fig:uzvz-ncep-corte2}]{\includegraphics[ ]{fig/tesis/uzvz-ncep-corte-2} }\caption{Transportes - fig:uzvz-ncep-corte}\label{fig:uzvz-ncep-corte}
\end{figure}

Hasta acá.

Complementa la figura anterior.

Desvío estándar por círculo de latitud:

\begin{figure}

{\centering \includegraphics[width=130mm, ]{fig/tesis/sd-gh-ncep-1} 

}

\caption{Desvío estándar por círculo de latitud.{fig:sd-gh-ncep}}\label{fig:sd-gh-ncep}
\end{figure}

Cosas para ver:\\
Latitud de mayor actividad de onda. Máximo en octubre en 300 hPa. Más
adelante, se hace la misma figura pero con el desvío estándar asociado a
cada número de onda.

\section{Temperatura}\label{temperatura}

\begin{figure}

{\centering \includegraphics[width=227mm,angle=90]{fig/tesis/t-ncep-1} 

}

\caption{Temperatura media. - fig:t-ncep}\label{fig:t-ncep}
\end{figure}

Cosas para ver:\\
Gradiente muy pequeño en 200 hPa. Gradiente inverso en estratósfera.
Núcleo cálido en \textasciitilde{}50° (que se va a ver mejor en la
anomalía zonal). Temperaturas frías en altas y bajas latitudes pero
relativamente cálidas en \textasciitilde{}50° en 100 hPa.

\begin{figure}

{\centering \includegraphics[width=130mm, ]{fig/tesis/tz-ncep-corte-1} 

}

\caption{Corte meridional de temperatura media.{fig:tz-ncep-corte}}\label{fig:tz-ncep-corte}
\end{figure}

\begin{figure}

{\centering \includegraphics[width=227mm,angle=90]{fig/tesis/tz-ncep-1} 

}

\caption{Anomalía zonal de temperatura. - fig:tz-ncep}\label{fig:tz-ncep}
\end{figure}

Corte zonal en -60°

\begin{figure}

{\centering \includegraphics[width=130mm, ]{fig/tesis/tz-ncep-corte60-1} 

}

\caption{Corte zonal de anomalía de temperatura en -60°.{fig:tz-ncep-corte60}}\label{fig:tz-ncep-corte60}
\end{figure}

Cosas para ver:\\
Coincidencia entre la onda estacionaria 1 en gh y de t (en primavera).

Propuesta: combinar mapa de T y T*

\section{Viento zonal}\label{viento-zonal}

\begin{figure}

{\centering \includegraphics[width=130mm, ]{fig/tesis/u-ncep-corte-1} 

}

\caption{Viento zonal medio.{fig:u-ncep-corte}}\label{fig:u-ncep-corte}
\end{figure}

Cosas para ver:\\
Extensión y localización vertical de los jets.

Campo medio:

\begin{figure}

{\centering \includegraphics[width=227mm,angle=90]{fig/tesis/u-ncep-1} 

}

\caption{Viento zonal.{fig:u-ncep}}\label{fig:u-ncep}
\end{figure}

Cosas para ver:\\
Jet polar en invierno y primavera en niveles altos (\textless{} 100
hPa). Jest subtropical en niveles ``medios''.

Anomalía zonal

\begin{figure}

{\centering \includegraphics[width=227mm,angle=90]{fig/tesis/uz-ncep-1} 

}

\caption{Anomalía zonal de viento zonal.{fig:uz-ncep}}\label{fig:uz-ncep}
\end{figure}

Cosas para ver (ambos):

\section{Viento meridional}\label{viento-meridional}

Campos medios.

Corte meridional (v medio zonal):

\begin{figure}

{\centering \includegraphics[width=130mm, ]{fig/tesis/v-ncep-corte-1} 

}

\caption{Media zonal del viento meridional.{fig:v-ncep-corte}}\label{fig:v-ncep-corte}
\end{figure}

Cosas para ver:\\
Dipolo entre niveles bajos y altos que alterna entre invierno y verano
(parte convergente en superficie y divergente en altura de la ITCZ que
se mueve hacia el hemisferio de verano). En altas latitudes, en
superficie hay máximos de viento del sur debido a los vientos
catabáticos de la antártida.

\begin{figure}

{\centering \includegraphics[width=227mm,angle=90]{fig/tesis/v-ncep-1} 

}

\caption{Viento meridional medio.{fig:v-ncep}}\label{fig:v-ncep}
\end{figure}

Cosas para ver:\\
No mucha actividad salvo por la onda 1 en niveles altos (consistente con
la onda 1 de geopotenical).

La anomalía zonal es casi igual. No poner gráfico pero aclarar que es no
hay casi diferencia ya que la media zonal es casi cero en casi todo el
dominio.

\section{Gradiente meridional de vorticidad
absoluta}\label{gradiente-meridional-de-vorticidad-absoluta}

\begin{figure}

{\centering \includegraphics[width=227mm,angle=90]{fig/tesis/etady-ncep-1} 

}

\caption{Gradiente meridional de vorticidad absoluta.{fig:etady-ncep}}\label{fig:etady-ncep}
\end{figure}

\section{Número de onda estacionaria}\label{numero-de-onda-estacionaria}

\begin{figure}
\includegraphics[ ]{fig/tesis/ks-ncep-1} \caption{Número de onda estacionario en 200hPa.{fig:ks-ncep}}\label{fig:ks-ncep}
\end{figure}

\begin{figure}

{\centering \includegraphics[width=130mm, ]{fig/tesis/ks-ncep-corte-1} 

}

\caption{Número de onda estacionario medio por círculo de latitud.{fig:ks-ncep-corte}}\label{fig:ks-ncep-corte}
\end{figure}

Cosas para ver:\\
Máximos asociado con los flancos del jet. Zona ``prohibida'' en 200 y
300 hPa.\\
La onda 3 es estacionaria en -60°.

\section{Función corriente}\label{funcion-corriente}

\begin{figure}

{\centering \includegraphics[width=227mm,angle=90]{fig/tesis/psi-ncep-1} 

}

\caption{Función corriente x 1099{fig:psi-ncep}}\label{fig:psi-ncep}
\end{figure}

\section{Ondas Quasiestacionarias}\label{ondas-quasiestacionarias}

\begin{itemize}
\tightlist
\item
  Fourier
\end{itemize}

\begin{figure}

{\centering \includegraphics[width=227mm,angle=90]{fig/tesis/r2-ncep-1} 

}

\caption{$R^2$ de Fourier.{fig:r2-ncep}}\label{fig:r2-ncep}
\end{figure}

Cosas para ver:\\
Estructura. Zona donde onda 3 explica más que la onda 1 (zona marcada en
negro)

\begin{figure}

{\centering \includegraphics[width=227mm,angle=90]{fig/tesis/ampl-ncep-1} 

}

\caption{Ampllitud de Fourier.{fig:ampl-ncep}}\label{fig:ampl-ncep}
\end{figure}

Cosas para ver:\\
Onda 1 y 2 principalmente en estratósfera pero baja, salvo en verano.
Onda 3 y 4 más de atmósfera media/alta. Región recuadrada: máximo de
amplitud de QS 3 y donde su R2 es mayor que la de QS 1.

\chapter{Onda 3}\label{onda-3}

\section{Características típicas}\label{caracteristicas-tipicas}

\begin{figure*}
\includegraphics[width=150mm, ]{fig/tesis/qs3-ncep-1} \caption{Media de reconstrucción de onda 3.{fig:qs3-ncep}}\label{fig:qs3-ncep}
\end{figure*}

Cosas para ver:\\
Solo en 300 porque la estructura es barotrópica (no se gana mucho
mirando varios niveles). Localización de los centros de altas y bajas.
Corrimiento de fase verano/invierno. Aparente ciclo anual con mínimo en
primavera, que luego se ve que no es tan así, parece mínimo porque la
fase varía mucho y el promedio se desdibuja mucho.

Esto es el promedio de las ondas 3, pero es idéntico a la onda 3 del
promedio.

\begin{figure*}
\includegraphics[width=150mm, ]{fig/tesis/qs3-ncep-corte-1} \caption{Corte{fig:qs3-ncep-corte}}\label{fig:qs3-ncep-corte}
\end{figure*}

Cosas para ver:\\
Estrucutra vertical barotrópica equivalente. Ciclo anual en la extensión
vertical (se ve también en los cortes de amplitud). Aunque notar que en
este corte la extensión en primavera parece la menor, pero de nuevo es
por la variabilidad en la fase, ya que en el corte de amplitud se ve que
la amplitud es mayor en altura incluso que en otoño.

\begin{figure*}
\includegraphics[width=150mm, ]{fig/tesis/qs3sd-ncep-1} \caption{Desvío estándar de la reconstrucción de QS3.{fig:qs3sd-ncep}}\label{fig:qs3sd-ncep}
\end{figure*}

No tengo idea de cómo interpretar esto\ldots{}

\begin{itemize}
\tightlist
\item
  Wavelets
\end{itemize}

La amplitud media obtenida mediante wavelets es virtualmente idéntica

\begin{figure}

{\centering \includegraphics[width=130mm, ]{fig/tesis/wavelet-fourier-ncep-1} 

}

\caption{Amplitud de wavelets (sombreados) y de fourier (contornos){fig:wavelet-fourier-ncep}}\label{fig:wavelet-fourier-ncep}
\end{figure}

\begin{figure}

{\centering \includegraphics[width=227mm,angle=90]{fig/tesis/waveletz-ncep-1} 

}

\caption{Campo medio de la amplitud de la onda 3 según wavelets (contornos) y su anomalía zonal (sombreado) en 300hPa.{fig:waveletz-ncep}}\label{fig:waveletz-ncep}
\end{figure}

Cosas para ver:\\
* La amplitud de la onda 3 es bastante zonal, pero tiene asimetrías
consistentes. En \textasciitilde{}60°S, la máxima amplitude se da en el
hemisferio oeste, mientras que en \textasciitilde{}50°S se da en el
hemisferio este.

\begin{figure}

{\centering \includegraphics[width=227mm,angle=90]{fig/tesis/wavelet-ncep-corte-1} 

}

\caption{Corte zonal en -60° de la amplitud media de la onda 3 según wavelets (contornos) y su anomalía zonal (sombreado).{fig:wavelet-ncep-corte}}\label{fig:wavelet-ncep-corte}
\end{figure}

Cosas para ver:\\
* Si bien el máximo medio de la amplitud se da en
\textasciitilde{}300hPa (Fig \autoref{fig:wavelet_fourier_comp}) igual
que en fourier, el análisis por longitud muestra algo un poco más
complejo. En verano y otoño, la máxima amplitud sigue en 300hPa, pero
ésta asciente a entre 100 y 50hPa en invierno y primavera al rederor de
120°O. Además, durante estas estaciones hay indicación de que el máximo
cambia de latitud con la altura.

Venajas y desventajas. Justificaicón de decisión.

\section{Antecedentes}\label{antecedentes}

Breve comentario sobre los índices usados en otros lados. Discutir
ventajas y debilidades.

\begin{itemize}
\tightlist
\item
  Amplitud
\item
  Fase (impacto en SA)
\end{itemize}

De todo eso, motiva decisión del índice.

\begin{itemize}
\tightlist
\item
  Niveles elegidos
\item
  Promedio vs.~máximo
\item
  Composiciones de campos y flujos.
\item
  Decisión del índice.
\end{itemize}

Quiero hacer el íncide a partir de la actividad de la onda 3 tomando la
región del máximo (latitud entre -65 y -40, y entre 700 y 100 hPa).
Variables posibles: amplitud media, amplitud máxima, r2, correlación
entre campo teórico y observado.

\section{Amplitud}\label{amplitud}

\subsection{Máximo o media.}\label{maximo-o-media.}

\begin{figure*}
\includegraphics[width=150mm, ]{fig/tesis/ampl-max-mean-1} \caption{Distribució de amplitud para 12 fechas. En rojo la amplitud máxima, en azul la amplitud media.{fig:ampl-max-mean}}\label{fig:ampl-max-mean}
\end{figure*}

\begin{figure*}
\includegraphics[width=150mm, ]{fig/tesis/ampl-max-mean-corte-1} \caption{Corte vertical de amplitud{fig:ampl-max-mean-corte}}\label{fig:ampl-max-mean-corte}
\end{figure*}

Cosas para ver:\\
Casos donde el máximo es mayor pero la media, menor. (1985-01-01 vs
1988-07-01 o ). 1987-11-01 vs 2008-01-01 muestra el caso: igual amplitud
máxima pero en 2008 está más ``concentrada''. Casos donde la actividad
está ligeramente fuera de la caja (2000-09-01).

\begin{figure*}
\includegraphics[width=150mm, ]{fig/tesis/ghz-ncep-select-1} \caption{Anomalía zonal geopotencial en 300hPa para fechas seleccionadas.{fig:ghz-ncep-select}}\label{fig:ghz-ncep-select}
\end{figure*}

Cosas para ver:\\
Analizar nivel de similitud entre los campos y similitud entre la
estructura de la ampltiud.

\begin{figure}
\includegraphics[ ]{fig/tesis/cor-mean-max-1} \caption{Correlación entre amplitud máxima y media.{fig:cor-mean-max}}\label{fig:cor-mean-max}
\end{figure}

Cosas para ver:\\
Relación lineal entre ambas. Ergo, da más o menso igual usar cualquiera.

Luego\ldots{} concluir que vamos a usar la media.

\begin{figure*}
\subfloat[Serie temporal\label{fig:ampl-ts1}]{\includegraphics[width=150mm, ]{fig/tesis/ampl-ts-1} }\newline\subfloat[Ciclo anual\label{fig:ampl-ts2}]{\includegraphics[width=150mm, ]{fig/tesis/ampl-ts-2} }\caption{Amplitud media - fig:ampl-ts}\label{fig:ampl-ts}
\end{figure*}

\section{Fase}\label{fase}

Además de eso, tengo la fase. Puedo tomar la fase media en la región o
la fase correspondiente a donde está el máximo de la amplitud (lo que
equivale al centro del anticiclón) o la moda de la fase.

(más explicación\ldots{})

¿Cuál usar?

\begin{figure}

{\centering \includegraphics[width=130mm, ]{fig/tesis/lon-phase-1} 

}

\caption{Fase promedio vs Fase del máximo{fig:lon-phase}}\label{fig:lon-phase}
\end{figure}

Cosas para ver:\\
Se ve que hay muchos casos donde coinciden bastante bien, pero otros que
se alejan mucho. La razón es que al tomar el promedio, puede quedar
cualquier cosa si la estructura cambia mucho con la latitud. Por
ejemplo, los casos marcados en rojo.

\begin{figure}
\includegraphics[ ]{fig/tesis/fase-mal-1} \caption{Campo de onda 3 reconstruido para junio de 2013. El punto rojo es la fase en la latitud de amplitud máxima; el punto negro, la fase promedio; el punto azul, la moda de la fase. El rectángulo la región donde puede encontrarse la fase. Los límites meridionales definidos por la región donde se calcula el índice y los zonales por 360/3.{fig:fase-mal}}\label{fig:fase-mal}
\end{figure}

En el primer caso el máximo de geopotencial se encuentra cercano a 0° lo
que hace que la fase oscile entre valores cercanos a 0° y a 120° en
distintas latitudes a pesar de que representan el mismo centro de
máxima. El promedio, por lo tanto, queda en el centro y termina estando
en una región donde el campo de geopotencial de onda 3 es nulo. Si bien
la fase de la amplitud máxima y la moda de la fase parecen ser muy
distintas, la naturaleza cíclica de la fase implica que representan
aproximadamente el mismo punto.

En tercer caso la estructura de onda 3 es de dos centros a distintas
latitudes. La fase promedio de ambas (punto negro) representa el punto
medio y, por lo tanto, está más cerca del mínimo de geopotencial que del
máximo. En este caso la moda de la fase es una mejor representación del
campo de onda 3 que la fase del máximo.

En el caso del medio, todo anda bien y los tres puntos coinciden.

Conclusión: vamos a usar la moda de la fase.

Ciclo anual de la fase.

\begin{figure}
\includegraphics[ ]{fig/tesis/fase-boxplot-1} \caption{Ciclo anual de la fase (20 mayores amplitudes para cada mes){fig:fase-boxplot}}\label{fig:fase-boxplot}
\end{figure}

Cosas para ver:\\
Se ve el ciclo anual que ya se veía en una figura anterior. Pero hay
mucha variabilidad.\\
Recordar que este es aproximadamente el centro del máximo de
geopotencial y que, por construcción, hay un centro de mínima a 60°. Eso
implica que en los casos donde hay puntos cerca de -120° o 0°, hay el
centro de mínima está en \textasciitilde{}-60°.\\
También notar que, como se trata en realidad de una variable cíclica,
los puntos extremos representan una situación similar.\\
¿Hay mucha diferencia entre el boxplot con todos los datos y el de sólo
los 20 más intensos? Yo no veo mucha y me parece que excluir a los 10
más débiles invita más preguntas que las que responde.

\begin{figure}

{\centering \includegraphics[width=227mm,angle=90]{fig/tesis/centros-10max-1} 

}

\caption{Centros de máxima (puntos rojos) y mínima (puntos azules) para los 10 años con mayor amplitud de la onda 3, para cada mes.{fig:centros-10max}}\label{fig:centros-10max}
\end{figure}

Cosas para ver:\\
La mayoría del tiempo estamos afectados por centro de máxima, pero hay
casos intensos donde hay mínima. Especialmente en junio y julio.

\section{Estaciones}\label{estaciones}

¿Cómo establecer una ``estacionalidad''? Algún criterio para agrupar
meses. Claramente las estaciones climatológicas de siempre no sirven.

Una forma es, a partir del índice, agrupar los meses según su amplitud y
fase media:

\begin{figure}
\includegraphics[ ]{fig/tesis/ampl-vs-fase-1} \caption{Amplitud y fase media para cada mes.{fig:ampl-vs-fase}}\label{fig:ampl-vs-fase}
\end{figure}

Otro ejemplo podría usar SVD:

\begin{figure}
\subfloat[Varianza explicada de cada componente\label{fig:eof1}]{\includegraphics[ ]{fig/tesis/eof-1} }\newline\subfloat[Campo asociado a las primeras 2 componentes\label{fig:eof2}]{\includegraphics[ ]{fig/tesis/eof-2} }\newline\subfloat[Factor de peso para cada mes\label{fig:eof3}]{\includegraphics[ ]{fig/tesis/eof-3} }\caption{Análisis SVD de QS3.probablemente basura{fig:eof}}\label{fig:eof}
\end{figure}

\begin{figure}
\includegraphics[ ]{fig/tesis/eof1-eof2-1} \caption{'Magnitud' y 'Fase' de cada mes a partir de las componentes principales{fig:eof1-eof2}}\label{fig:eof1-eof2}
\end{figure}

Cosas para ver:\\
Los campos son descriptos bien por sólo 2 componentes principales de
igual magnitud que esencialmente describen el corrimiento de fase de la
onda. La ``magnitud'' del vector de componentes mapea bien a la amplitud
de fourier y la diferencia mapea bien a la fasae. No hay nada nuevo por
acá.

Grupos: * Enero + Febrero + Marzo es un grupo bastante seguro. Se ven
juntos en el diagrama amplitud-fase, tienne alta correlación entre ellos
y en componentes principales tienen PC1 significativamente mayor que
PC2. * Junio + Julio parecen ir por su cuenta. Tienen mucha correlación
entre ellos y ambos tienen PC2 mayor que PC1 aunque su fase difiere
bastante.

Los otros meses quedan medio mezclados, pero pueden agruparse como meses
de ``transición''.

\section{Persistencia (ver dónde va)}\label{persistencia-ver-donde-va}

\begin{figure}
\includegraphics[ ]{fig/tesis/lag-cor-1} \caption{Correlación lageada para cada mes con los 12 sigientes.{fig:lag-cor}}\label{fig:lag-cor}
\end{figure}

\section{R2}\label{r2}

Se puede estimar de dos maneras distintas: a partir del ajuste de
fourier para cada nivel y latitud (figura blabla) y haciendo un
promedio, o reconstruyendo el campo tridimensional de la onda 3 y
haciendo la correlación (global) con el campo tridimensional observado.
Esta segunda forma da casi siempre un valor menor.

\begin{figure}
\includegraphics[ ]{fig/tesis/r2-cor2-1} \caption{Relación entre R2 medio y R2 reconstruido.{fig:r2-cor2}}\label{fig:r2-cor2}
\end{figure}

Para ver:\\
La relación ya no es lineal y hay bastante más scatter. Ergo, hay
diferencia en la información.\\
Dos regímenes: Cuando el R\^{}2 es bajo, la relación es ``menor'' que
1:1 y hay algunos casos donde el R2 reconstruido es mayor que el r2
medio. Para R2 más grandes, la pendiente es 1. Modelar la
relación\ldots{} ¿Cómo se interpreta?

\begin{figure*}
\includegraphics[width=150mm, ]{fig/tesis/r2-corte-1} \caption{R2 medio{fig:r2-corte}}\label{fig:r2-corte}
\end{figure*}

Cosas para ver:\\
Comparando con las figuras anteriores, casos donde la amplitud es grande
pero el R2 no tanto.

\begin{figure*}
\subfloat[Serie temporal\label{fig:r2-ts1}]{\includegraphics[width=150mm, ]{fig/tesis/r2-ts-1} }\newline\subfloat[Ciclo anual\label{fig:r2-ts2}]{\includegraphics[width=150mm, ]{fig/tesis/r2-ts-2} }\caption{R2 medio - fig:r2-ts}\label{fig:r2-ts}
\end{figure*}

Cosas para ver:\\
El ciclo anual no es para nada tan claro. Varios outliers. La
correlación reconstruida (azul) no tiene casi ciclo anual.

Una ventaja de la correlación entre el campo real y el reconstruido es
que puede hacerse para cada punto y analizar la variación espacial de la
misma.

\begin{figure}
\includegraphics[ ]{fig/tesis/cor-campo-1} \caption{Correlación cuadrada media para estaciones según onda3.{fig:cor-campo}}\label{fig:cor-campo}
\end{figure}

Cosas para ver:\\
Esto es, para cada punto de grilla, la correlación entre el campo
observado y el reconstruido en todos los meses y años. Además de la
dependencia latitudinal de la importancia de la onda 3 (que se puede ver
en los cortes anteriores), se ve la dependencia zonal. La onda 3 es más
importante en el Pacífico sur que en el Atlántico o el Índico. Además,
se ve un patrón de altas correlaciones que asemejan a un tren de
ondas.\\
¿Confirma? lo que se ve en en análisis de wavelets.

Conclusión: no voy a usar el r2 a partir del campo reconstruido.

\section{Regresiones}\label{regresiones}

\subsection{Geopotencial}\label{geopotencial}

\begin{figure}

{\centering \includegraphics[width=150mm, ]{fig/tesis/regr-gh-ncep-1} 

}

\caption{Regresión sobre amplitud.{fig:regr-gh-ncep}}\label{fig:regr-gh-ncep}
\end{figure}

Cosas para ver:\\
* Además los patrones de onda 3, en julio y diciembre aparece un patrón
de SAM positivo y negativo respectivamente.

\begin{figure}
\includegraphics[ ]{fig/tesis/regr-gh-polar-1} \caption{Igual que figura  XX, pero en proyección polar para julio y septiembre.{fig:regr-gh-polar}}\label{fig:regr-gh-polar}
\end{figure}

\begin{figure}
\includegraphics[ ]{fig/tesis/sam-ampl-1} \caption{Relación entre amplitud media de la onda 3 y el SAM{fig:sam-ampl}}\label{fig:sam-ampl}
\end{figure}

\subsection{Función Corriente}\label{funcion-corriente-1}

\begin{figure}

{\centering \includegraphics[width=150mm, ]{fig/tesis/regr-psi-ncep-1} 

}

\caption{Regresión de Psi con la amplitud.{fig:regr-psi-ncep}}\label{fig:regr-psi-ncep}
\end{figure}

\section{Composición de campos.}\label{composicion-de-campos.}

Descripción de la seleccion.

\begin{figure}

{\centering \includegraphics[width=227mm,angle=90]{fig/tesis/seleccion-tabla-1} 

}

\caption{Tabla de selección{fig:seleccion-tabla}}\label{fig:seleccion-tabla}
\end{figure}

Cosas para ver:\\
Años con coincidencia, años sin coindicentica. Meses donde la fase
coincide (julio) vs meses donde no coindice (septiembre). También, años
donde hay seguidilla de meses seleccionados (1999). Aunque posiblemente
sea casualidad (no hay mucha persistencia mes a mes.)

Pequeña digresión: Efecto de la fase.\\
La climatología de la fase se va a discutir más adelante, pero\ldots{}
discutir el efecto de promediar campos con similar amplitud pero fase
distinta. Del gráfico, septiembre tiene 1997 y 2003 con fase a 180°, lo
que significa que va a a haber cancelación parcial. Enero, por el
contrario, no tiene ningún año en contrafase, aunque sí algunos a 90°,
que desdibujan el patrón.

\begin{figure}

{\centering \includegraphics[width=130mm, ]{fig/tesis/interaccion-tabla-1} 

}

\caption{Tabla de interacción{fig:interaccion-tabla}}\label{fig:interaccion-tabla}
\end{figure}

Cosas para ver:\\
Ambos criterios coinciden en casi todos los años seleccionados, así que
no hay mucha diferencia. En efecto, las composiciones son casi iguales
(no se muestra). Voy a usar la amplitud.

\begin{figure}

{\centering \includegraphics[width=227mm,angle=90]{fig/tesis/gh-comp-1} 

}

\caption{Composición de campos{fig:gh-comp}}\label{fig:gh-comp}
\end{figure}

\begin{figure}

{\centering \includegraphics[width=150mm, ]{fig/tesis/gh-qs3-select-ene-1} 

}

\caption{Campos para los 10 eneros seleccionados.{fig:gh-qs3-select-ene}}\label{fig:gh-qs3-select-ene}
\end{figure}

\begin{figure}

{\centering \includegraphics[width=150mm, ]{fig/tesis/gh-qs3-select-sep-1} 

}

\caption{Campos para los 10 septiembres seleccionados.{fig:gh-qs3-select-sep}}\label{fig:gh-qs3-select-sep}
\end{figure}

Estos gráficos me parecen importantes para ver lo que hay ``adentro'' de
la composición, pero no sé bien qué decir sobre ellos. Supongo que lo
principal es que hay años donde la onda

\section{Fuentes de variabilidad
interna}\label{fuentes-de-variabilidad-interna}

(Discusión escrita más de papers), Pero nos concentramos en la fuente
externa.

\section{Fuentes externas}\label{fuentes-externas}

\subsection{SST}\label{sst}

\begin{figure}

{\centering \includegraphics[width=227mm,angle=90]{fig/tesis/regr-sst-ncep-1} 

}

\caption{Regresión de SST con la amplitud de la onda 3{fig:regr-sst-ncep}}\label{fig:regr-sst-ncep}
\end{figure}

Campos de correlación con SST y OLR, principalmente ¿Discusión de otros
forzantes?

\chapter{Experimentos}\label{experimentos}

\section{Validación SPEEDY}\label{validacion-speedy}

Validación de la corrida control

\begin{itemize}
\tightlist
\item
  Comparación campos medios.
\end{itemize}

Acá un problemita es que en speedy no tengo el nivel de 50hPa, sólo
tengo 925, 850, 700, 500, 300, 200, 100, 30. Podría usar 30, pero eso es
la tapa del modelo\ldots{}

\begin{figure}
\includegraphics[ ]{fig/tesis/cor-sp-nc-1} \caption{Correlación lineal entre campos de SPEEDY y NCEP.{fig:cor-sp-nc}}\label{fig:cor-sp-nc}
\end{figure}

\begin{itemize}
\item
  Altura geopotencial (gh) Para el caso del campo total, la correlación
  del campo es buena (\textgreater{}0.8)en casi todos los niveles y
  meses, excepto en 30hPa durante verano donde los campos ¡está
  anticorrelacionados! La parte asimétrica zonal muestra valores
  menores, indicando que gran parte de la correlación del campo total se
  debe a la capacidad del modelo de reproducir el gradiente meridional.
  Sin embargo, se siguen obteniendo correlaciones \textgreater{}0.6 en
  casi todos los niveles y estaciones. Se observa un mínimo relativo en
  500hPa donde en se tienen correlaciones menores durante casi todo el
  año y uno en niveles altos centrado en invierno y primavera.
\item
  Viento zonal (U) Las correlaciones con el campo total son
  \textgreater{}=0.8 en todo el año y todos los niveles, sin embargo, la
  parte asimétrica muestra correlaciones mucho más baja con un máximo de
  \textasciitilde{}0.6 en 925hPa. Esto indica que el modelo resuelve
  correctamente la estructura media del Jet, pero no sus variaciones
  zonales.
\item
  Viento meridional (V) Los campos de correlación son prácticamente
  idénticos entre parte total y parte asimétrica. Ésta muestra un patrón
  de bajas correlaciones en general.
\item
  Temperatura (T) La correlación con el campo total muestra una
  estructura similar que la altura geopotencial, con una excelente
  correlación en todos los meses para niveles mayores a 200hPa, pero
  anticorrelacionado en niveles altos en todos los meses salvo en
  invierno. La parte asimétrica muestra correlaciones bajas en todos los
  niveles salvo en 925hPa.
\item
  Gradiente meridional de vorticidad absoluta. Tiene correlación
  moderada con un mínimo relativo en 200hPa en verano que en las otras
  estacione se convierte en un máximo.
\end{itemize}

\subsection{Altura Geopotencial}\label{altura-geopotencial-1}

Anomalía

\begin{figure}

{\centering \includegraphics[width=227mm,angle=90]{fig/tesis/ghz-sp-nc-1} 

}

\caption{Anomalía zonal de altura geopotencial (speedy sombreado, ncep contornos){fig:ghz-sp-nc}}\label{fig:ghz-sp-nc}
\end{figure}

\begin{figure}

{\centering \includegraphics[width=227mm,angle=90]{fig/tesis/ghz-dif-sp-nc-1} 

}

\caption{Diferencia entre speedy y ncep{fig:ghz-dif-sp-nc}}\label{fig:ghz-dif-sp-nc}
\end{figure}

Veredicto:\\
* Agarra bien la anomalía zonal aunque con magnitud menor.

\begin{figure}

{\centering \includegraphics[width=130mm, ]{fig/tesis/ghz-sp-nc-corte60-1} 

}

\caption{Corte zonal de anomalía de geopotencial en -60° (speedy sombreado, ncep contornos).{fig:ghz-sp-nc-corte60}}\label{fig:ghz-sp-nc-corte60}
\end{figure}

El corte zonal evidencia que además de tener menor amplitud, la
estructura vertical de las anomalías es barotrópica equivalente en
Speedy, a diferencia de la estructura baroclínica de NCEP.

\subsection{Temperatura}\label{temperatura-1}

\begin{figure}
\includegraphics[ ]{fig/tesis/t-nc-sp-1} \caption{Temperatura{fig:t-nc-sp}}\label{fig:t-nc-sp}
\end{figure}

\begin{figure}
\includegraphics[ ]{fig/tesis/tz-sp-nc-1} \caption{T*{fig:tz-sp-nc}}\label{fig:tz-sp-nc}
\end{figure}

En 850 tiene una fuerte onda 1 en latitudes polares producida por la
topografía de la Antártida y representa bien las anomalías causadas por
los continentes, aunque en menor amplitud.\\
En 200, falla miserablemente. La onda 1 polar que se ve bien claro en
NCEP ni aparece en SPEEDY, mientras que en latitudes medias aparecen
ligeras anomalías que no se observan en las observaciones.

\subsection{Viento zonal}\label{viento-zonal-1}

\begin{figure}

{\centering \includegraphics[width=130mm, ]{fig/tesis/u-sp-nc-corte-1} 

}

\caption{Viento zonal medio (speedy contornos, ncep sombreado).{fig:u-sp-nc-corte}}\label{fig:u-sp-nc-corte}
\end{figure}

Cosas para ver:\\
* Speedy no logra desarrollar un jet polar por la falta de niveles
verticales en la estratósfera. Tampoco reproduce los estes
estratosféricos en latitudes bajas. Su jet subtropical es más intenso y
su máximo se da ligéramente en niveles más altos en NCEP.

Campo medio (me parece que no lo voy a poner, no agrega información que
no esté en el anterior y en el de geopotencial.)

\begin{figure}

{\centering \includegraphics[width=227mm,angle=90]{fig/tesis/u-sp-nc-1} 

}

\caption{Viento zonal (contornos ncep, sombreado speedy).{fig:u-sp-nc}}\label{fig:u-sp-nc}
\end{figure}

\begin{figure}
\includegraphics[ ]{fig/tesis/u-dif-sp-nc-1} \caption{Diferencia entre ncep y speedy en viento zonal{fig:u-dif-sp-nc}}\label{fig:u-dif-sp-nc}
\end{figure}

Cosas para ver:\\
Jet polar en invierno y primavera en niveles altos (\textless{} 100
hPa). Jest subtropical en niveles ``medios''.

\subsection{Gradiente meridional de vorticidad
absoluta}\label{gradiente-meridional-de-vorticidad-absoluta-1}

\begin{figure}

{\centering \includegraphics[width=227mm,angle=90]{fig/tesis/etady-sp-nc-1} 

}

\caption{Gradiente meridional de vorticidad absoluta (speedy).{fig:etady-sp-nc}}\label{fig:etady-sp-nc}
\end{figure}

Comparando con la figura \autoref{fig:eta-dy-ncep}, los gradientes son
menores y más zonales. Es significativo que la región ``prohibida'' en
invierno de niveles altos es menor en 200hPa y casi desaparece en
300hPa.

\subsection{Número de onda
estacionaria}\label{numero-de-onda-estacionaria-1}

\begin{figure}
\includegraphics[ ]{fig/tesis/ks-sp-1} \caption{Número de onda estacionario en 300hPa (speedy).{fig:ks-sp}}\label{fig:ks-sp}
\end{figure}

Comparando con \autoref{fig:k-steady-ncep} la mayor diferencia es la
desaparición de una región de propagación impedida en
\textasciitilde{}-40° en el Índico y el Pacífico en Otoño.

\begin{figure}

{\centering \includegraphics[width=130mm, ]{fig/tesis/ks-sp-nc-corte-1} 

}

\caption{Número de onda estacionario medio por círculo de latitud.{fig:ks-sp-nc-corte}}\label{fig:ks-sp-nc-corte}
\end{figure}

En promedio zonal, sin embargo, SPEEDY funciona bien.

\subsection{Función corriente}\label{funcion-corriente-2}

\begin{figure}

{\centering \includegraphics[width=227mm,angle=90]{fig/tesis/psi-sp-1} 

}

\caption{Función corriente x 1099{fig:psi-sp}}\label{fig:psi-sp}
\end{figure}

\subsection{Onda 3}\label{onda-3-1}

\begin{figure*}
\includegraphics[width=150mm, ]{fig/tesis/ampl-sp-nc-1} \caption{Amplitud de Fourier (speedy en sombreado, ncep en contornos).{fig:ampl-sp-nc}}\label{fig:ampl-sp-nc}
\end{figure*}

\begin{figure*}
\includegraphics[width=150mm, ]{fig/tesis/qs2-sp-nc-1} \caption{Media de reconstrucción de onda 3 (sombreado speedy, contornos ncep){fig:qs2-sp-nc}}\label{fig:qs2-sp-nc}
\end{figure*}

La onda 3 no está muy bien representada en el modelo. Aunque la
estructura vertical y la posición meridional está bien (salvo en otoño),
la amplitud es mucho menor. La fase, además, está corrida ligeramente en
verano, pero quedando en cuadratura en invierno y defasado 180 en
primavera.

\section{Comparación}\label{comparacion}

Comparación entre corridas \#\#\# Altura geopotencial

\begin{figure}

{\centering \includegraphics[width=227mm,angle=90]{fig/tesis/ghz-sp-runs-1} 

}

\caption{Anomalía zonal de altura geopotencial.{fig:ghz-sp-runs}}\label{fig:ghz-sp-runs}
\end{figure}

\begin{figure}
\includegraphics[ ]{fig/tesis/ghz-dif-sp-runs-1} \caption{Diferencia Control - corrida para Z*{fig:ghz-dif-sp-runs}}\label{fig:ghz-dif-sp-runs}
\end{figure}

\subsection{Temperatura}\label{temperatura-2}

ONo hay casi diferencia entre las corridas.

\begin{figure}
\includegraphics[ ]{fig/tesis/t-sp-runs-1} \caption{Temperatura media en 850hPa.{fig:t-sp-runs}}\label{fig:t-sp-runs}
\end{figure}

\begin{figure}
\includegraphics[ ]{fig/tesis/tz-sp-runs-1} \caption{Temperatura media en 850hPa.{fig:tz-sp-runs}}\label{fig:tz-sp-runs}
\end{figure}

\begin{figure}

{\centering \includegraphics[width=227mm,angle=90]{fig/tesis/tz-dif-sp-runs-1} 

}

\caption{Diferencia Control - corrida para T*{fig:tz-dif-sp-runs}}\label{fig:tz-dif-sp-runs}
\end{figure}

\subsection{Viento zonal}\label{viento-zonal-2}

\begin{figure}

{\centering \includegraphics[width=227mm,angle=90]{fig/tesis/uz-sp-runs-1} 

}

\caption{Viento zonal{fig:uz-sp-runs}}\label{fig:uz-sp-runs}
\end{figure}

\begin{figure}
\includegraphics[ ]{fig/tesis/u-dif-sp-runs-1} \caption{Diferencia control - corrida para U.{fig:u-dif-sp-runs}}\label{fig:u-dif-sp-runs}
\end{figure}

\subsection{Función corriente}\label{funcion-corriente-3}

\begin{figure}

{\centering \includegraphics[width=227mm,angle=90]{fig/tesis/psi-sp-runs-1} 

}

\caption{Anomalía zonal de función corriente y flujos de acción de onda.{fig:psi-sp-runs}}\label{fig:psi-sp-runs}
\end{figure}

\begin{figure}
\includegraphics[ ]{fig/tesis/psiz-dif-sp-runs-1} \caption{Diferencia en psi.z y flujos de acción de onda.{fig:psiz-dif-sp-runs}}\label{fig:psiz-dif-sp-runs}
\end{figure}

\subsection{Onda 3}\label{onda-3-2}

\begin{figure}
\includegraphics[ ]{fig/tesis/ampl-sp-runs-1} \caption{Amplitud media de la onda 3 para cada corrida.{fig:ampl-sp-runs}}\label{fig:ampl-sp-runs}
\end{figure}

\begin{figure}
\includegraphics[ ]{fig/tesis/ampl-dif-sp-runs-1} \caption{Diferencia de amplitud entre la corrida control y cada corrida.{fig:ampl-dif-sp-runs}}\label{fig:ampl-dif-sp-runs}
\end{figure}

\begin{figure}
\includegraphics[ ]{fig/tesis/index-sp-boxplot-1} \caption{Ciclo anual de amplitud de onda 3.{fig:index-sp-boxplot}}\label{fig:index-sp-boxplot}
\end{figure}

\section{Regresión}\label{regresion}

\begin{figure}
\includegraphics[ ]{fig/tesis/regr-psi-sp-runs-1} \caption{Regresión en función corriente.{fig:regr-psi-sp-runs}}\label{fig:regr-psi-sp-runs}
\end{figure}

\section{Cosas inesperadas\ldots{}}\label{cosas-inesperadas}

\begin{itemize}
\tightlist
\item
  ??
\item
  protif!
\end{itemize}

\chapter{Conclusiones}\label{conclusiones}

\chapter{Agradecimientos}\label{agradecimientos}

\chapter*{Referencias}\label{referencias}
\addcontentsline{toc}{chapter}{Referencias}

\hypertarget{refs}{}
\hypertarget{ref-Vera2004}{}
Vera, Carolina, Gabriel Silvestri, Vicente Barros, y Andrea Carril.
2004. «Differences in El Niño response over the Southern Hemisphere».
\emph{Journal of Climate} 17 (9): 1741-53.
doi:\href{https://doi.org/10.1175/1520-0442(2004)017\%3C1741:DIENRO\%3E2.0.CO;2}{10.1175/1520-0442(2004)017\textless{}1741:DIENRO\textgreater{}2.0.CO;2}.


\end{document}
