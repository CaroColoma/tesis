\documentclass[spanish,a4paper,12pt]{book}
\usepackage{lmodern}
\usepackage{amssymb,amsmath}
\usepackage{ifxetex,ifluatex}
\usepackage{fixltx2e} % provides \textsubscript
\ifnum 0\ifxetex 1\fi\ifluatex 1\fi=0 % if pdftex
  \usepackage[T1]{fontenc}
  \usepackage[utf8]{inputenc}
\else % if luatex or xelatex
  \ifxetex
    \usepackage{mathspec}
  \else
    \usepackage{fontspec}
  \fi
  \defaultfontfeatures{Ligatures=TeX,Scale=MatchLowercase}
\fi
% use upquote if available, for straight quotes in verbatim environments
\IfFileExists{upquote.sty}{\usepackage{upquote}}{}
% use microtype if available
\IfFileExists{microtype.sty}{%
\usepackage{microtype}
\UseMicrotypeSet[protrusion]{basicmath} % disable protrusion for tt fonts
}{}
\usepackage[inner = 3cm, outer = 2.5cm, top = 2.5cm, bottom = 2.5cm]{geometry}
\usepackage{hyperref}
\hypersetup{unicode=true,
            pdftitle={Estudio de los mecanismos físicos asociados con el patrón de onda 3 de la circulación atmosférica del Hemisferio Sur},
            pdfauthor={Elio Campitelli},
            pdfborder={0 0 0},
            breaklinks=true}
\urlstyle{same}  % don't use monospace font for urls
\ifnum 0\ifxetex 1\fi\ifluatex 1\fi=0 % if pdftex
  \usepackage[shorthands=off,main=spanish]{babel}
\else
  \usepackage{polyglossia}
  \setmainlanguage[]{spanish}
\fi
\usepackage{longtable,booktabs}
\usepackage{graphicx,grffile}
\makeatletter
\def\maxwidth{\ifdim\Gin@nat@width>\linewidth\linewidth\else\Gin@nat@width\fi}
\def\maxheight{\ifdim\Gin@nat@height>\textheight\textheight\else\Gin@nat@height\fi}
\makeatother
% Scale images if necessary, so that they will not overflow the page
% margins by default, and it is still possible to overwrite the defaults
% using explicit options in \includegraphics[width, height, ...]{}
\setkeys{Gin}{width=\maxwidth,height=\maxheight,keepaspectratio}
\IfFileExists{parskip.sty}{%
\usepackage{parskip}
}{% else
\setlength{\parindent}{0pt}
\setlength{\parskip}{6pt plus 2pt minus 1pt}
}
\setlength{\emergencystretch}{3em}  % prevent overfull lines
\providecommand{\tightlist}{%
  \setlength{\itemsep}{0pt}\setlength{\parskip}{0pt}}
\setcounter{secnumdepth}{5}
% Redefines (sub)paragraphs to behave more like sections
\ifx\paragraph\undefined\else
\let\oldparagraph\paragraph
\renewcommand{\paragraph}[1]{\oldparagraph{#1}\mbox{}}
\fi
\ifx\subparagraph\undefined\else
\let\oldsubparagraph\subparagraph
\renewcommand{\subparagraph}[1]{\oldsubparagraph{#1}\mbox{}}
\fi

%%% Use protect on footnotes to avoid problems with footnotes in titles
\let\rmarkdownfootnote\footnote%
\def\footnote{\protect\rmarkdownfootnote}

%%% Change title format to be more compact
\usepackage{titling}

% Create subtitle command for use in maketitle
\newcommand{\subtitle}[1]{
  \posttitle{
    \begin{center}\large#1\end{center}
    }
}

\setlength{\droptitle}{-2em}
  \title{Estudio de los mecanismos físicos asociados con el patrón de onda 3 de
la circulación atmosférica del Hemisferio Sur}
  \pretitle{\vspace{\droptitle}\centering\huge}
  \posttitle{\par}
  \author{Elio Campitelli}
  \preauthor{\centering\large\emph}
  \postauthor{\par}
  \date{}
  \predate{}\postdate{}

\usepackage{setspace}
\setstretch{1.5}
\usepackage{subfig}
\usepackage{hyperref}
\usepackage{marginnote}
\usepackage[nomarkers,figuresonly]{endfloat}
\usepackage{pdflscape}
\DeclareDelayedFloatFlavor{landscape}{figure}
\usepackage[spanish]{todonotes}
\usepackage{wrapfig}
\usepackage{helvet}
\renewcommand{\familydefault}{\sfdefault}
    
\makeatletter
\@ifpackageloaded{endfloat}
{
 \efloat@condopen{fff}
 \pretocmd\@chapter{\immediate\write\efloat@postfff{\noexpand\stepcounter{chapter}}}{}{\fail}
 \AtBeginFigures{\setcounter{chapter}{0}}
}{}
\makeatother
\usepackage{helvet}
\renewcommand{\familydefault}{\sfdefault}

\begin{document}
\maketitle

Resumen.

\chapter*{Agradecimientos}

A las médicas y enfermeras del Sanatorio Güemes que me cuidaron durante
mi internación.

\newpage

\begin{center}\begin{minipage}{\dimexpr\paperwidth-7cm}
\chapter*{Resumen}
Este es el abstract balbalbalabla
\end{minipage}
\end{center}

\setcounter{tocdepth}{3} \tableofcontents

\listoffigures
\newpage

\chapter{Introducción}\label{introduccion}

La variabilidad del clima en el sur de Sudamérica (SSA) es afectada
tanto por la variabilidad en regiones de latitudes medias y polares
como, remotamente, por las condiciones del clima en regiones tropicales
y extratropicales (ej. Vera et~al.,
\protect\hyperlink{ref-Vera2006}{2006}). Dado que éstas últimas exhiben
altos niveles de predictibilidad (en escalas estacionales asociados con
el fenómeno del Niño-Oscilación del Sur (ENSO) y en escalas
intraestacionales con la oscilación de Madden-Julian (MJO)), su
influencia ha sido y es más extensamente estudiada. Sin embargo,
estudios recientes de la variabilidad climática en las regiones
subpolares y polares del Hemisferio Sur (HS) destacan la actividad del
Modo Anular del Sur (SAM, por sus siglas en inglés) como una fuente de
variabilidad climática de significativa influencia sobre los continentes
del HS y en particular del SSA (ej. Silvestri y Vera
(\protect\hyperlink{ref-Silvestri2009}{2009}) en escalas interanuales y
Alvarez et~al. (\protect\hyperlink{ref-Alvarez2014}{2014}) en escalas
intraestacionales).

La distribución espacial del SAM en su fase positiva se caracteriza por
una anomalía negativa de presión o altura geopotencial sobre el
continente antártico y anomalías de signo opuesto en latitudes medias.
Esta estructura zonalmente simétrica generalmente está asociada con un
patrón de onda planetario de números de onda zonal entre 3 y 4. La
alternancia del signo entre las anomalías polares y extratropicales del
SAM se asocia con intensificaciones y debilitamiento de los vientos
Oestes que caracterizan las latitudes medias del HS. Si bien hay
evidencia de que los flujos de cantidad de movimiento por las ondas
planetarias de baja frecuencia son importantes en el proceso de
debilitamiento de los Oestes (Lorenz y Hartmann,
\protect\hyperlink{ref-Lorenz2001}{2001}; Simpson et~al.,
\protect\hyperlink{ref-Simpson2013}{2013}) las causas por las cuales
estas variaciones se asocian con el desarrollo de la QS3 aún no se
conocen con certeza.

Las ondas planetarias de número de onda 1 y 3 (QS1 y QS3) son las
principales asimetrías zonales presentes en el flujo medio del HS (Loon
y Jenne, Roy, \protect\hyperlink{ref-Loon1972}{1972}; Trenberth,
\protect\hyperlink{ref-Trenberth1980a}{1980}). Estas ondas zonales
tienden a ser cuasi-estacionarias y exhiben importantes variabilidades
temporales (Loon y Jenne, Roy, \protect\hyperlink{ref-Loon1972}{1972}).
En particular, se ha documentado que la QS3 presenta una estructura
barotrópica con variabilidad en escalas diarias-semanales (Kidson,
\protect\hyperlink{ref-Kidson1988}{1988}), estacionales (Mo y White,
\protect\hyperlink{ref-Mo1985}{1985}) y más largas (Karoly,
\protect\hyperlink{ref-Karoly1989}{1989}). Mo y White
(\protect\hyperlink{ref-Mo1985}{1985}) mostraron que existen ubicaciones
preferenciales para los centros de acción de la QS3. Por la antigüedad
de muchos de estos trabajos, sus climatologías tienen limitaciones
inherentes a la poca calidad de datos disponible en el HS previa a la
era satelital.

Trenberth y Mo (\protect\hyperlink{ref-Trenberth1985}{1985}) mostraron
una recurrencia importante en la ocurrencia de anticiclones de bloqueo
simultáneos en diferentes regiones del HS (sur de Sudamérica, sur de
Nueva Zelandia y porción central del Océano Indico) favorecida por el
establecimiento de un patrón de QS3, aunque dejaron abierta la
posibilidad de que éste se trate de un tren de ondas localizado en vez
de una verdadera onda zonal. Desde ese momento hasta la actualidad
diferentes estudios se concentraron en entender la influencia de los
bloqueos sobre el clima de Sudamérica y su relación con la QS3 (ej. Rao
et~al., \protect\hyperlink{ref-Rao2004}{2004}) o de su efecto en el
hielo marítimo (ej. {\textbf{???}}) pero, como se mencionó
anteriormente, muy poca atención ha recibido el estudio de las causas
que dan lugar al establecimiento de este patrón en un primer lugar.
Quintanar y Mechoso
(\protect\hyperlink{ref-Quintanar1995}{1995}\protect\hyperlink{ref-Quintanar1995}{b})
realizaron experimentos de sensibilidad similares a los propuestos en
este trabajo pero centrados en la QS1. Encontraron que las condiciones
térmico-orográficas sobre la Antártida no eran suficientes para explicar
la QS1 de latitudes subpolares, por lo que concluyeron que los forzantes
remotos deben jugar un papel importante. Wang et~al.
(\protect\hyperlink{ref-Wang2013}{2013}) encontraron que la destrucción
y recuperación de la capa de ozono está asociada a un aumento y
disminución de la actividad de onda plantaria respectivamente, pero su
análisis no separa entre distintos números de onda. Cabe destacar que
Hobbs y Raphael (\protect\hyperlink{ref-Hobbs2010}{2010}) pusieron en
duda la utilidad de analizar la circulación del hemisferio sur en ondas
planetarias zonalmente simétricas y propusieron que ésta está mejor
caracterizada por un par de anticilones al sur de Nueva Zelanda y al sur
de Sudamérica.

Cai et~al. (\protect\hyperlink{ref-Cai1999}{1999}) evaluaron la
habilidad del modelo CSIRO para representar la QS3 y encontraron que
proporcionaba una representación adecuada de la misma. Raphael
(\protect\hyperlink{ref-Raphael1998}{1998}) examinó la QS3 en el modelo
CCM del NCAR (versiones 1 y 3) y encontró diferencias en cómo cada
versión simulaba la QS3, siendo los resultados sensibles al grado de
representación del modelo tanto de la interacción mar-atmósfera como de
las condiciones de hielo marino en las zonas polares. Este es un
resultado importante ya que a pesar de que la QS3 puede desarrollarse
solamente por la dinámica interna de la atmósfera (ej. Simpson et~al.,
\protect\hyperlink{ref-Simpson2013}{2013}), estos trabajos muestran que
los modelos pueden reproducirla y que existen evidencias de que las
condiciones superficiales pueden influenciar su actividad. Asimismo,
Raphael (\protect\hyperlink{ref-Raphael2003}{2003}) encontró importantes
variaciones interanuales experimentadas por la QS3 entre 1958 y 1996,
aparentemente relacionadas con variaciones en la frecuencia del ENSO,
que también influenciaron las asimetrías del SAM (Fogt et~al.,
\protect\hyperlink{ref-Fogt2012}{2012}).

En consecuencia el objetivo general de este plan es entender los
mecanismos que explican el desarrollo de la QS3 en la circulación del
HS. Los objetivos particulares son

\begin{itemize}
\tightlist
\item
  caracterizar la climatología de la QS3,
\item
  explorar la influencia de las condiciones oceánicas superficiales
  tanto en los trópicos como en los extratrópicos sobre la actividad de
  la QS3,
\item
  explorar la sensibilidad de la QS3 en general a las condiciones
  superficie.
\end{itemize}

\chapter{Datos y Metodologías}\label{datos-y-metodologias}

En este capítulo se describen los datos utilizados en la presente
investigación y las metodologías que se aplicaron.

\section{Datos y modelo}\label{datos-y-modelo}

En los capítulos~\ref{climatologia-observada} y~\ref{onda-3} se analizan
datos mensuales de altura geopotencial, temperatura, viento zonal,
viento meridional y función corriente provenientes del Reanálisis
NCEP/NCAR (Kalnay et~al., \protect\hyperlink{ref-Kalnay1996}{1996}) (de
aquí en adelante NCEP) entre enero de 1985 y diciembre de 2015. Los
mismos poseen una resolución de 2.5° en longitud y latitud y 17 niveles
verticales entre 1000hPa y 10hPa. A partir de los datos mensuales se
calcularon las medias estacionales definidas a partir de las estaciones
climatológicas del hemisferio sur.

Se utilizó el modelo SPEEDY (Molteni,
\protect\hyperlink{ref-Molteni2003}{2003}) para realizar corridas de
sensibilidad (descriptas en el
Capítulo~\ref{experimentos-de-sensibilidad}). El modelo SPEEDY es un
modelo de complejidad intermedia basado en ecuaciones primitivas
espectrales y parametrizaciones simplificadas. En su versión 41, posee
una resolución horizontal espectral de T30 (3,75° en longitud y latitud
y 8 niveles verticales entre 925hPa y 30hPa. El modelo incluye
parametrizaciones de las diferentes formas de convección, radiación,
flujos y difusión vertical.

Una primera gran limitación de SPEEDY es su pobre representación de la
estratósfera. Sólo el más alto de sus 8 niveles está en la estatósfera
(30hPa) y al ser la tapa del modelo, que tiene una ``esponja'' para
evitar la propagación de ondas de gravedad, no es un nivel con
información confiable. Esto limita seriamente la posibilidad de
describir numéricamente lo que ocurre en la estratosfera, como por
ejemplo la dinámica del vórtice polar.

Sin embargo por su bajo costo computacional y su buen desempeño para
simular ciertas características globales de interés para Sudamérica
(Barreiro et~al., \protect\hyperlink{ref-Barreiro2014}{2014}) se decidió
utilizarlo en este trabajo.

\section{Metodología}\label{metodologia}

\emph{Perturbaciones}

Muchas variables atmosféricas varían con la longitud en menor medida que
con la latitud o la altura de manera que resulta natural descomponer
cualquier variable \(\phi_{(x, y, z, t)}\) en un promedio zonal y las
desviaciones con respecto al mismo según:

\[
\phi_{(x, y, z, t)} = [\phi]_{(y, z, t)} + \phi_{(x, y, z, t)}^*
\] donde los corchetes indican el promedio zonal y el asterísco indica
el desvío con respecto al mismo. \([\phi]\) representa la circulación
simétrica zonal y es independiente de la longitud mientras que
\(\phi^*\) representa la circulación asimétrica estacionaria.

Estas ondas cuasiestacionarias (QS) representan una parte improtante de
los flujos meridionales de calor y cantidad de movimiento junto con las
ondas transcientes (James, \protect\hyperlink{ref-James}{1994}) y en
gran medida son forzadas por forzantes superficales como orografía y
contrastes de temperatura así como por interacciones con los
transcientes (Rao et~al., \protect\hyperlink{ref-Rao2004}{2004}).

En este trabajo las anomalías zonales se calcularon como la diferencia
entre la variable y su promedio zonal para cada círculo de latitud.

\emph{Fourier}

La caracterización de las ondas cuasiestacionarias a partir de Fourier
puede realizarse descomponiendo los campos medios climatológicos
(\emph{método AM}, por \emph{A}mplitud de la \emph{M}edia) o calculando
las características medias de la descomposición de los campos
instantáneos --mensuales, en este caso-- (\emph{método MA}, por
\emph{M}edia de la \emph{A}mplitud instantánea). Si bien pueden dar
resultados similares, ambas estrategias brindan distinta información.

A partir del método AM se obtiene una descripción de las ondas presentes
en los campos medios y que sobreviven a la interferencia destructiva del
promedio temporal, brindando información sobre las ondas estacionarias.
El método MA, en cambio, indica las propiedades medias de las ondas
planetarias independientemente de su fase, mezclando información de las
anomalías zonales instantáneas que no necesariamente son estacionarias.

Para los datos de NCEP, no se encontraron diferencias importantes entre
ambas metodologías, por lo que se muestran los resultados sólo del
método AM.

Para la descomposición de los campos de geopotencial en modos de Fourier
se utilizó la función \texttt{FitQsWave()} del paquete \texttt{metR}
(Campitelli, \protect\hyperlink{ref-R-metR}{2017}).

En la descomposición de Fourier, el estudio de la fase requiere un
tratamiento especial ya que se trata de una variable circular. Para el
cálculo de estadísticas circulares se se utilizó el paquete
\texttt{circular} (Lund et~al.,
\protect\hyperlink{ref-R-circular}{2017}).

\emph{Wavelets}

El análisis de wavelets es similar a Fourier pero permite describir
oscilaciones localizadas en el dominio (Torrence y Compo,
\protect\hyperlink{ref-Torrence1998}{1998}). En el contexto de las ondas
cuasiestacionarias, esto implica que en vez de obtener una amplitud y
fase por cada círculo de latitud, se tiene una amplitud local que
depende de la longitud. En la ciencias atmosféricas el análisis de
wavelets se usa extensivamente en el análisis de periodicidades en el
dominio tiempo (ej. Raphael, \protect\hyperlink{ref-Raphael2004}{2004};
Kinnard et~al., \protect\hyperlink{ref-Kinnard2011}{2011}) y de
procesamiento de imágenes (ej. Desrochers y Yee,
\protect\hyperlink{ref-Desrochers1999}{1999}), pero existen pocos
estudios que lo apliquen a dominios espaciales (ej. Pinault,
\protect\hyperlink{ref-Pinault2016}{2016}) aunque sí hay aplicaciones de
este tipo en la literatura de ecología (ej, Mi et~al.,
\protect\hyperlink{ref-Mi2005}{2005}).

Para el cálculo de wavelets se utilizó la función
\texttt{WaveletTransform()} del paquete \texttt{WaveletComp} (Roesch y
Schmidbauer, \protect\hyperlink{ref-R-WaveletComp}{2014}) que utiliza un
wavelet de Morlet.

\emph{Índice de QS3}

En el \autoref{onda-3} se elaboró un índice de actividad de la QS3 a
partir de la amplitud de Fourier promediada entre 65°S y 40°S de latitud
y entre 700hPa y 100hPa. Existen múltiples índices similares presentes
en la literatura. Algunos son definidos a partir anomalía zonales de
geopotencial en puntos fijos cercanos a donde climatológicamente se dan
los máximos de geopotencial asociados a la QS3. Así construyeron sus
índices Mo y White (\protect\hyperlink{ref-Mo1985}{1985}), Cai et~al.
(\protect\hyperlink{ref-Cai1999}{1999}) y ({\textbf{???}}) pero cada
trabajo utilizó puntos ligeramente distintos a causa de las distintas
climatologías analizadas y ({\textbf{???}}) además se basó en promedios
trimestrales de en vez de medias mensuales. Yuan y Li
(\protect\hyperlink{ref-Yuan2008}{2008}), utilizando una técnica
distinta, creó una serie temporal de actividad de la QS3 a partir de la
primer componente principal del campo de viento meridional superficial,
el cual resulta en un patrón de onda 3 consistente con climatologías
previas.

La principal limitación de estas metodologías es que, al basarse en
patrones estacionarios (puntos fijos, componentes principales), no
permiten cambios de fase en la onda planetaria. En particular, no
capturan correctamente el corrimiento estacional en la posición de la
QS3 (de aproximadamente 15° entre verano e invierno) (Loon y Jenne, Roy,
\protect\hyperlink{ref-Loon1972}{1972}) ni permiten capturar casos de
actividad de onda intensa pero alejada de las zonas climatológicamente
activas. Irving y Simmonds (\protect\hyperlink{ref-Irving2015}{2015})
construyó un índice de actividad de onda planetaria a partir de la
transformada de Hilbert que no tiene estas limitaciones pero no
distingue entre la actividad de distintos números de onda.

\emph{Interferencia constructiva y destructiva}

En la \autoref{fase} se estimó la proporción de casos de interferencia
destructiva a partir de la frecuencia relativa de casos donde la
diferencia de fase con respecto a la fase media es de entre 40° y 80°.
Dicha metodología se aplicó a cada mes. Dicho criterio teórico surge de
considerar la suma de dos ondas sinusoidales de igual amplitud

\[
\cos\left (k\phi \right) + \cos(k(\phi - \alpha)) = 2\cos\left( \frac{k\alpha}{2} \right)\cos\left( k\phi - \frac{\alpha}{2}\right) 
\] donde \(k\) es el número de onda, \(\phi\) la latitud, y \(\alpha\)
es la diferencia de fase entre las ondas. El primer término
multiplicativo del lado derecho es la amplitud de la nueva onda, la cual
depende de la diferencia de fase. Si la misma es menor a 1/3 o mayor a
2/3 de longitud de onda, el valor absoludo de la amplitud es mayor que 1
y la sumatoria de ondas es constructiva. De lo contrario, la onda
obtenida tiene una amplitud menor que las originales y la interferencia
es destructiva.

\emph{Herramientas dinámicas}

La influencia tropical en la variabilidad del clima extratropical se
explica mediante la propgación meridional de ondas de Rossby. La teoría
lineal de ondas de Rossby barotrópicas predice que ésta propagación
depende de las condiciones del entorno (James,
\protect\hyperlink{ref-James}{1994}). En particular, el número de onda
meridional está dado por

\[
l = \pm \sqrt{\frac{\eta_{y}}{U} - k^2}
\]

donde \(\eta_{y}\) es el gradiente meridional de vorticidad absoluta,
\(U\) es la velocidad zonal del estado básico y \(k\) es el número de
onda zonal. La propagación meridional sólo es posible si \(l\) es real.
Esto requiere que

\[
\frac{\eta_{y}}{U} = K_s \ge k 
\] donde \(K_s\) es el número de onda estacionario, el cual debe ser
mayor que el número de onda zonal para permitir la propagación
meridional.

El flujo de actividad de onda, es un vector que, bajo ciertas
suposiciones, es paralelo a la velocidad de grupo de las ondas de
Rossby, lo cual permite cuantificar la propagación meridional y zonal de
las mismas (James, \protect\hyperlink{ref-James}{1994}). En este
estudio, ésta variable se la calculó a partir de las anomalías zonales
de función corriente según

\[
\begin{aligned}
F_\lambda &= \frac{p}{2000a^2\cos\phi}\left[ \left( \frac{\partial \psi^*}{\partial \lambda} \right)^2 - \psi^*\frac{\partial^2 \psi^*}{\partial \lambda^2}  \right] \\
F_\phi &= \frac{p}{2000a^2} \left( \frac{\partial \psi^*}{\partial \lambda}\frac{\partial \psi^*}{\partial \phi}  - \psi^* \frac{\partial^2 \psi^*}{\partial \lambda \partial \phi} \right) 
\end{aligned}
\]

donde \(p\) es la presión, \(\phi\) la latidud, \(\lambda\) la longitud,
\(a\) es el radio de la Tierra (tomado como 6371km) y \(\psi^*\) es la
anomalía zonal de la función corriente (Vera et~al.,
\protect\hyperlink{ref-Vera2004}{2004}).

El análisis de componentes principales se realizó mediante la función
\texttt{EOF()} del paquete \texttt{metR} (Campitelli,
\protect\hyperlink{ref-R-metR}{2017}), la cual realiza una
descomposición en valores singulares de la matriz de datos. Este método
es preferible al cálculo de autovalores y autovectores sobre la matriz
de correlación o covarianza ya que es numéricamente más estable
(¿cita?). Los datos de geopotencial fueron pesados por la raiz cuadrada
del coseno de la latitud.

Se calcularon campos de anomalía zonal de geopotencial eliminando la
influencia de las ondas planetarias 1 y 2. Para esto se reconstruyeron
los campos de anomalía zonal de geopotencial de QS1 y QS2 mediante
Fourier y luego se restaron al campo de anomalía zonal total.

\chapter{Climatología observada}\label{climatologia-observada}

En este capítulo se presentan campos medios y anomalías zonales de
altura geopotencial, temperatura, viento zonal, viento meridinal,
función corriente, gradiente meridional de vorticidad absoluta y el
número de onda estacionario, como introducción general al estado medio
de la atmósfera sobre el cual se desarrollan las ondas estacionarias.
Luego se analizan los campos de amplitud y varianza explicada por las
ondas cuasiestacionarias (QS) en sí mismas.

\section{Altura geopotencial}\label{altura-geopotencial}

\begin{landscape}\begin{figure}

{\centering \includegraphics[width=247mm,]{fig/tesis/gh-ncep-1} 

}

\caption{Altura geopotencial media (NCEP). Contornos cada 250 mgp.}\label{fig:gh-ncep}
\end{figure}
\end{landscape}

El campo de altura geopotencial media (Z, \autoref{fig:gh-ncep}) muestra
una estructura marcadamente zonal en todos los niveles y estaciones. En
verano el gradiente meridional de Z es máximo en 200hPa, reduciéndose en
500hPa y por encima de 100hPa. En 50hPa el gradiente es prácticamente
nulo y en niveles superiores, éste se invierte en comparación a los
inferiores (no se muestra). En otoño el máximo de gradiente todavía se
da en 200hPa, pero continua siendo intenso en niveles superiores. En
invierno y primavera, el mayor gradiente se da en 50hPa y es mucho más
intenso que los observados en los demás niveles o estaciones. En
contraste con el resto de los niveles, 50hPa y 100hPa tienen mucha más
variabilidad estacional.

El aumento del gradiente meridional de geopotencial en invierno y
primavera en niveles altos está relacionado con la generación del
vórtice polar que aisla las latitudes polares de las latitudes medias.
En 200hPa, en cambio, es evidente el gradiente asociado con el jet
subtropical, que es más intenso en invierno y más débil en verano.

En la \autoref{fig:sd-gh.ncep} se muestra, para cada latitud y mes, el
desvío estándar de Z con respecto a la media zonal (\(\sigma_z\)). En
todos los niveles la variación meridional es muy similar con un máximo
principal al rededor de 60°S durante todo el año y un máximo secundario
en 30°S que aparece sólo entre junio y septiembre. El ciclo anual de
\(\sigma_z\) también es similar entre niveles aunque con marcadas
diferencias. En 100hPa el máximo absoluto se da en octubre, mientras que
en los niveles más bajos, éste se da en agosto.

\begin{figure*}
\includegraphics[width=155mm, ]{fig/tesis/sd-gh-ncep-1} \caption{Desvío estándar de Z por círculo de latitud (NCEP).}\label{fig:sd-gh-ncep}
\end{figure*}

\begin{landscape}\begin{figure}

{\centering \includegraphics[width=247mm,]{fig/tesis/ghz-ncep-1} 

}

\caption{Anomalía zonal de altura geopotencial (NCEP).}\label{fig:ghz-ncep}
\end{figure}
\end{landscape}

Las anomalías zonales de geopotencial (Z*, \autoref{fig:ghz-ncep})
muestran una preponderancia de la onda 1 (QS1) con una amplitud máxima
en la estratósfera de primavera. Pueden diferenciarse dos QS1 distintas;
una centrada en \textasciitilde{}60°S y con el centro anticiclónico al
rededor de la línea de fecha, y la otra centrada en 75°S sobre la costa
del continente antártico y el centro anticiclónico entre 120 y 60°O.
Quintanar y Mechoso
(\protect\hyperlink{ref-Quintanar1995}{1995}\protect\hyperlink{ref-Quintanar1995}{b})
concluyeron que la primer QS1 está asociada principalmente con forzantes
de latitudes bajas mientras que la segunda responde a la orografía del
continente antártico.

En latitudes tropicales, en verano hay una anomalía negativa sobre el
Pacífico este con máxima amplitud en 200hPa que está presente en las
otras estaciones con menor intensidad. Sobre Sudamérica, en verano y
primavera en ese mismo nivel aparece un centro anticiclónico con un
centro ciclónico al noroeste. Estas anomalías (la alta de Bolivia y la
baja del noroeste) son características del Sistema Monzónico
Sudamericano (Vera et~al., \protect\hyperlink{ref-Vera2006}{2006}).

En la \autoref{fig:ghz-ncep-corte60} se muestra un corte zonal en 60°S
de Z*. Se aprecia la coherencia vertical de la QS1 y es evidente la
inclinación hacia el oeste con la altura en todas las estaciones salvo
en verano, en coincidencia con lo encontrado por Quintanar y Mechoso
(\protect\hyperlink{ref-Quintanar1995a}{1995}\protect\hyperlink{ref-Quintanar1995a}{a}).
Karoly (\protect\hyperlink{ref-Karoly1985}{1985}) describió una
estructura equivalente barotrópica tanto en verano como en invierno en
55°S, pero esto se debe a la falta de niveles verticales por encima de
los 100hPa que es donde la inclinación es más importante.

La inclinación hacia el oeste con la latitud (\autoref{fig:ghz-ncep}) y
con la altura (\autoref{fig:ghz-ncep-corte60}) indican que las
perturbaciones estacionarias están asociadas con transporte hacia el
polo tanto de cantidad de movimiento zonal como de temperatura, por las
perturbaciones (James, \protect\hyperlink{ref-James}{1994}). En verano
las anomalías zonales tienen una estructura barotrópica equivalente y
carecen de inclinación en la horizontal.

\begin{figure*}
\includegraphics[width=155mm, ]{fig/tesis/ghz-ncep-corte60-1} \caption{Corte zonal de anomalía de geopotencial en -60° (NCEP).}\label{fig:ghz-ncep-corte60}
\end{figure*}

\section{Temperatura}\label{temperatura}

\begin{landscape}\begin{figure}

{\centering \includegraphics[width=247mm,]{fig/tesis/t-ncep-1} 

}

\caption{Temperatura media (NCEP). Contornos cada 5K.}\label{fig:t-ncep}
\end{figure}
\end{landscape}

La distribución horizontal de la temperatura media
(\autoref{fig:t-ncep}), al igual que la altura geopotencial media, tiene
una estructura principalmente zonal en todos los niveles y estaciones.
Por debajo de los 200hPa, donde el gradiente meridional de temperatura
es mínimo, la temperatura disminuye con la latitud en todas las
estaciones. Por encima de este nivel, en cambio, en verano la
temperatura crece con la latitud, y en el resto de las estaciones
muestra un máximo centrado en 60°S en otoño y en 45°S en invierno y
primavera. El nivel de 850hPa se ven las asimetrías zonales más
importantes asociadas con los contrastes de temperatura entre continente
y océanos, como el máximo sobre Australia en verano.

En 300hPa, el gradiente meridional de temperatura en latitudes medias
tiene un importante ciclo anual con máximo en invierno y mínimo en
verano. En niveles inferiores, el ciclo anual es menos marcado y más
dependiente de la latitud. En 500hPa, en 150°E en 30°S se da un máximo
del gradiente meridional de temperatura en invierno con mínimo en verano
mientras que en 45°S el ciclo es el inverso.

\begin{figure*}
\includegraphics[width=155mm, ]{fig/tesis/t-ncep-corte-1} \caption{Media zonal de la temperatura para cada nivel y latitud (NCEP).}\label{fig:t-ncep-corte}
\end{figure*}

El promedio zonal de la temperatura se muestra en la
\autoref{fig:t-ncep-corte}. Al norte de los 45° durante todo el año la
tempertura decrece con la altura por debajo de los 100hPa
aproximadamente, donde alcanza un mínimo que marca la tropopausa, y
crece por encima. La altura del mínimo de temperatura varía mucho al sur
de los 45°, siendo mínima en verano (300hPa) y máxima en invierno y
otoño (30hPa). Hay que tener encuenta, sin embargo, que la tropopausa no
está bien definida en el invierno Antártico (Court,
\protect\hyperlink{ref-Court1942}{1942} Zängl
(\protect\hyperlink{ref-Zangl2001}{2001})) por lo que el uso de la
tropopausa térmica no es conveniente. Zängl
(\protect\hyperlink{ref-Zangl2001}{2001}), utilizando datos de ERA para
el período 1979-93, encontraron que la tropopausa térmica varía entre
los 320hPa en enero y los 170hPa en agosto aunque advierten que el uso
del criterio término es ``problemático'' en el invierno antártico.

\begin{landscape}\begin{figure}

{\centering \includegraphics[width=247mm,]{fig/tesis/tz-ncep-1} 

}

\caption{Anomalía zonal de temperatura (NCEP).}\label{fig:tz-ncep}
\end{figure}
\end{landscape}

En la \autoref{fig:tz-ncep} se muestran las anomalías zonales de
temperatura media. En 850hPa se aprecia el efecto del contraste de
temperatura entre el suelo y el mar. Se observan anomalías positivas
sobre los continentes y negativas sobre los océanos en todas las
estaciones, aunque más intensas en verano y primavera. En niveles más
altos éstas pierden su intensidad pero reaparecen en 100hPa con signo
invertido. Estas características tienen su correlato en la altura
geopotencial y corresponden a circulaciones tipo monzónico.

En invierno y primavera, los niveles altos están dominados por una QS1
con máximo en el sur de Australia y mínimo en el Atlántico sur. En
niveles más bajos, la onda disminuye su amplitud y se defasa hacia el
este y queda casi en cuadratura con la onda de niveles altos
(\autoref{fig:t-ncep-corte60}) presentando un máximo en 850hPa en
Antártida occidental. En otoño esta onda está presente pero con amplitud
muy reducida y máxima en niveles medios. Finalmente, en verano ésta
desaparece por encima de 100hPa.

\begin{figure*}
\includegraphics[width=155mm, ]{fig/tesis/t-ncep-corte60-1} \caption{Corte zonal de anomalía zonal de temperatura en -60° (NCEP).}\label{fig:t-ncep-corte60}
\end{figure*}

En la \autoref{fig:t-nceo-corte60} se muestra un corte vertical-zonal en
60°S de la anomalía zonal de temperatura. Las anomalías por debajo de
300hPa mantienen su intensidad durante todo el año, aunque tienen más
extensión en invierno y primavera en comparación con verano y otoño y
son barotrópicas equivalentes. Por encima de 300hPa, en cambio, se
observa un importante ciclo anual con máximo en primavera y mínimo en
verano con importante inclinación hacia el oeste con la altura.

\section{Viento zonal}\label{viento-zonal}

\begin{figure*}
\includegraphics[width=155mm, ]{fig/tesis/u-ncep-corte-1} \caption{Media zonal del viento zonal para cada nivel y latitud (NCEP).}\label{fig:u-ncep-corte}
\end{figure*}

La media zonal del viento zonal (\autoref{fig:u-ncep-corte}) muestra dos
máximos, uno en latitudes medias en 200hPa y otro en latitudes plares en
la estatósfera, correspondientes al jet subtropical y subpolar,
respectivamente. El primero está presente durante todo el año aunque con
mayor intensidad y corrido hacia latitudes más ecuatoriales en invierno
y primavera. El segundo está presente principalmente en invierno y
primavera, e incipiente en otoño. Además enn la estratósfera se observan
vientos del este en latitudes bajas que son más intensos en verano y
otoño.

\begin{landscape}\begin{figure}

{\centering \includegraphics[width=247mm,]{fig/tesis/u-ncep-1} 

}

\caption{Viento zonal medio (NCEP).}\label{fig:u-ncep}
\end{figure}
\end{landscape}

En la \autoref{fig:u-ncep} se muestra el viento zonal medio. Se observa,
principalmente en invierno y primavera por encima de 100hPa, que el jet
subpolar es más intenso al sur de África, donde además se encuentra en
una latitud más ecuatorial que en la región del Pacífico. El jet
subtropical también tiene un máximo al sur de África y otro al norte de
Nueva Zelanda --especialmente en invierno--, donde además se produce una
bifurcación del jet. Se trata de una región de persistentes y frecuentes
bloqueos (ej. Trenberth y Mo,
\protect\hyperlink{ref-Trenberth1985}{1985}).

Esta bifurcación del jet sobre Nueva Zelanda se evidencia en el campo de
anomalías zonales de viento zonal (\autoref{fig:uz-ncep}) como una
anomalía negativa sobre la isla acompañada por anomalías positivas al
norte y sur. Este campo también presenta varios pares de QS1
antisimétricos respecto a 60°S. Estas anomalías se corresponden con la
variación meridional del jet observado en la \autoref{fig:u-ncep} y son
consistentes con la QS1 de geopotencial observada en la
\autoref{fig:ghz-ncep}. Por el balance de viento geostrófico centros
anticilónico de Z* están flanqueados por anomalías zonales negativas de
viento zonal al norte y positivas al sur y viceversa para los centros
ciclónicos. Además, como el viento zonal no depende del parámetro de
Coriolis como Z, esto permite la identificación de QS en regiones
tropicales. Un ejemplo de esto XXXX???

En verano, entre 300hPa y 100hPa sobre el Pacífico ecuatorial existe una
zona de anomalías del viento zonal positivas al este y negativas al
oeste. Consistente con los campos de temperatura, esto implica
divergencias en niveles altos y convergencias en niveles bajos (no se
muestra). Evidencia de la circulación tropical forzada por la
temperatura superficial del pacífico.

\begin{landscape}\begin{figure}

{\centering \includegraphics[width=247mm,]{fig/tesis/uz-ncep-1} 

}

\caption{Anomalía zonal de viento zonal (NCEP).}\label{fig:uz-ncep}
\end{figure}
\end{landscape}

Los campos de viento zonal, temperatura y altura geopotencial están
ligados por el balance de viento geostrófico y de viento térmico. El
máximo del jet se encuentra en regiones de máxima baroclinicidad y
máximo gradiente meridional de Z y donde el gradiente meridional de
temperatura se anula.

\section{Viento meridional}\label{viento-meridional}

\begin{figure*}
\includegraphics[width=155mm, ]{fig/tesis/v-ncep-corte-1} \caption{Media zonal del viento meridional (NCEP).}\label{fig:v-ncep-corte}
\end{figure*}

El corte vertical-meridional del promedio zonal viento meridional medio
(\autoref{fig:v-ncep-corte}) muestra los máximos tropicales presentes en
superficie y altura en todas las estaciones, relacionados con la
circulación de Hadley. En verano, la rama ascendente se encuentra en el
hemisferio sur y se tiene convergencias en niveles bajos y divergencias
en niveles altos. En invierno, en cambio, sólo se ve la rama
descendente, mucho más intensa que en verano, que genera convergencias
en niveles altos y divergencias en niveles bajos al rededor de los 30°S.

Presente durante todo el año, también se observa un maximo de vientos
del sur en la costa antártica. Esta es la señal de los vientos
catabáticos antárticos producidos por una capa muy estable cerca de
superficie y la consistente inclinación de la topografía del continente
(King y Turner, \protect\hyperlink{ref-King1997}{1997}). Aunque los
datos allí pueden tener limitaciones por la falta de observaciones y la
pobre representación de la orografía en los modelos.

\begin{landscape}\begin{figure}

{\centering \includegraphics[width=247mm,]{fig/tesis/v-ncep-1} 

}

\caption{Viento meridional medio (NCEP).}\label{fig:v-ncep}
\end{figure}
\end{landscape}

Los campos horizontales de V se muestran en la \autoref{fig:v-ncep}.
Consistente con los campos de Z* (\autoref{fig:ghz-ncep}), en niveles
altos se observa una QS1 que alcanza su máximo en la estratósfera de
primavera.

En invierno entre 500hPa y 100hPa, existe evidencia de un tren de ondas
de Rossby que se propaga desde el Índico occidental sudeste llegando a
su máxima latitud en 150°O donde comienza a propagarse hacia el norte
hasta llegar al sur de Sudamérica. Este tren de ondas puede
identificarse en el campo de Z* (\autoref{fig:ghz-ncep}), pero con mayor
dificultad debido a la interferencia de la QS1 y a la dependencia de Z*
con el parámetro de Coriolis, es decir, la latitud. El tren de ondas se
distingue en la troposfera alta, también en primavera y con menor
intensidad en verano y otoño.

Las anomalías zonales del viento meridional también permiten distinguir
otras características cuasiestacionarias del clima, como aquellas
asociadas con los monzones. En el invierno, en los trópicos se puede
observar la anomalía positiva en 850hPa en la costa oeste de África
asociada con el flujo hacia el monzón de la India. En altura, el monzón
de la India se evidencia en esa estación como anomalía de viento hacia
el sur producto de la divergencia de niveles altos generada por la
convección anómala. Por otra parte, en verano se evidencia en la
troposfera alta sobre Sudamérica tropical anomalías positivas sobre el
centro-este del continente y negativas en los océanos subyacentes. Tal
patrón de ondas estacionario se relaciona con la presencia del Alta de
Bolivia y las vaguadas estacionarias a sus lados, siendo la del este
típicamente llamada la Baja del Nordeste (Vera et~al.,
\protect\hyperlink{ref-Vera2006}{2006}).

No se muestran los campos de anomalías zonales de V ya que son
virtualmente idénticos a los campos de V porque la media zonal de esta
variable es virtualmente nula en todo el dominio.

\begin{landscape}\end{landscape}

\section{Función corriente}\label{funcion-corriente}

\begin{landscape}\begin{figure}

{\centering \includegraphics[width=247mm,]{fig/tesis/psi-ncep-1} 

}

\caption{Función corriente media en $\sigma = 0.2101$ (contornos cada $2\times10^{-11}m^2/s$), anomalía zonal de función corriente (sombreado,  $1\times10^{-9}m^2/s$) y flujos de actividad de onda medios (NCEP).}\label{fig:psi-ncep}
\end{figure}
\end{landscape}

Para analizar la influencia tropical en la circulación no es posible
usar la altura geopotencial, ya que el balance geostrófico pierde
validez cerca del ecuador. Por lo tanto, es útil analizar el campo de
función corriente (\(\psi\)). El reanálisis de NCEP provee esta variable
en niveles sigma en vez de presión. En la \autoref{fig:psi-ncep} se
muestra \(\psi\) en el nivel 0.2101 sigma, que equivale aproximadamente
a 250hPa. Además del campo medio, se muestran las anomalías zonales en
sombreado y los flujos de actividad de onda en flechas.

Las características del campo de \(\psi\), tanto el total como las
anomalías zonales, es similar al de Z (\autoref{fig:gh-ncep}, 200hPa)
con una estructura eminentemente zonal y un aumento del gradiente
meridional en invierno y primavera y los mismos centros de anomalías. La
principal diferencia es, además del cambio de signo dada por la
dependencia de \(\psi\) con f, es que los patrones presentes en las
latitudes tropicales se ven con mayor magnitud que los de latitudes
medias y altas.

Los flujos de actividad de onda en verano muestran transporte de energía
desde el Pacífico este hacia el sur de África pasando por el Atlántico
que se sostiene durante todo el año con menor intensidad. Desde ese
lugar también se observa transferencia de energía hacia el hemisferio
norte, que se junta con otra región de flujos intensos que viene desde
el Pacífico oeste. Sobre el Índico, los flujos son de mayor magnitud en
invierno transportando energía hacia el sur.

\section{Propagacion Meridional de Ondas de
Rossby}\label{propagacion-meridional-de-ondas-de-rossby}

\begin{landscape}\begin{figure}

{\centering \includegraphics[width=247mm,]{fig/tesis/etady-ncep-1} 

}

\caption{Gradiente meridional de vorticidad absoluta ($1\times10^11(ms)^{-1}$)}\label{fig:etady-ncep}
\end{figure}
\end{landscape}

La \autoref{fig:etady-ncep} muestra el campo de \(\psi_y\) con una
región con valores negativos centrada sobre Nueva Zelanda, en invierno
entre 300hPa y 200hPa, coincidiendo con la región de bloqueos. Está
flanqueada por el jet subtropical intenso y el jet subpolar más al sur,
dando lugar a gradientes meridionales de viento zonal negativos más
intensos que \(\beta\). Otras regiones con valores negativos se
distinguen en la costa antártica.

Valores de \(\psi_y\) negativos son un factor que impiden la propagación
meridional de ondas de Rossby barotrópicas. Esta figura reproduce y
extiende el resultado de Berbery et~al.
(\protect\hyperlink{ref-Berbery1992}{1992}) (su Figura~3) utilizando 5
años de análisis objetivo del Centro Europeo de Predicción a Plazo Medio
(ECMWF). Asimismo, aún con \(\psi_y\) positivo, las ondas de Rossby
barotrópicas sólo se pueden propagar si su número de onda zonal es menor
que el número de onda estacionario (James,
\protect\hyperlink{ref-James}{1994}). En la \autoref{fig:ks-ncep} se
muestra el número de onda estacionario para el nivel de 200hPa. Entre
otoño y primavera, se destaca claramente la región del Pacífico Oeste en
la cual la propagación meridional está inhibida. Además, en todas las
estaciones, las ondas cortas no pueden propagarse meridionalmente en
latitudes altas. La \autoref{fig:ks-ncep} muestra que las ondas de
Rossby barotrópicas con k menores o iguales a 3 pueden propagarse
meridionalmente en las cuatro estaciones hasta aproximadamente los 60°S,
excepto en la franja de longitudes entre 60°E y 120°O, donde quedan
atrapadas al sur de 45°S (excepto en verano).

La \autoref{fig:ks-ncep-corte} se muestra un corte vertical en 180° del
número de onda estacionario. Se observa que en la troposfera alta se dan
las condiciones para la propagación desde latitudes más bajas hasta las
subpolares. Además se destaca que la zona de inhibición de la
propagación meridional es una característica de la troposfera alta
solamente, lo que confirma de alguna manera el papel del jet subtropical
en producirla.

\begin{landscape}\begin{figure}

{\centering \includegraphics[width=247mm,]{fig/tesis/ks-ncep-1} 

}

\caption{Número de onda estacionario en 200hPa (NCEP).}\label{fig:ks-ncep}
\end{figure}
\end{landscape}

\section{Ondas Quasiestacionarias}\label{ondas-quasiestacionarias}

La \autoref{fig:r2-ncep} muestra el corte vertical-meridional de \(r^2\)
de Fourier para las ondas estacionarias 1 a 4. Consistente con lo
encontrado previamente en el campo de Z*, la QS1 explica la mayor parte
de la variabilidad en todo el dominio al sur de los 45°S. La QS2 es
preponderante en la estratósfera ecuatorial, en la costa antártica y
alrededor de 35°S, donde es el modo dominante en toda la columna de aire
en verano. La QS3, a diferencia de las ondas anteriores es importante en
una región reducida. Explica una parte substancial de la varianza en
niveles bajos al rededor de los 45°S y mayormente en invierno. La QS4
explica muy poca varianza a excepción de cerca de superficie entre 15°S
y 30°S. Ondas más cortas son aún menos importantes (no se muestra).

\begin{landscape}\begin{figure}

{\centering \includegraphics[width=247mm,]{fig/tesis/r2-ncep-1} 

}

\caption{$R^2$ de Fourier para números de onda 1 a 4 (NCEP).}\label{fig:r2-ncep}
\end{figure}
\end{landscape}

El \(r^2\) permite analizar la importancia relativa de cada modo con
respecto a la variabilidad total, pero lo que importa desde el punto de
vista físico es la amplitud de la onda. La \autoref{fig:ampl-ncep}
muestra cortes verticales-meridionales de la amplitud. Las diferencias
entre los campos de \(r^2\) y los de amplitud son evidentes comparando
con la \autoref{fig:ampl-ncep} (notar la escala logarítmica en los
colores). La amplitud de la QS1 muestra un máximo bien definido centrado
en 60°S que en verano se encuentra en niveles más bajos que en las otras
estaciones. También existe un máximo relativo entre 15°S y 30°S en
verano que migra a latitudes más altas en invierno y primavera. El mismo
está presente también en las otras ondas estacionarias.

En el caso de la QS2, se evidencia que a pesar de tener máximos de
\(r^2\) en la estratósfera al norte de 45°S, alcanza su máxima amplitud
al sur de esa latitud y en 200hPa en verano y en 30hPa en invierno. Su
actividad en la costa antártica se extiende en toda la tropósfera en
invierno (a pesar de que en \(r^2\) pierde importancia por encima de los
200hPa)

La región de amplitud máxima de la QS3, coincide aproximadamente con la
región de máximo \(r^2\) entre y otoño y primavera, aunque con menos
actividad en superficie y extensión en toda la columna. En verano, en
cambio aparece un máximo de amplitud importante que no se observa en el
campo de \(r^2\).

Finalmente, fuera de la superficie, la QS4 presenta un máximo de
amplitud bien definido sólo en verano. El máximo entre 15°S y 30°S sigue
presente.

\begin{landscape}\begin{figure}

{\centering \includegraphics[width=247mm,]{fig/tesis/ampl-ncep-1} 

}

\caption{Ampllitud de Fourier para números de onda 1 a 4 (NCEP).}\label{fig:ampl-ncep}
\end{figure}
\end{landscape}

XXX Acá podría poner alguna reflexión general de lo que se ve. Por
ejemplo, cómo en verano la variabilidad está más acotada a la tropóstera
mientras que en invierno y primavera hay más contacto con la
estratósfera. XXX

\chapter{Onda 3}\label{onda-3}

En este capítulo se analiza la climatología observada de la QS3 a partir
de su reconstrucción mediante descomposición de Fourier. Se estudia su
amplitud y fase y se propone una división estacional alternativa a
partir de estas variables y el análisis de componentes principales.
Además, se muestran la estructura típica de la altura geopotencial, la
función corriente y la SST asociada a la QS3 a partir de regresiones con
respecto a la amplitud de la QS3.

\section{Características típicas}\label{caracteristicas-tipicas}

\begin{figure*}
\includegraphics[width=155mm, ]{fig/tesis/qs3-ncep-1} \caption{Z* media reconstruida a partir de la QS3 (NCEP).}\label{fig:qs3-ncep}
\end{figure*}

En la \autoref{fig:qs3-ncep} se muestra el campo de Z* reconstruido sólo
a partir de la QS3 en 300 hPa, que ilustra lo que sucede en todos los
niveles dado que su estructura es barotrópica equivalente (como se verá
en la \autoref{fig:qs3-ncep-corte}). De acuerdo con lo encontrado en la
amplitud de Fourier para la QS3 (\autoref{fig:ampl-ncep}), la amplitud
es máxima entre 60°S y 45°S con menor intensidad en primavera. En el
verano la fase es tal que una de las tres anomalías anticiclónicas se
ubica hacia el este del sur de Sudamérica. Se observa que existe un
corrimiento de la fase entre verano e invierno de poco más de 15° (algo
ya observado por Loon y Jenne, Roy
(\protect\hyperlink{ref-Loon1972}{1972}) y Mo y White
(\protect\hyperlink{ref-Mo1985}{1985})) anticipando que el efecto de la
QS3 sobre cada lugar pueda tener una componente estacional.

Además, las anomalías presentan una inclinación meridional que, como la
teoría de ondas de Rossby indica, está asociada a transportes de
cantidad de movimiento hacia el polo. La inclinación es más importante
en las estaciones de transición, pero menor en verano y no detectable en
el invierno.

\begin{figure*}
\includegraphics[width=155mm, ]{fig/tesis/qs3-ncep-corte-1} \caption{Corte zonal en 60°S de Z* reconstruida a partir de la QS3 (NCEP).}\label{fig:qs3-ncep-corte}
\end{figure*}

La estructura vertical de las perturbaciones zonales de geopotencial
reconstruidas a partir de la QS3 se presenta en la
\autoref{fig:qs3-ncep-corte}. En todas las estaciones la mayor amplitud
se da en 300hPa, con el máximo desarrollo vertical en invierno, seguido
por el otoño y verano. Se destaca además la variación estacional de la
fase, descripta previamente. Asimismo, se observa que existe una
variación estacional en la inclinación vertical de las perturbaciones.
En invierno se observa una ligera inclinación hacia el oeste por debajo
de los 300hPa. Algo que no se detecta en las otras estaciones.

\begin{figure*}
\includegraphics[width=155mm, ]{fig/tesis/qs3sd-ncep-1} \caption{Desvío estándar temporal de Z* reconstruida a patrir de la QS3. Se incluyen en negro, contornos que describen la posición de los centros de las perturbaciones (NCEP).}\label{fig:qs3sd-ncep}
\end{figure*}

En la \autoref{fig:qs3sd-ncep} se muestra el desvío estándar de las
perturbaciones reconstruidas a partir de la QS3. La variabilidad máxima
se da entre los centros ciclónicos y anticiclónicos (marcados en la
\autoref{fig:qs3sd-ncep} con contornos negros), indicando que la
variabilidad del geopotencial asociada a la QS3 se debería
principalmente corrimientos de la fase que ocurren dentro de cada
estación.

\subsection{Wavelets}\label{wavelets}

\begin{figure*}
\includegraphics[width=155mm, ]{fig/tesis/wavelet-fourier-ncep-1} \caption{Amplitud de la QS3 según wavelets (sombreados) y Fourier (contornos) (NCEP).}\label{fig:wavelet-fourier-ncep}
\end{figure*}

Como se describió en el Capítulo~\ref{metodologia}, se utilizó la
metodología de wavelets como alternativa a la de Fourier para aislar la
QS3. La amplitud media obtenida mediante wavelets es virtualmente
idéntica a la amplitud obtenida con Fourier
(\autoref{fig:wavelet-fourier-ncep}).

La metodología de wavelets también permite obtener información sobre la
variación meridional de la amplitud de la QS3, obteniendo así campos
horizontales de la amplitud de la QS3. Al igual que con Fourier, esto
puede hacerse promediando estacionalmente la amplitud de los campos
mensuales o calculando la amplitud de los campos estacionales. Los
valores de amplitud media zonal como los mostrados en la
\autoref{fig:wavelet-fourier-ncep} son similares en ambas metodologías
(no se muestra) pero sí hay diferencias en las anomalías con respecto a
la media zonal.

\begin{figure}

{\centering \includegraphics[width=155mm, ]{fig/tesis/waveletz-ncep-1} 

}

\caption{Anomalía zonal de la amplitud de la QS3 según wavelets en 300hPa para el método AM y MA (\autoref{metodologia}) (NCEP).}\label{fig:waveletz-ncep}
\end{figure}

Las anomalías zonales de la amplitud de la QS3 según wavelets se
muestran en la \autoref{fig:waveletz-ncep} para los dos métodos
utilizados para el nivel de 300hPa. Los valores positivos representan
regiones donde la amplitud de la QS3 es mayor que la media zonal y
viceversa. Al norte de los 30°S, ambas metodologías coinciden resultados
similares: en verano, otoño y primavera hay anomalías zonales positivas
sobre el Pacífico y negativas en el Índico, indicando que en estas
latitudes la QS3 tiene mayor amplitud en el hemisferio occidental que en
el oriental. En invierno, no hay anomalías zonales importantes en esta
región.

Al sur de 30°S las dos metodologías divergen. Ambas metodologías
presentan en todas las estaciones, una región elongada de máxima
ampltiud desde el sur del Índico centrado en 45°S que se desplaza hacia
el este en latitudes más altas hasta llegar al los 60°S aproximadamente.
La principal diferencia entre las metodologías es que la amplitud de la
media estacional suele tener los máximos desvíos en 45°S, mientras que
la media de la amplitud instantánea tiene su máximo al sur de los 60°S.

\begin{figure*}
\includegraphics[width=155mm, ]{fig/tesis/estacionaridad-1} \caption{Estacionariedad de la QS3 estimada como la división entre la amplitud del método AM y la amplitud del método MA en 300hPa (NCEP).}\label{fig:estacionaridad}
\end{figure*}

Estas diferencias en la \autoref{fig:waveletz-ncep} están relacionadas
con la estacionariedad de las ondas y se puede utilizar la relación
entre ambas para generar un índice de estacionariedad definido como el
cociente entre la amplitud de la media y la media de la amplitud.
Valores cercanos a 1, indican que el carácter estacionario de las ondas
sería el dominante, mientras que valores cercanos a 0 implican que las
ondas no tienen una localización preferencial. El temapo de este índice
de estacionariedad se muestra en la \autoref{fig:estacionaridad}. La
máxima estacionaridad se consigue en el verano tropical, pero esto se
debe a que la actividad de ondas estacionarias es casi nula.

Aproximadamente entre 30° y 60° se observa una banda de alta
estacionaridad, consistente con la región con mayor amplitud de QS3
(\autoref{fig:qs3-ncep}), que presenta un máximo al sur del Índico. Esto
sugiere que esta región la actividad de onda es particularmente
estacionaria, en contra de la sugerencia de Hobbs y Raphael
(\protect\hyperlink{ref-Hobbs2010}{2010}) de que el principal patrón
estacionario se da en el Pacífico Sur.

Estas observaciones destacan la utilidad de wavelets en el análisis de
ondas cuasiestacionarias. Mientras que el tratamiento con Fourier asume
a las ondas como una propiedad media de cada círculo de latitud,
wavelets permite reconocer su heterogeneidad meridional. Evaluando esta
heterogeneidad, sería posible distinguir entre campos donde una onda con
un determinado número de onda está presente en todo un círculo de
latitud de campos donde ésta está localizada en una región acotada.

Por otro lado, la no ortogonalidad de los wavelets complejizan la
interpretación de los resultados ya que no es posible la separación de
un campo en modos oscilatorios con distinto número de onda. El análisis
de una QS específica, por lo tanto, está contaminado por la actividad de
otras QS con longitud de onda cercana.

Wavelets, en resumen, puede entenderse como una \emph{corrección} a
Fourier que agrega información de asimetrías zonales. Si embargo, dado
que la variabilidad zonal es del orden de un 10\% de la amplitud media,
en lo que sigue de la tesis se utilizará sólo Fourier, dejando el
análisis e interpretación de estudios de la QS3 utilizando wavelets para
futuros trabajos.

\section{Amplitud}\label{amplitud}

Existen varios estadísticos que podrían utilizarse para representar la
amplitud de la QS3 en una región extendida, como la media, la máxima, la
moda, la mediana, etc. En este caso, se estudió la posibilidad de
representarla con la media o la máxima. La zona de interés sobre la cual
se calcularon el promedio y el máximo es la región comprendida entre los
65°S y 40°S y entre 700hPa y 100hPa.

\begin{figure*}
\includegraphics[width=155mm, ]{fig/tesis/ampl-max-mean-1} \caption{Amplitud de la QS3 máxima y media para 9 casos seleccionados.}\label{fig:ampl-max-mean}
\end{figure*}

\begin{figure*}
\includegraphics[width=155mm, ]{fig/tesis/ampl-max-mean-corte-1} \caption{Corte vertical de amplitud de la QS3 para 9 casos seleccionados.}\label{fig:ampl-max-mean-corte}
\end{figure*}

\begin{figure*}
\includegraphics[width=155mm, ]{fig/tesis/ghz-ncep-select-1} \caption{Z* en 300hPa con QS1 y QS2 eliminadas para 9 casos seleccionados.}\label{fig:ghz-ncep-select}
\end{figure*}

A modo exploratorio se seleccionaron 9 casos que representan distintas
características de la amplitud media y la máxima. Sus magnitudes
correspondientes se muestran en la \autoref{fig:ampl-max-mean}, y los
campos de Z* (con las contribuciones de las QS1 y QS2 removidas como se
explicó en el Capítulo~\ref{metodologia}) en la
\autoref{fig:ghz-ncep-select}.

Comparando el caso de mayo de 1997 con el de abril de 2012, ambos tienen
una media muy similar (\autoref{fig:ampl-max-mean}), pero la máxima del
primero es menor que la del segundo. El corte vertical de la amplitud
(\autoref{fig:ampl-max-mean-corte}) muestra que la amplitud de la QS3 en
abril de 2012 era más grande pero más localizada en la troposfera alta
que en el caso de mayo de 1997 en el que la amplitud es mayor en la
estratosfera. Esto se refleja en la comparación de las perturbaciones de
geopotencial (\autoref{fig:ghz-ncep-select}), donde la QS3 se aprecia
mucho más claramente en 2012 que en 1997.

Enero de 1985 y julio de 1988 también son similares en amplitud media
pero distintos en amplitud máxima. La estructura vertical de la amplitud
muestran diferencias en la extensión similares al par de meses anterior,
con la amplitud en enero de 1985 presentando un máximo bien definido en
la tropósfera mientras que en julio de 1988 alcanzando valores
importantes en niveles estratosféricos. Los campos de Z* son muy
similares en cuanto a intensidad y claridad de la QS3. Los dos meses
presentan un tren de ondas que ocupan aproximadamente 1/3 de círculo de
latitud. A pesar de que la amplitud máxima de 1985 es menor que la de
1988, el tren de ondas de 1988 se ve algo más claro que el de 1985.

En los dos pares de casos anteriores, los meses tenían distinta amplitud
máxima y similar amplitud media, el par noviembre de 1987 y enero de
2008 son el caso lo contrario. El corte vertical de la amplitud de la
QS3 indica que ésta está más extendida vertical y meridionalmente en
noviembre de 1987 que en enero de 2008. Pero si se observa el campo de
Z*, si bien ambos casos tienen una QS3 clara, las anomalías en enero de
2008 son más intensas, especialmente en la región del Índico.

El par septiembre de 2000 y diciembre de 1990 es más claro. Ambas
medidas de amplitud son mayores en diciembre de 1990 que en septiembre
de 2000 y puede verse que ambos meses tienen un desarrollo vertical
similar de la amplitud de la QS3. El campo de Z* es consistente con
estas observaciones, mostrando que en diciembre de 1990 la estructura de
la QS3 era más zonal y detectable. Sin embargo, las anomalías que sí
están presentes en septiembre de 2000 --un tren de ondas similar al de
enero de 1985, aunque con distinta fase-- son más intensas, por lo que
su efecto local puede ser mayor que las de diciembre de 1990.

Más extrema aún es la diferencia entre septiembre de 2000 y octubre de
2003. Ambos meses tienen métricas de amplitud similares tanto en media
como en máxima y una extensión vertical de la QS3 similar. No obstante,
la QS3 de octubre de 2003 es apenas distinguible en comparación con
septiembre de 2000.

Estos casos ilustran la diversidad de situaciones en que se puede
desarrollar la QS3 y las limitaciones que un estudio
dinámico-estadístico como el que sigue, puede tener. Algunos problemas
son inherentes a cualquier intento de representar una estructura con
variabilidad espacial a partir de un sólo número o indicador y otros
están relacionados con la limitación de la descomposición de Fourier que
trata toda onda como una onda planetaria con igual estructura a lo largo
de todo un círculo de latitud.

\begin{figure*}
\includegraphics[width=155mm, ]{fig/tesis/cor-mean-max-1} \caption{Valores de correlación entre la amplitud máxima y media, ambas computadas sobre el período 1985-2015. La línea sólida representa un suavizado utilizando LOESS y la línea punteada es la lína x = y.}\label{fig:cor-mean-max}
\end{figure*}

Estos 9 casos fueron seleccionados específicamente para ilustrar estas
limitaciones y no son necesariamente representativos de la totalidad de
casos posibles. Como se muestra en la \autoref{fig:cor-mean-max}, la
amplitud media y la máxima tienen una relación lineal con un coeficiente
alto de correlación asociado (\(r^2>0.9\)). Debido a esto, a fines
estadísticos la elección de una u otra métrica no tendría, en promedio,
una influencia importante (aunque podría tenerla en casos individuales,
como lo ilustraron los casos previos).

A partir de esta discusión, se definió como índice de la QS3 (\(I_3\))
el promedio de la amplitud entre 65°S y 40°S y entre 700hPa y 100hPa.

El ciclo anual de \(I_3\) se muestra en la Figura~\ref{fig:ampl-ts1} a
través de boxplots de los valores mensuales individuales. Los valores
mayores de amplitud se dan en agosto y los mínimos en diciembre, y la
variabilidad sigue un ciclo similar. Este ciclo anual es consistente con
las climatologías previas (Loon y Jenne, Roy,
\protect\hyperlink{ref-Loon1972}{1972}; Karoly,
\protect\hyperlink{ref-Karoly1985}{1985}; Raphael,
\protect\hyperlink{ref-Raphael2004}{2004}). Se nota que \(I_3\) no es
mínimo en los meses de primavera, en contraste con lo observado en el
análisis climatológico (\autoref{fig:ampl-ncep}) y los campos
reconstruidos (\autoref{fig:qs3-ncep}). Esta aparente contradicción
surge, como se verá más adelante, en la falta de considerar también la
fase de la QS3 (\autoref{fase}).

\begin{figure*}
\subfloat[Boxplot para cada mes. Los puntos son los valores mensuales individuales.\label{fig:ampl-ts1}]{\includegraphics[width=155mm, ]{fig/tesis/ampl-ts-1} }\newline\subfloat[Serie temporal. Las líneas horizontales representan la media anual, en rojo (azul) cuando ésta es mayor (menor) que la media de todo el período marcada con línea punteada.\label{fig:ampl-ts2}]{\includegraphics[width=155mm, ]{fig/tesis/ampl-ts-2} }\caption{Índice $I_3$}\label{fig:ampl-ts}
\end{figure*}

La serie temporal de \(I_3\) se muestra en la Figura~\ref{fig:ampl-ts2}.
La serie exhibe tanto variaciones interanuales como subanuales de la
amplitud. Además, se observan series series de años con anomalías
positivas consistentemente seguidos por anoamlías negativas (1985-1990,
1992-1996 y 1999-2005) y otras con persistencia de anomalías positivas o
negativas (2005-2009 y 2012-2015). No hay evidencia visual de
periodicidades ni de una tendencia secular.

Un análisis de wavelets de la amplitud (previa eliminación del ciclo
anual) presenta picos alrededor de 3, 12 y 24 meses, aunque ninguno es
estadísticamente significativo (no se muestra). Raphael
(\protect\hyperlink{ref-Raphael2004}{2004}), también utilizando wavelets
en el perído 1958-2001, encontró que su índice de actividad de la QS3
mostraba variabilidad importante de entre 4 y 12 años, algo que no se
encontó en este estudio, quizás porque ambos índices capturan diferentes
características o porque los períodos analizados son distintos.

\section{Fase}\label{fase}

Resulta evidente que además de la amplitud, las ondas planetarias se
caracterizan por su fase. Los resultados de trabajos previos incluidos
en la Introducción (como por ejemplo los de Trenberth y Mo
(\protect\hyperlink{ref-Trenberth1985}{1985})) y así como los de las
secciones anteriores muestran que la QS3 puede presentar cambios
importantes en la fase.

\begin{landscape}\begin{figure}

{\centering \includegraphics[width=247mm,]{fig/tesis/fase-boxplot-1} 

}

\caption{Fase media para cada mes del año a partir de los 20 años con mayor amplitud y el rango definido por $\pm$ 1 desvío estándar  (puntos negros y barras negras). En rojo y azul se identifica respectivamente la localización del máximo y el mínimo de perturbación de Z para cada año individual}\label{fig:fase-boxplot}
\end{figure}
\end{landscape}

La fase de la QS3 se definió para cada mes como la fase promedio dentro
de la región de interés y de manera que represente la localización del
máximo de la perturbación de Z en la vecindad de Sudamérica (entre 120°E
y 0° de longitud). La \autoref{fig:fase-boxplot} muestra la fase media
de la QS3 de cada mes del año para los 20 años con mayores valores de
\(I_3\). Se incluye el rango delimitado por \(\pm\) 1 desvío estándar.
Además, se indica la localización de los centros de máxima y mínima
anomalía de geopotencial asociados con los 20 casos más extremos,
identificados a partir de de la amplitud media. El mapa se muestra para
referencia. Si bien la posición de los puntos en la dirección horizontal
tiene relación con la longitud, en el eje vertical no tiene relación con
la latitud y sí en cambio con los meses del año.

Se observa el ciclo anual en la fase ya se podía apreciar en la
\autoref{fig:qs3-ncep}. La fase media se centra en 55°O en enero y en
90°O en junio con ubicaciones intermedias en los meses de transición. El
continente queda al este del máximo de geopotencial en invierno lo que,
por balance geostrófico, implicaría anomalías positivas del viento
meridional, lo que favorecería los vientos del sur. En verano, en
cambio, se da la situación contraria. Esto confirma que el efecto de la
QS3 sobre Sudamérica depende crucialmente de su fase.

La variabilidad interanual de la fase para cada mes es considerable y de
una magnitud comparable a la amplitud del ciclo anual. En particular, es
notorio el aumento en la variabilidad de la fase en los meses de
primavera, al punto de que en noviembre la fase prácticamente no tiene
una posición predilecta.

La gran variabilidad presente durante los meses de primavera, en
comparación con el resto del año, explica por qué en los campos medios
la QS3 aparece débil a pesar de que su amplitud mensual no es pequeña.
Como se mencionó en la \autoref{metodologia}, al hacer el promedio, los
campos que están defasados en entre 1/3 y 2/3 de longitud de onda (entre
40° y 80° en el caso de la QS3) interfieren destructivamente entre
ellos, eliminando la señal en los campos medios. En primavera, más del
18\% de los meses tienen algún grado de interferencia destructiva con el
campo medio, comparado con el 8\% en verano.

Observando la distribución de los centros ciclónicos y anticiclónicos
(puntos azules y rojos, respectivamente) se nota que, a pesar de que en
promedio Sudamérica está afectada por centros anticiclónicos asociados a
la QS3, existe un número no despreciable de años en los que se observa
un centro ciclónico sobre el continente. En particular, sumando
noviembre y diciembre se pueden observar 9 años donde esto sucedió.

\section{Estaciones}\label{estaciones}

En las secciones anteriores se mostraron campos medios estacionales
utilizando la definición tradicional de las estaciones climatológicas
(verano = DEF, otoño = MAM, invierno = JJA, primavera = SON). Sin
embargo, como éstas son definidas a partir del ciclo anual de
temperatura en latitudes medias, no constituyen necesariamente el mejor
agrupamiento de los datos para otras variables u otras latitudes (por
ejemplo, la Antártida (King y Turner,
\protect\hyperlink{ref-King1997}{1997})).

Una metodología muy utilizada para la clasificación de campos es el
análisis de componentes principales (PCA). Como se explicó en la
\autoref{metodologia}, se aplicó esta metodología a los campos mensuales
de Z* reconstruidos a partir de la QS3 pesados por la raiz cuadrada del
coseno de la latitud.

La \autoref{eoftabla} muestra la varianza explicada de cada componente
principal obtenida a partir de los campos reconstruidos de QS3. Las
primeras dos componentes explican más del 80\% de la varianza y cada una
explica una parte similar de la varianza, indicando que se trata de
autovalores degenerados. Sabiendo, además, que los campos de QS3
prácticamente sólo tienen dos grados de libertad (amplitud y fase), la
elección de seleccionar las dos primera componentes es natural además de
justificada por la \autoref{eoftabla}.

\begin{longtable}[]{@{}lrrrrr@{}}
\caption{Varianza explicada por las 5 primeras componentes principales
de los campos de QS3 reconstruidos.\label{eoftabla}}\tabularnewline
\toprule
& PC 1 & PC 2 & PC 3 & PC 4 & PC 5\tabularnewline
\midrule
\endfirsthead
\toprule
& PC 1 & PC 2 & PC 3 & PC 4 & PC 5\tabularnewline
\midrule
\endhead
\(R^2\) & 0.436 & 0.397 & 0.054 & 0.038 & 0.026\tabularnewline
\bottomrule
\end{longtable}

\begin{figure*}
\includegraphics[width=155mm, ]{fig/tesis/eof-field-1} \caption{Primeras dos componentes principales del campo de Z* reconstruido a partir de la QS3.}\label{fig:eof-field}
\end{figure*}

Las dos primeras componentes principales del campo de QS3
(\autoref{fig:eof-field}) se encuentran en cuadratura cuya combinación
lineal resulta en otra QS3 cuya fase depende de la amplitud relativa de
cada componente. En verano predomina la PC1, mientras que en invierno
predomina la PC2 como se muestra en la \autoref{fig:pc1-pc2}.

Se observa que enero, febrero y marzo tienen preponderancia del PC1 y
casi nulo PC2 por lo que podrían agruparse. Abril, mayo, agosto,
septiembre y octubre tienen una mezcla similar de componentes, pero es
conveniente separar los dos primeros para respetar la progresión
temporal. Junio y julio tienen un comportamiento diferente al de los
demás meses con gran magnitud de PC2. Noviembre y diciembre aparecen
como \emph{outliers} en este diagrama debido a que su mayor variabilidad
(como se notó en la \autoref{fig:fase-boxplot}) hace que no predomine
ninguna componente principal. Podrían ser clasificarlos juntos como
meses de ``no estacionaridad'' indicando que se trata de una época del
año donde la QS3 no está presente.

En la \autoref{fig:pc1-pc2} también se hace una posible modificación de
las estaciones clásicas adaptada para el análisis de la QS3 basada en la
posición media de cada mes en el espacio de componentes principales.
Enero, febrero y marzo tienen preponderancia del PC1 y casi nulo PC2,
abril, mayo, agosto, septiembre y octubre tienen una mezcla similar de
componentes, pero es conveniente separar los dos primeros para respetar
la progresión temporal. Junio y julio no están tan juntos como los demás
meses, pero se los puede agrupar por tener gran magnitud de PC2.
Finalmente, noviembre y diciembre aparecen como \emph{outliers} en este
diagrama debido a que su mayor variabilidad (como se notó en la
\autoref{fase}) hace que no predomine ninguna componente principal. Es
posible clasificarlos juntos como meses de ``no estacionaridad''
indicando que se trata de una época del año donde la QS3 no está
presente.

\begin{figure*}
\includegraphics[width=155mm, ]{fig/tesis/pc1-pc2-1} \caption{Valor medio de las dos primeras componentes principales del campo de Z* reconstruido a partir de la QS3 para cada mes. Las líneas unen cada mes siguiendo el orden anual y los colores separan en las 5 "estaciones" definidas en el texto.}\label{fig:pc1-pc2}
\end{figure*}

El efecto de esta nueva descripción estacional se presenta en la
\autoref{fig:qs3-qsseason-ncep}, donde se muestra el campo de Z*
reconstruido sólo a partir de la QS3 en 300 hPa y en la
\autoref{qs3-qsseason-ncep} que muestra el corte vertical-zonal
correspondiente. Comparando con las figuras \ref{fig:qs3-season-ncep} y
\ref{qs3-season-ncep} del Capítulo~\ref{onda-3} se ve que los campos de
las estaciones de transición (AM y ASO) son más similares entre sí tanto
en el campo horizontal como en el corte meridional. Como era de
esperarse, el bimestre ND prácticamente carece de valores medios
significativos asociados con la QS3.

\begin{figure*}
\includegraphics[width=155mm, ]{fig/tesis/qs3-qsseason-ncep-1} \caption{Z* media reconstruida a partir de la QS3 en 300hPa según las estaciones definidas en el texto.}\label{fig:qs3-qsseason-ncep}
\end{figure*}

\begin{figure*}
\includegraphics[width=155mm, ]{fig/tesis/qs3-qsseason-ncep-corte-1} \caption{Corte en 52.5°S de la Z* media reconstruida a partir de la QS3 en 300hPa según las estaciones definidas en el texto.}\label{fig:qs3-qsseason-ncep-corte}
\end{figure*}

La inclinación vertical de las perturbaciones que en la
\autoref{fig:qs3-ncep-corte} aparecía sólo levemente en invierno se
observa con más claridad en la nueva descripción estacional
(\autoref{fig:qs3-season-ncep-corte}). En el bimestre JJ la inclinación
hacia el oeste es importante en la troposfera, estando presente también
en AM y ASO en menor medida.

La teoría de ondas de Rossby indica que una inclinación hacia el oeste
está asociada con transporte de calor hacia el polo por las
perturbaciones y propagación vertical de las mismas (James,
\protect\hyperlink{ref-James}{1994}). Esto podría explicar la variación
estacional de la extensión vertical de la QS3, la cual llegaría a la
estratósfera alta en JJ, cuando la inclinación es máxima y quedaría
atrapada en niveles más bajos en verano cuando la inclinación es mínima.

\begin{figure*}
\includegraphics[width=155mm, ]{fig/tesis/lag-cor-1} \caption{Correlación lageada para cada mes con los 12 sigientes.}\label{fig:lag-cor}
\end{figure*}

En complemento se presenta la \autoref{fig:lag-cor}, que muestra la
correlación lageada del campo de QS3 correspondiente a cada mes con los
demás. Es decir, el valor de enero con el febrero siguiente representa
la correlación entre los campos de QS3 de todos los eneros con los
campos de todos los febreros que les siguen. La línea escalonada marca
la separación del año de manera que un número a la izquierda de ésta
implica correlación de ese mes con meses del año siguiente. Las
correlaciones justo a la izquierda de la línea escalonada son positivas
y relativamente altas para casi todos los meses salvo noviembre,
diciembre y agosto. Esto implica que el comportamiento de la QS3 en
estos meses tiene poca similitud de un año a otro. Noviembre y diciembre
también presentan bajas correlaciones en general con los demás meses, lo
que es coherente con los resultados de las secciones anteriores, es
decir, al ser meses con actividad de la onda 3 poco estacionaria, sus
campos de QS3 no son consistentemente similares con ningún otro mes.
Esta interpretación no parece posible para agosto, ya que su
variabilidad no es particularmente alta (\autoref{fig:fase-boxplot}).

Los valores de un mes a la derecha de la línea escalonada son en
generalmente altos indicando buena concordancia entre los campos de un
mes y el siguiente. Para esto nuevamente las excepciones son noviembre,
diciembre y julio.

Las correlaciones entre meses corridos 6 meses son bajas para los meses
de verano e invierno y medias para los meses de transición. Es decir,
los meses de verano son muy distintos de los de invierno, mientras que
los de transición son medianamente parecidos a todos. Esta es una
consecuencia del ciclo anual de la fase (\autoref{fig:fase-boxplot}).

El uso de componentes principales para el análisis de una onda que
cambia de fase es similar a la metodología utilizada para el monitoreo
de la MJO (Wheeler y Hendon, \protect\hyperlink{ref-Wheeler2004}{2004})
por lo que sería posible su utilización como indicador de la actividad
de la QS3 distinto de la amplitud media de Fourier. Una desventaja de la
descomposición estacional obtenida es que no todas las estaciones tienen
la misma cantidad de meses, lo cual dificulta la comparación estadística
entre distintas estaciones. La exploración de dicho indicador está por
fuera del objetivo de este trabajo.

\section{R2}\label{r2}

En el \autoref{ondas-quasiestacionarias} se discutió la relevancia del
parámetro R2 como indicador de las características de la QS3. En
particular, se mostró que la estructura vertical de la varianza
explicada por la QS3 para cada estación (\autoref{fig:r2-ncep}) se
caracteriza por máxixmos valores cerca de 45°S entre la superifice y los
200hPa y un ciclo anual con máximo en invierno y mínimo en verano. En
esta sección se explora su estructura horizontal. Para esto, se toma
como \(r^2\) la correlación cuadrada entre el valor de Z* y el de QS3
correspondiente para cada punto de grilla y cada mes.

\begin{figure*}
\includegraphics[width=155mm, ]{fig/tesis/cor-campo-1} \caption{Valores de Correlación cuadrada media entre Z* en 300hPa y QS3 (sombrados). Los contornos indican los centros de máximas anomalías positivas (rojo) y negativas (azul)}\label{fig:cor-campo}
\end{figure*}

Los campos horizontales de \(r^2\) para 300hPa se muestran en la
\autoref{fig:cor-campo}. En las cuatro estaciones la QS3 explica la
mayor parte de la varianza en el hemiferio oeste entre 60°S y 45°S.
Lejos de ser homogéneos, los campos muestran tres máximos localizados
con cierta persistencia durante el año. El primero, al sur del Índico,
está presente en verano y otoño en 60°E que se encuentra más hacia el
este en invierno y primavera. Algo similar sucede con el segundo máximo
ubicado al sur del Pacífico, que se encuentra en 180° en otoño y en 120°
en primavera. Se observa un tercer máximo en el Atlántico sur cuya
ubicación varía longitudinalmente poco. En verano y otoño se distingue
un máximo en latitudes bajas en el pacífico central.

Si se compara la posición de los máximos de \(r^2\) con los centros de
QS3, en verano, el máximo del Índico, por ejemplo, coincide con un
centro anticiclónico. Pero en otoño el máximo de \(r^2\) sobre esa
cuenca se encuentra entre dos centros de la onda y lo mismo pasa con el
máximo del Atlántico. Éste último coincide con un centro anticiclónico,
aunque en invierno está más cerca de uno ciclónico. En suma, no parece
haber una asociación entre los máximos de \(r^2\) y la ubicación de los
centros de la QS3.

Se realizó el mismo análisis en base a las estaciones definidas en la
\autoref{estaciones} según las características de la QS3 y se confirmó
que las características generales no se modifican.Las principales
diferencias son un debilitamiento de los máximos de R2 durante EFM y una
fuerte intensificación del máximo del Atlántico durante DN (no se
muestra).

\section{Regresiones}\label{regresiones}

Los campos de QS3 reconstruida a partir de fourier permiten conocer la
forma idealizada del campo de geopotencial asociado a esa onda, pero por
su naturaleza no permiten conocer el estado típico de la atmósfera
cuando la QS3 está activa. En esta sección se exploran brevemente las
anomalías típicas de diferentes variables explicadas por la actividad de
la QS3, a través del cómputo de mapas de regresión lineal entre \(I_3\)
(\autoref{amplitud}) y diferentes variables. En esta metodología, la
serie de \(I_3\) está estandarizada de manera que los valores de
regresión representan la porción de anomalía de la variable en cuestión
asociado con un cambio de la amplitud de la QS3 de 1 desvío estándar en
magnitud. En ese sentido los valores de regresión tienen las unidades de
la variable regresionada. Para facilitar la descripción, la misma se
realiza asumiendo que se muestran las anomalías regresionadas
correspondientes a un cambio positivo en la amplitud de la QS3 de 1
desvío estándar.

\subsection{Geopotencial}\label{geopotencial}

\begin{landscape}\begin{figure}

{\centering \includegraphics[width=247mm,]{fig/tesis/regr-gh-ncep-1} 

}

\caption{Regresión de Z en 300hPa con $I_3$.}\label{fig:regr-gh-ncep}
\end{figure}
\end{landscape}

En la \autoref{fig:regr-gh-ncep} se muestra la regresión del campo de Z
en 300hPa con \(I_3\) para los doce meses del año. Existe mucha
heterogeneidad en las características observadas en los distintos meses.

En enero, se observa un centro anómalamente negativo importante al sur
del Pacífico, embebido en un patrón hemisférico de onda 3, en el cual se
distingue un tren de ondas extendido entre el este de Nueva Zelanda, y
Sudamérica. En febrero, la estructura se mantiene similar, pero con
menor propagación meridional, y el centro anticiclónico en Sudamérica
intensificado y un debilitamiento del ubicado al sur del Índico. En
marzo, está presente una QS3 con mínima variación meridional que ocupa
todo el círculo de latitud entre 60°S y 45°S. Abril es similar a marzo,
pero sin centros ciclónicos significativos. En mayo, se distingue
nuevamente un tren de ondas con propagación meridional pero en vez de
terminar al sur de Sudamérica, lo hace en el Mar de Weddell. Junio, al
igual que marzo, presenta una estructura de QS3 zonal, pero menos
definida y, consistente con el corrimiento de la fase observado en la
\autoref{fase} (\autoref{fig:fase-boxplot}), con los centros ciclónicos
y anticiclónicos desplazados hacia el oeste.

En julio, el patroń de QS3 se encuentra superpuesto con un fuerte patrón
anular, con anomalías positiva de geopotencial en latitudes polares y
negativas en latitudes medias. En la \autoref{fig:regr-gh-polar} se
muestran los mismos campos para julio y diciembre que en la
\autoref{fig:regr-gh-ncep} pero en proyección estereográfica polar. En
esta proyección es fácil identificar un patrón anular similar en julio a
una fase negativa del SAM y a una fase positiva del SAM en diciembre. En
efecto, la correlación entre el índice SAM e \(I_3\) en julio es-0.57
(p-valor \(\sim 9\times 10^{-4}\)) y en diciembre es 0.41 (p-valor
\(\sim 0.0225\))\footnote{Correlación de Pearson, tests a dos colas}.
Estos dos meses son los únicos con una relación significativa con el SAM
al nivel del 95\% de confianza.

\begin{figure*}
\includegraphics[width=155mm, ]{fig/tesis/regr-gh-polar-1} \caption{Igual que \autoref{fig:regr-gh-ncep}, pero en coordenadas estereográficas polares para julio y septiembre.}\label{fig:regr-gh-polar}
\end{figure*}

La \autoref{fig:regr-gh-ncep} muestra que en agosto el campo de
regresión tiene una estructura de QS3 zonal salvo en la región de
Sudamérica, donde el centro anticiclónico se encuentra a mayor latitud,
sobre la Península Antártica. En septiembre se observa, embebido sobre
el patrón de onda 3, un tren de ondas sobre el Pacífico con importante
propagación meridional, similar al de otros meses (como enero, por
ejemplo). En octubre, el patrón de QS3 se ve claramente en los centros
de anomalías positivas, pero no tanto en los negativos. En este mes, hay
evidencias de propagación meridional de trenes de onda. En particular se
distingue un tren de ondas extendido entre el Pacífico central y el
Atlántico, con fase tal que exhibe un centro negativo sobre Sudamérica.
El campo de regresión en noviembre es muy similar al de abril; con
evidencias de un tren de ondas extendido entre el sur de Australia y
Sudamérica. En diciembre el patrón es similar al de enero pero
modificado por el patrón anular similar a la fase positiva del SAM,
discutido previamente.

\subsection{Función Corriente}\label{funcion-corriente-1}

\begin{landscape}\begin{figure}

{\centering \includegraphics[width=247mm,]{fig/tesis/regr-psi-ncep-1} 

}

\caption{Regresión de la anomalía zonal de $\psi$ en $\sigma = 0.2101$ con $I_3$ y flujos de actividad de onda calculados a partir de la misma.}\label{fig:regr-psi-ncep}
\end{figure}
\end{landscape}

Como se mostró previamente la función corriente (\(\psi\)) resulta una
mejor variable que Z para describir anomalías de circulación tanto sobre
latitudes ecuatoriales como polares. Asimismo, los mapas de regresión de
Z explicadas por la QS3 muestran en varios meses del año evidencias de
trenes de onda extendidos entre zonas tropicales y extratropicales. En
este sentido, esta subsección describe las características de la
regresión entre las anomalías zonales de \(\psi\) con \(I_3\). Se
incluye además los flujos de actividad de onda calculados a partir de
tal regresión. El estudio de estos flujos permite confirmar (o no) que
una determinada alternancia de centros de diferente signo está
relacionada (o nó) con un tren de onda.

La \autoref{fig:regr-psi-ncep} muestra evidencias de propagación
meridional de ondas de Rossby desde la zona ecuatorial y atravesando el
Pacífico Sur hasta Sudamérica en varios meses del año (enero, febrero,
mayo, junio, agosto, septiembre, octubre y diciembre) aunque con
diferentes características.

En enero y febrero, trenes de onda parecen emanar desde la zona
ecuatorial ubicada al noreste de Nueva Zelanda, aunque con diferente
fase y reflejan a diferentes longitudes. En mayo, se observa un tren
intenso desde el norte de Australia hasta hacia el sur de Sudamérica. En
junio la actividad de onda que emana desde Australia lo hace
principalmente a lo largo de los subtrópicos para luego, en el Pacífico
sudeste, dispersarse hacia el sur e influenciar el sur de Sudamérica. En
agosto y septiembre, la actividad de onda que influye el sur de
Sudamérica emana principalmente de la vecindad de Nueva Zelanda. En
octubre, domina el flujo de actividad de onda que emana desde el norte
de Nueva Zelanda hacia el Pacífico ecuatorial central para luego
dispersar hacia el sudeste hasta Sudamérica y el Atlántico Sur. En
noviembre, se distingue una dispersión de energía desde el sur de
Australia, hacia el sur, que se observa más fuerte y coherente en
diciembre influenciando Sudamérica.

Cabe mencionar que se hicieron también análisis de regresión entre las
anomalías de SST e \(I_3\) con el fin de explorar en qué medida los
trenes de onda descriptos previamente están relacionados con variaciones
en la temperatura de superficie del océano, en especial de los océanos
tropicales, pero no se encontraron relaciones significativas. Teniendo
en cuenta las limitaciones que tiene la descripción de las
características de la QS3 solo a partir de la amplitud, y también
considerando la imposibilidad de los análisis con observaciones de poder
separar limpiamente los factores asociados con todas las posibles
causas, motivó los experimentos con el modelo Speedy que se discuten en
el capítulo siguiente.

\chapter{Simulaciones con el modelo
SPEEDY}\label{simulaciones-con-el-modelo-speedy}

El model SPEEDY fue utilizado para explorar las posibles causas que
expliquen el desarrollo de un patrón de onda 3 en la circulación del
hemisferio sur, más allá de las causas relacionadas con las vacilaciones
del flujo de los oestes. En este capítulo se presentan y discuten los
resultados de la corrida control así como de los experimentos de
sensibilidad realizados.

\section{Validación}\label{validacion}

Se realizó primero una corrida Control utilizando el modelo SPEEDY (de
aquí en adelante SPEEDY). La misma cubrió el período 1985-2014, con los
modelo de suelo y hielo activados y SST históricas provenientes de la
base de datos HadISST (Rayner,
\protect\hyperlink{ref-Rayner2003}{2003}). Se utilizó una atmósfera en
reposo como condición inicial observando que el tiempo de \emph{spin-up}
es menor que un mes, por lo que se utilizó todos los meses de
simulación.

Para la validación de la corrida control de SPEEDY se utilizaron los
reanálisis del NCEP/NCAR (de aquí en adelante NCEP) correspondientes a
diferentes variables. La descripción se concentra en mostrar los
resultados sobre los niveles de 200hPa y 500hPa como representativos de
los niveles altos y medios respectivamente, de la troposfera. Las
conclusiones no cambian substancialmente en el resto de los niveles.

Como se mencionó previamente (\autoref{datos-y-modelo}), la
representación de la estratósfera en SPEEDY es muy limitada debido a que
el único nivel estratosférico disponible (30hPa) es la tapa del modelo,
funcionando como ``esponja'' que evita la propagación de ondas de
gravedad.

\subsection{Altura Geopotencial}\label{altura-geopotencial-1}

\begin{landscape}\begin{figure}

{\centering \includegraphics[width=247mm,]{fig/tesis/ghz-sp-nc-1} 

}

\caption{Z* (SPEEDY sombreado, NCEP contornos).}\label{fig:ghz-sp-nc}
\end{figure}
\end{landscape}

La validación del campo medio de altura geopotencial (no se muestra)
confirma que está bien representado por el modelo. La comparación de las
anomalías zonales del campo medio representadas por SPEEDY se muestra en
la \autoref{fig:ghz-sp-nc} donde en sombreado se muestra el campo de
SPEEDY y en contornos el correspondiente a NCEP (convención que se
mantendrá en el resto de las figuras de validación). SPEEDY representa
correctamente la estructura aproximadamente barotrópica equivalente de
las anomalías zonales. El patrón de QS1 y su intensificación en invierno
son también bien simulados, aunque no en magnitud. En verano es
demasiado débil y en invierno y primavera, demasiado intenso.

La ubicación de los máximos y mínimos de Z* en SPEEDY es aproximadamente
la correcta, aunque no logra capturar parte de la estructura fina. En
500hPa durante otoño, invierno y primavera, SPEEDY presenta un sólo
máximo en 120°O a pesar de que NCEP muestra dos máximos, uno que se
mantiene en la línea de fecha durante todo el año y otro entre 120°O y
60°O según la época del año.

\begin{landscape}\begin{figure}

{\centering \includegraphics[width=247mm,]{fig/tesis/ghz-dif-sp-nc-1} 

}

\caption{Diferencia de Z* entre SPEEDY y NCEP}\label{fig:ghz-dif-sp-nc}
\end{figure}
\end{landscape}

En la \autoref{fig:ghz-dif-sp-nc} se muestra la diferencia entre el
campo de Z* de NCEP y SPEEDY. En verano y otoño la principal diferencia
radica en que NCEP muestra una alta más intensa de la QS1 al norte de
60°S. En invierno y primavera se observa un tren de ondas con
propagación medirional que une el Índico con el Atlántico con número de
onda planetaria 3. Un tren de ondas similar fue identificado en las
observaciones en la \autoref{sec:viento-meridional}. Su aparición al
hacer la resta NCEP - SPEEDY indica que el mismo no está presente en la
corrida Control.

\begin{figure*}
\includegraphics[width=155mm, ]{fig/tesis/ghz-sp-nc-corte60-1} \caption{Corte zonal de Z* en 60°S (SPEEDY sombreado, NCEP contornos).}\label{fig:ghz-sp-nc-corte60}
\end{figure*}

En la \autoref{fig:ghz-sp-nc-corte60} se muestra el corte de Z* en 60°S.
El modelo presenta en líneas generales las anomalías de un determinado
signo en la misma porción del hemisferio que NCEP. Además, el modelo es
capaz de reproducir crudamente la variación estacional de la ubicación
de los extremos de las anomalías en la vertical, entre verano e
invierno. Sin embargo, se evidencia la falta de inclinación en la
vertical de las anomalías zonales de SPEEDY, las cuales son mucho más
barotrópicas que en NCEP. Esto indica que la QS1 en SPEEDY no estáría
asociada principalmente al transporte de calor hacia el polo ni, en
consecuencia, a la propagación vertical de ondas de Rossby.

\subsection{Temperatura}\label{temperatura-1}

\begin{landscape}\begin{figure}

{\centering \includegraphics[width=247mm,]{fig/tesis/t-nc-sp-1} 

}

\caption{Campo medio de temperatura (SPEEDY sombreado, NCEP contornos).}\label{fig:t-nc-sp}
\end{figure}
\end{landscape}

En la \autoref{fig:t-nc-sp} se muestra el campo medio de temperatura de
SPEEDY y NCEP. En 850hPa y 500hPa, ambos campos son muy similares, tanto
en el gradiente meridional como en las anomalas zonales (que se muestran
mejor en la \autoref{fig:tz-nc-sp}). En niveles altos (200hPa) las
simulación control diverge considerablemente de las observaciones. En
verano y en otoño, SPEEDY muestra un gradiente meridional mucho más
importante que NCEP y en invierno y primavera el gradiente máximo se da
entre 30°S y 45°S para SPEEDY, y en 60°S en NCEP.

La \autoref{fig:tz-sp-nc} presenta las anomalías zonales de temperatura
simuladas y observadas. Ambas coinciden en niveles bajos, donde la
influencia superficial es importante, pero son diferentes en altura. En
500hPa, las anomalías Antárticas de SPEEDY coinciden aproximadamente en
ubicación con las de NCEP aunque son ligeramente más débiles en invierno
y primavera. En 60°S, en verano hay buena coincidencia, pero entre otoño
e invierno la QS1 observada en esas latitudes virtualmente desaparece en
SPEEDY mientras que en primavera vuelve a crecer incluso con mayor
intensidad que en NCEP. En 200hPa, SPEEDY carece casi totalmente de
anomalías zonales significativas al sur de los 45°S durante todo el año
a diferencia de NCEP, que muestra una estructura de QS1 bien definida
con máxima amplitud en primavera. Al norte de esa latitud las anomalías
de SPEEDY coinciden mejor con NCEP.

\begin{landscape}\begin{figure}

{\centering \includegraphics[width=247mm,]{fig/tesis/tz-sp-nc-1} 

}

\caption{Anomalía zonal de temperatura (SPEEDY sombreado, NCEP contornos)}\label{fig:tz-sp-nc}
\end{figure}
\end{landscape}

\subsection{Viento zonal}\label{viento-zonal-1}

\begin{figure*}
\includegraphics[width=155mm, ]{fig/tesis/u-sp-nc-corte-1} \caption{Media zonal del viento zonal (SPEEDY sombreado, NCEP contornos).}\label{fig:u-sp-nc-corte}
\end{figure*}

En la \autoref{fig:u-sp-nc-corte} se muestra el viento zonal medio para
SPEEDY y NCEP. El modelo es capaz de representar el menos crudamente,
varias de las características observadas más importantes, como ser la
existencia de regiones de vientos del oeste intensos en regiones
subtropicales y subpolares, así como sus variaciones estacionales. Sin
embargo el jet simulado en verano se encuentra más al norte y
ligeramente más elevado, así como los jets subtropicales simulados en
las latitudes subtropicales son considerablemente más intensos en todas
las estaciones. Se especula que esto podría deberse a una celda de
Hadley simulada más intensificada y con menos variaciones estacionales.
Se destaca además la ausencia de un máximo asociado al jet subpolar, lo
cual podría deberse a la falta de suficientes niveles verticales en la
porción superior.

Dada la importancia del jet polar en la dinámica atmosférica durante los
meses de invierno y primavera, su mala representación es una limitación
muy importante del modelo SPEEDY.

\subsection{Gradiente meridional de vorticidad
absoluta}\label{gradiente-meridional-de-vorticidad-absoluta}

\begin{landscape}\begin{figure}

{\centering \includegraphics[width=247mm,]{fig/tesis/etady-sp-nc-1} 

}

\caption{Gradiente meridional de vorticidad absoluta ($1\times10^11(ms)^{-1}$) (SPEEDY).}\label{fig:etady-sp-nc}
\end{figure}
\end{landscape}

Comparando con la figura \autoref{fig:etady-ncep} (nivel de 200hPa), se
ve que la franja de máximo gradiente presente en todas las estaciones es
más zonal en el caso de SPEEDY y corrida hacia el sur en verano. La
región de gradientes negativos que se desarrolla en invierno sobre Nueva
Zelanda tiene menor extensión y no aparece en otoño ni primavera.

En la \autoref{fig:ks-sp-nc-corte} se muestra un corte meridional del
número de onda estacionario en 180°. En verano no hay regiones
prohibidas en ningún modelo, pero la región entre 45°S y 60°S aparece
como un mínimo en SPEEDY y un máximo en NCEP. Esto implica que la
propagación meridional inhibida para un amplio rango de números de onda
en SPEEDY pero no en NCEP. En particular, la QS3 puede propagarse
meridionalmente en las observaciones, pero no en el modelo. En otoño la
situación es inversa: al rededor de 40°S, NCEP muestra una región con
número de onda estacionario imaginario, impidiendo la propagación
meridional, mientras que en SPEEDY no existe tan impedimento. La QS3
nuevamente se ve afectada, teniendo propagación meridional irrestricta
al norte de 60°S en SPEEDY pero quedando atrapada al sur de 50°S en las
observaciones. En invierno y primavera la cooncordancia entre SPEEDY y
NCEP es mayor.

\begin{figure*}
\includegraphics[width=155mm, ]{fig/tesis/ks-sp-nc-corte-1} \caption{Número de onda estacionario en 200hPa en 180°O.}\label{fig:ks-sp-nc-corte}
\end{figure*}

\subsection{Función corriente}\label{funcion-corriente-2}

\begin{landscape}\begin{figure}

{\centering \includegraphics[width=247mm,]{fig/tesis/psi-sp-1} 

}

\caption{Función corriente media en 200hPa (contornos cada $2\times10^{-11}m^2/s$), anomalía zonal de función corriente (sombreado,  $1\times10^{-9}m^2/s$) y flujos de actividad de onda medios.}\label{fig:psi-sp}
\end{figure}
\end{landscape}

La función corriente en 200hPa de SPEEDY se muestra en la
\autoref{fig:psi-sp} (donde el sombreado corresponde a la anomalía zonal
y las felchas a los flujos de actividad de onda) en comparación con los
mismos campos de NCEP (\autoref{fig:psi-ncep}; notar que en NCEP está en
coordenadas \(\sigma\)) existe una correspondencia general buena en los
trópicos. La localización de los máximos y mínimos coincide en
aproximadamente en todas las estaciones. En verano y primavera no
aparece la alta de Bolivia relacionada con el SAMS, pero sí la baja del
noroeste. La intensidad de las anomalías es menor en todas las
estaciones, especialmente en el HN. Consecuentemente, también tienen
menor magnitud los flujos de actividad de onda

\subsection{Onda 3}\label{onda-3-1}

\begin{figure*}
\includegraphics[width=155mm, ]{fig/tesis/ampl-sp-nc-1} \caption{Amplitud de la QS3 a partir de Fourier (SPEEDY sombreado, NCEP contornos).}\label{fig:ampl-sp-nc}
\end{figure*}

En la \autoref{fig:ampl-sp-nc} se muestra la amplitud de la QS3 media
para SPEEDY en contornos y NCEP en líneas. En verano, entre 45°S y 60°S
SPEEDY coincide con NCEP en la localización y extension del máximo, pero
subestima la intensidad. El máximo secundario en latitudes bajas aparece
más al sur y en un nivel más alto en SPEEDY. En otoño, speedy carece
casi por completo de señal de QS3 en comparación con NCEP. Si se calcula
la amplitud media de la QS3 SPEEDY sí tiene una señal importante (no se
muestra), indicando que la diferencia con NCEP se debe no a la falta de
una onad 3, sino a su característica no estacionaria. En invierno, la
amplitud de la QS3 media está subestimada en SPEEDY y corrida hacia el
polo. Además, está mucho más restringida a niveles troposféricos en
comparación con NCEP; posiblemente como consecuencia de la falta de
niveles verticales y la mala representación del jet polar. En primavera,
por el contrario, la señal en SPEEDY es considerablemente mayor que en
NCEP.

La QS3 reconstruida para SPEEDY se muestra en la \autoref{fig:qs3-sp-nc}
donde además de confirmarse la variación de la amplitud prevista a
partir de la \autoref{ampl-sp-nc}, se puede apreciar la estructura
horizontal. En verano, invierno y privamera, la inclinación de los
centros es hacia el este, contraria a las observaciones e indicando
transporte de cantidad de movimiento zonal hacia el norte en vez de
hacia el sur. En otoño el campo de QS3 de SPEEDY es virtualmente nulo.

\begin{figure*}
\includegraphics[width=155mm, ]{fig/tesis/qs3-sp-nc-1} \caption{Z* reconstruida a partir de la QS3 (SPEEDY sombreado, NCEP contornos).}\label{fig:qs3-sp-nc}
\end{figure*}

\section{Experimentos de
sensibilidad}\label{experimentos-de-sensibilidad}

Además de la corrida Control, se realizaron 3 corridas para evaluar la
sensibilidad de la QS3 a distintos parámetros del modelo.
\textbf{NOLAND} es idéndica a \textbf{Control} a excepción de que se
desactivaron los modelos de mar, hielo y tierra y se reemplazó la SST
por su media climatológica mensual. \textbf{SSTZONAL} es similar, pero
la SST fue reemplazada por su media zonal mensual.

\subsection{Altura geopotencial}\label{altura-geopotencial-2}

Los campos de Z* en 200hPa para cada corrida se muestra en la
\autoref{fig:ghz-sp-runs}. La corrida NOLAND no presenta diferencias
obvias con respecto a Control. SSTZONAL en cambio, muestra una clara
disminución de las anomalías zonales. Todas las corridas presentan una
QS1 en altas latitudes con máximo en invierno pero su intensidad es
notablemente menor en SSTZONAL en comparación a las otras dos. La QS1 en
latitudes polares, en cambio, mantiene su amplitud en todas las
corridas.

\begin{landscape}\begin{figure}

{\centering \includegraphics[width=247mm,]{fig/tesis/ghz-sp-runs-1} 

}

\caption{Z* en 200hPa para cada corrida de SPEEDY}\label{fig:ghz-sp-runs}
\end{figure}
\end{landscape}

La figura \autoref{fig:ghz-dif-sp-runs} muestra la diferencia entre Z*
de la corrida Control y las corridas de sensibilidad. NOLAND sólo
muestra cambios significativos en invierno y primavera, donde la
diferencia es un tren de ondas de propagación meridional muy similar al
observado en la \autoref{fig:ghz-dif-sp-nc} y presente en las
observaciones (\autoref{sec:viento-meridional}). Es decir que a pesar de
que la corrida Control carece del mismo, éste se desarrolla
correctamente si se elimina la variabilidad de la temperatura
superficial y su interacción con la atmósfera.

La diferencia entre SSTZONAL y Control se da principalmente entre 45°S y
60°S y responde a la casi desaparición de la QS1 en esas latitudes.
Estos resultados son consistentes con los de Quintanar y Mechoso
(\protect\hyperlink{ref-Quintanar1995}{1995}\protect\hyperlink{ref-Quintanar1995}{b})
dado que la eliminación de las anomalías zonales reduciría la convección
anómala que genera los trenes de onda que ellos concluyeron son el
principal sostén de este patrón. En invierno y primavera, está presente
el tren de ondas observado en NOLAND - Control, pero con menor amplitud.
Esto quizás se deba a que éste se encuentra enmascarado por los otros
cambios.

\begin{landscape}\begin{figure}

{\centering \includegraphics[width=247mm,]{fig/tesis/ghz-dif-sp-runs-1} 

}

\caption{Diferencia de Z* en 200hPa entre cada corrida y la corrida Control.}\label{fig:ghz-dif-sp-runs}
\end{figure}
\end{landscape}

\subsection{Viento zonal}\label{viento-zonal-2}

\begin{landscape}\begin{figure}

{\centering \includegraphics[width=247mm,]{fig/tesis/u-dif-sp-runs-1} 

}

\caption{Diferencia de viento zonal en 200hPa entre cada corrida y la corrida Control.}\label{fig:u-dif-sp-runs}
\end{figure}
\end{landscape}

La diferencia del viento meridional entre corridas se muestra en la
\autoref{fig:u-dif-sp-runs}. NOLAND - Control casi nulas diferencias en
otoño. En verano NOLAND tiene vientos más inensos al norte y sur de
Nueva Zelanda y menos intensos sobre ésta región, indicando una
intensificación de los bloqueos con respeto a la corrida Control; el
mismo patrón se observa en primavera. En invierno y primavera hay
valores positivos en latitudes bajas indicando un debilitamiento de los
alisios.

En SSTZONAL el patrón de aumento de los bloqueos se ve intensificado. En
otoño, invierno y primavera se observan franjas de valores negativos al
sur de 30° y positivos al norte, indicando que el jet se encuentra
desplazado hacia el norte. En las regiones ecuatoriales, los alisios son
más intensos durante todo el año, lo cual resulta paradójico dada la
estructura tipo Niño observada en la \autoref{fig:tzdif-sp-runs}.

\subsection{Función corriente}\label{funcion-corriente-3}

\begin{landscape}\begin{figure}

{\centering \includegraphics[width=247mm,]{fig/tesis/psi-sp-runs-1} 

}

\caption{Función corriente media en 200hPa (contornos cada $2\times10^{-11}m^2/s$), anomalía zonal de función corriente (sombreado,  $1\times10^{-9}m^2/s$) y flujos de actividad de onda medios para cada corrida.}\label{fig:psi-sp-runs}
\end{figure}
\end{landscape}

En la \autoref{fig:psi-sp-runs} se muestra la anomalía zonal de la
función corriente para cada corrida y los flujos de acción de onda. La
misma es consistente con las observaciones generales de la
\autoref{fig:ghz-sp-runs}. Para todos los meses las anomalías en
SSTZONAL son muy menores que las otras dos corridas y los flujos de
actividad de onda son casi nulos en el HS y menores que las otras dos
corridas en los trópicos del HN. La persistencia de las anomalías en el
HN en comparación con el HS sugiere que el forzante principal de estos
patrones de onda es el orográfico, ya que éste es el único forzante
zonalmente asimétrico en SSTZONAL.

Para la corrida NOLAND, las diferencias son mínimas en verano y otoño.
En invierno y primavera se observan anomalías más intensas al rededor de
30° y los flujos de acción de onda que son más hacia el sur al rededor
de 60°E en comparación a la corrida Control. Nuevamente, esto es
consistente con lo visto en la \autoref{fig:ghz-dif-sp-runs} donde en
NOLAND aparecía el tren de ondas desde el Índico hacia el pacífico que
estaba ausente en la corrida Control.

\subsection{Onda 3}\label{onda-3-2}

\begin{landscape}\begin{figure}

{\centering \includegraphics[width=247mm,]{fig/tesis/ampl-sp-runs-1} 

}

\caption{Amplitud de la QS3 para cada corrida según Fourier y método AM.}\label{fig:ampl-sp-runs}
\end{figure}
\end{landscape}

En la \autoref{fig:ampl-sp-runs} se muestra la amplitud de la QS3 para
cada estación y cada corrida. En verano se observa una reducción de la
señal de la QS3 en 60°S entre Control, NOLAND y SSTZONAL además de un
ligero corrimiento hacia el sur. La señal de más al norte, en cambio, no
presenta cambios importantes más allá del mismo corrimiento que la señal
anterior. En otoño la poca señal existente en Control prácticamente
desaparece en en SSTZONAl, pero no hay cambios en NOLAND. En SSTZONAL
hay un aumento de la amplitud de la QS3 en la costa antártica.

En invierno, NOLAND tiene la mayor señal de QS3 entre las simulaciones y
Control, la mínima. Esto posiblemente se deba a la aparición del tren de
ondas desde el Índico identificado en la \autoref{fig:ghz-dif-sp-runs}.
Primavera es la estación con el cambio más dramático. Pasa de tener la
señal más alta en la Corrrida control a tener una de las menores en
NOLAND y SSTZONAL. La amplitud de la QS3 en 60°S y 300hPa desaparece
prácticamente en su totalidad

Es importante notar que estas observaciones son sensibles a la
metodología utilizada. La \autoref{fig:ampl-mean-sp-runs} muestra la
amplitud media de la QS3 para cada corrida y es útil compararla con la
\autoref{fig:ampl-sp-runs} (que muestra la amplitud de la QS3 media).
Verano se comporta de manera similar, con una ligera disminución de la
señal en NOLAND y SSTZONAL.

Otoño tiene un gran cambio. Tanto la corrida Control como NOLAND y
SSTZONAL tienen una señal fuerte que no cambia significativamente entre
corridas. Esto indica que la baja señal en otoño en SPEEDY tiene es
causada por que la onda 3 está presente mes a mes, pero no
estacionariamente. Lo mismo parece suceder invierno de la corrida
Control y primavera de NOLAND y SSTZONAL.

\begin{landscape}\begin{figure}

{\centering \includegraphics[width=247mm,]{fig/tesis/ampl-mean-sp-runs-1} 

}

\caption{Amplitud de la QS3 para cada corrida según Fourier y método MA.}\label{fig:ampl-mean-sp-runs}
\end{figure}
\end{landscape}

\begin{figure*}
\includegraphics[width=155mm, ]{fig/tesis/sd-fase-sp-runs-1} \caption{Desvío estándar (en grados) de la fase media mensual para cada estación y cada corrida.}\label{fig:sd-fase-sp-runs}
\end{figure*}

Para profundizar en esta observación, la \autoref{fig:sd-fase-sp-runs}
muestra el desvío estándar de la fase media mensual para cada estación y
corrida. Si se asume distribución normal, aproximadamente el 95\% de los
datos están en un rango de \(\pm 2\sigma\) al rededor de la media. Como
la fase está acotada entre 0° y 120°, valores de \(\sigma\) por encima
de 30° implica que los datos están distribuidos casi uniformemente en
todo el dominio.

En verano, el desvío es mínimo y aumenta ligeramente en NOLAND y
SSTZONAL pero siempre por debajo de los 30°. En otoño, todas las
corridas muestran valores altos muy por encima de los 30°, indicando que
hay muy poca estacionaridad en todas las corridas. En invierno, la
corrida Control está en el límite de los 30° y NOLAND y SSTZONAL tienen
valores menores. En primavera, la corrida Control tiene valores mínimos
de \(\sigma\) mientras que NOLAND y SSTZONAL presentan valores muy
altos, por encima de los 30°. Estos valores son consistentes con lo
observado en la comparación de las figuras~\ref{fig:ampl-dif-sp-runs}
y~\ref{fig: ampl-mean-sp-runs} y sugieren que las diferencias observadas
entre las corridas en la señal de la QS3 se deben principalmente a mayor
o menor estacionaridad y no a una mayor o menor amplitud de las onda
planetaria 3.

\chapter{Conclusiones}\label{conclusiones}

\chapter*{Referencias}\label{referencias}
\addcontentsline{toc}{chapter}{Referencias}

\hypertarget{refs}{}
\hypertarget{ref-Alvarez2014}{}
Alvarez, M.S., Vera, C.S., Kiladis, G.N., Liebmann, B., 2014.
Intraseasonal variability in South America during the cold season. 42,
11-12, 3253-3269.

\hypertarget{ref-Barreiro2014}{}
Barreiro, M., Díaz, N., Renom, M., 2014. Role of the global oceans and
land-atmosphere interaction on summertime interdecadal variability over
northern Argentina. 42, 7-8, 1733-1753.

\hypertarget{ref-Berbery1992}{}
Berbery, E.H., Nogués-Paegle, J., Horel, J.D., 1992. Wavelike southern
hemisphere extratropical teleconnections. 49, 2, 155-177.

\hypertarget{ref-Cai1999}{}
Cai, W., Baines, P.G., Gordon, H.B., 1999. Southern mid- to
high-latitude variability, a zonal wavenumber-3 pattern, and the
Antarctic circumpolar wave in the CSIRO coupled model. 12, 10,
3087-3104.

\hypertarget{ref-R-metR}{}
Campitelli, E., 2017. metR: Tools for Easier Analysis of Meteorological
Fields. \url{https://github.com/eliocamp/metR}

\hypertarget{ref-Court1942}{}
Court, A., 1942. Tropopause Disappearance During the Antarctic Winter.
23, 5, 220-238.

\hypertarget{ref-Desrochers1999}{}
Desrochers, P.R., Yee, S.Y.K., 1999. Wavelet Applications for
Mesocyclone Identification in Doppler Radar Observations. 38, 7,
965-980.

\hypertarget{ref-Fogt2012}{}
Fogt, R.L., Jones, J.M., Renwick, J., 2012. Seasonal zonal asymmetries
in the southern annular mode and their impact on regional temperature
anomalies. 25, 18, 6253-6270.

\hypertarget{ref-Hobbs2010}{}
Hobbs, W.R., Raphael, M.N., 2010. Characterizing the zonally asymmetric
component of the SH circulation. 35, 5, 859-873.

\hypertarget{ref-Irving2015}{}
Irving, D., Simmonds, I., 2015. A novel approach to diagnosing Southern
Hemisphere planetary wave activity and its influence on regional climate
variability. 28, 23, 9041-9057.

\hypertarget{ref-James}{}
James, I.N., 1994. Introduction to circulating atmospheres. Cambridge
University Press.

\hypertarget{ref-Kalnay1996}{}
Kalnay, E., Kanamitsu, M., Kistler, R., Collins, W., Deaven, D., Gandin,
L., Iredell, M. et~al., 1996. The NCEP/NCAR 40-year reanalysis project.
77, 3, 437-471.

\hypertarget{ref-Karoly1985}{}
Karoly, D.J., 1985. An atmospheric climatology of the Southern
Hemisphere based on ten years of daily numerical analyses(1972-82). II-
Standing wave climatology. 33, 3, 105-116.

\hypertarget{ref-Karoly1989}{}
Karoly, D.J., 1989. Southern Hemisphere Circulation Features Associated
with El Nino-Southern Ocscillation Events. 2, 11, 1239-1252.

\hypertarget{ref-Kidson1988}{}
Kidson, J.W., 1988. Interannual Variations in the Southern Hemisphere
Circulation. 1, 12, 1177-1198.

\hypertarget{ref-King1997}{}
King, J.C., Turner, J., 1997. Antarctic Meteorology and Climatology.
Cambridge University Press.409 págs

\hypertarget{ref-Kinnard2011}{}
Kinnard, C., Zdanowicz, C.M., Fisher, D. a, Isaksson, E., Vernal, A. de,
Thompson, L.G., 2011. Reconstructed changes in Arctic sea ice over the
past 1,450 years. 479, 7374, 509-12.

\hypertarget{ref-Loon1972}{}
Loon, H. van, Jenne, Roy, L., 1972. The Zonal Harmonic Standing Waves in
the Southern Hemisphe. 77, 6, 992-1003.

\hypertarget{ref-Lorenz2001}{}
Lorenz, D.J., Hartmann, D.L., 2001. Eddy--Zonal Flow Feedback in the
Southern Hemisphere. 58, 21, 3312-3327.

\hypertarget{ref-R-circular}{}
Lund, U., Agostinelli, C., Arai, H., Gagliardi, A., Portugues, E.G.,
Giunchi, D., Irisson, J.-O. et~al., 2017. circular: Circular Statistics.
\url{https://CRAN.R-project.org/package=circular}

\hypertarget{ref-Mi2005}{}
Mi, X., Ren, H., Ouyang, Z., Wei, W., Ma, K., 2005. The use of the
Mexican Hat and the Morlet wavelets for detection of ecological
patterns. 179, 1, 1-19.

\hypertarget{ref-Mo1985}{}
Mo, K.C., White, G.H., 1985. Teleconnections in the Southern Hemisphere.
113, 1, 22-37.

\hypertarget{ref-Molteni2003}{}
Molteni, F., 2003. Atmospheric simulations using a GCM with simplified
physical parametrizations. I: model climatology and variability in
multi-decadal experiments. 20, 2, 175-191.

\hypertarget{ref-Pinault2016}{}
Pinault, J.L., 2016. Long Wave Resonance in Tropical Oceans and
Implications on Climate: The Pacific Ocean. 173, 6, 2119-2145.

\hypertarget{ref-Quintanar1995a}{}
Quintanar, A.I., Mechoso, C.R., 1995a. Quasi-stationary waves in the
Southern Hemisphere. Part I: observational data. 8, 11, 2659-2672.

\hypertarget{ref-Quintanar1995}{}
Quintanar, A.I., Mechoso, C.R., 1995b. Quasi-Stationary Waves in the
Southern Hemisphere. Part II: Generation Mechanisms. 8, 11, 2673-2690.

\hypertarget{ref-Rao2004}{}
Rao, V.B., Fernandez, J.P.R., Franchito, S.H., 2004. Quasi-stationary
waves in the southern hemisphere during El Nina and La Nina events. 22,
3, 789-806.

\hypertarget{ref-Raphael2004}{}
Raphael, M.N., 2004. A zonal wave 3 index for the Southern Hemisphere.
31, 23, 1-4.

\hypertarget{ref-Raphael1998}{}
Raphael, M.N., 1998. Quasi-stationary waves in the southern hemisphere:
an examination of their simulation by the NCAR climate system model,
with and without an interactive ocean. 11, 6, 1405-1419.

\hypertarget{ref-Raphael2003}{}
Raphael, M.N., 2003. Recent, Large-Scale Changes in the Extratropical
Southern Hemisphere Atmospheric Circulation. 16, 2915-2924.

\hypertarget{ref-Rayner2003}{}
Rayner, N.A., 2003. Global analyses of sea surface temperature, sea ice,
and night marine air temperature since the late nineteenth century. 108,
D14, 4407.

\hypertarget{ref-R-WaveletComp}{}
Roesch, A., Schmidbauer, H., 2014. WaveletComp: Computational Wavelet
Analysis. \url{https://CRAN.R-project.org/package=WaveletComp}

\hypertarget{ref-Silvestri2009}{}
Silvestri, G., Vera, C., 2009. Nonstationary impacts of the southern
annular mode on Southern Hemisphere climate. 22, 22, 6142-6148.

\hypertarget{ref-Simpson2013}{}
Simpson, I.R., Shepherd, T.G., Hitchcock, P., Scinocca, J.F., 2013.
Southern annular mode dynamics in observations and models. part ii: Eddy
Feedbacks. 26, 14, 5220-5241.

\hypertarget{ref-Torrence1998}{}
Torrence, C., Compo, G., 1998. A Practical Guide to Wavelet Analysis.
79, 61-78.

\hypertarget{ref-Trenberth1980a}{}
Trenberth, K.E., 1980. Planetary Waves at 500 mb in the Southern
Hemisphere. 108, 9, 1378-1389.

\hypertarget{ref-Trenberth1985}{}
Trenberth, K.E., Mo, K.C., 1985. Blocking in the Southern Hemisphere.
113, 1, 3-21.

\hypertarget{ref-Vera2006}{}
Vera, C., Higgins, W., Amador, J., Ambrizzi, T., Garreaud, R., Gochis,
D., Gutzler, D. et~al., 2006. Toward a unified view of the American
monsoon systems. 19, 20, 4977-5000.

\hypertarget{ref-Vera2004}{}
Vera, C., Silvestri, G., Barros, V., Carril, A., 2004. Differences in El
Niño response over the Southern Hemisphere. 17, 9, 1741-1753.

\hypertarget{ref-Wang2013}{}
Wang, L., Kushner, P.J., Waugh, D.W., 2013. Southern hemisphere
stationary wave response to changes of ozone and greenhouse gases. 26,
24, 10205-10217.

\hypertarget{ref-Wheeler2004}{}
Wheeler, M.C., Hendon, H.H., 2004. An All-Season Real-Time Multivariate
MJO Index: Development of an Index for Monitoring and Prediction. 132,
8, 1917-1932.

\hypertarget{ref-Yuan2008}{}
Yuan, X., Li, C., 2008. Climate modes in southern high latitudes and
their impacts on Antarctic sea ice. 113, 6, 1-13.

\hypertarget{ref-Zangl2001}{}
Zängl, G., 2001. The tropopause in the polar regions. 14, 14, 3117-3139.


\end{document}
